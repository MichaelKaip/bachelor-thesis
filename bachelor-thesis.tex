\documentclass[
fontsize=10pt, 
listof = totoc,
parskip = half	
]{report}
\usepackage[utf8]{inputenc}
\usepackage[german]{babel}
\usepackage[fixlanguage]{babelbib}
\selectbiblanguage{german}
\usepackage[dvipsnames, table]{xcolor}
\usepackage{colortbl}
\usepackage[most]{tcolorbox}
\usepackage[]{acronym}
\usepackage[T1]{fontenc}
\usepackage[absolute]{textpos}
\usepackage{amsmath}
\usepackage{amsthm}
\usepackage{amsfonts}
\usepackage[ruled,vlined,linesnumbered,ngerman]{algorithm2e}
\usepackage{tabu}

\usepackage{enumerate}
\usepackage{graphicx}
\usepackage{mdframed}
\usepackage{float}
\usepackage{lipsum}
\usepackage{fontawesome}  
\usepackage{esvect}
\usepackage{booktabs} % table style
\usepackage{tabularx}
\usepackage{multirow} % table style
\usepackage{amssymb}
\usepackage{nicefrac}
\usepackage{cancel}
\usepackage{polynom}
\usepackage{stmaryrd} 
\usepackage{caption}
\usepackage{subcaption}
\usepackage{paralist}
\usepackage{physics}
\usepackage{amsmath}
\usepackage{tikz}
\usepackage{mathdots}
\usepackage{yhmath}
\usepackage{cancel}
\usepackage{blindtext}

\usepackage[left=2.5cm,right=2.5cm,top=2cm,bottom=2cm]{geometry}
\usepackage{fancyhdr}
\pagestyle{fancy}
\renewcommand{\headrulewidth}{0.2pt}
\renewcommand{\footrulewidth}{0pt}

\usepackage{chngcntr}
%\counterwithout{figure}{chapter}
%\counterwithout{table}{chapter}


\usepackage[hidelinks, plainpages=false, pdfpagelabels]{hyperref}


\newenvironment{tableEnum}{\begin{varwidth}[t]{\linewidth}\begin{compactenum}[1)]}{\end{compactenum}\end{varwidth}\vspace{0.08cm}}
\newenvironment{tableItem}{\begin{varwidth}[t]{\linewidth}\begin{compactitem}[-]}{\end{compactitem}\end{varwidth}\vspace{0.08cm}}

\author{Michael Kaip}

\date{}

% Extension for amsmath matrix environment - matrix | vector
\makeatletter
\renewcommand*\env@matrix[1][*\c@MaxMatrixCols c]{%
	\hskip -\arraycolsep
	\let\@ifnextchar\new@ifnextchar
	\array{#1}}
\makeatother

%Extension for roman numbers
\newcommand{\uproman}[1]{\uppercase\expandafter{\romannumeral#1}}
\newcommand{\lowroman}[1]{\romannumeral#1\relax}

\newtheorem{definition}{Definition}

\newenvironment{conditions}
{\par\vspace{\abovedisplayskip}\noindent\begin{tabular}{>{$}l<{$} @{${}:{}$} l}}
	{\end{tabular}\par\vspace{\belowdisplayskip}}


\setcounter{tocdepth}{3}
\setcounter{secnumdepth}{3}


%%%%%%%%%%%%%%%%%%%%%%%%%%%%%%%%%%%%%%%%

\begin{document}
\begin{titlepage}
	\vspace*{-\headsep}\vspace{-\headheight}
	\noindent
	\includegraphics[scale=0.38]{logo}
	\hfill
	\textcolor{white}{placeholder}\\[-1ex]
	\rule{\linewidth}{1pt}
	
	\vfill\vfill
	
	\begin{center}
		BACHELORARBEIT
		\begin{huge}
			\\[2ex]
			\textbf{Untersuchung der Möglichkeiten zur \\ Entwicklung einer allgemeingültigen Anordnungsklassifikation \\ von Lamellengraphit}
			\\[6ex]
		\end{huge}
		Vorgelegt von:
		\\[2ex]
		\begin{huge}
			\textbf{Michael Kaip}
		\end{huge}
		\\[2ex]
		Studiengang Ingenieurinformatik
		\\[28ex]
		Erstgutachter:
		\\[2ex]
		\textbf{Prof. Dr.-Ing. Mohammad Abuosba}
		\\[4ex]
		Zweitgutachter:
		\\[2ex]
		\textbf{Dipl.-Mathematiker Ulrich Sonntag}
		\\[40ex]
		Berlin, den 12. Juli 2021
	\end{center}
\end{titlepage}
	
\clearpage

\begingroup
\pagestyle{empty}
\null
\newpage
\endgroup

\pagenumbering{gobble}

\chapter*{\centering Eidesstattliche Erklärung}

Ich versichere, dass ich die vorliegende Abschlussarbeit selbstständig und ohne unerlaubte Hilfe Dritter verfasst und keine anderen als die angegebenen Quellen und Hilfsmittel verwendet habe. Alle Stellen, die inhaltlich oder wörtlich aus Veröffentlichungen stammen, sind als solche kenntlich gemacht. Diese Arbeit lag in gleicher oder ähnlicher Weise noch keiner Prüfungsbehörde vor und wurde bisher nicht veröffentlicht.

\vspace{4cm}

\includegraphics[width=\textwidth, height=\textheight, keepaspectratio]{pics/signature}

\newpage\null\thispagestyle{empty}\newpage

\chapter*{\centering Danksagungen}

Vor allem möchte ich Herrn Prof. Dr.-Ing. Mohammad Abuosba von der HTW Berlin für die Betreuung meiner Bachelorarbeit als Erstgutachter danken, sowie Herrn Dipl.-Mathematiker Ulrich Sonntag von der Gesellschaft zur Förderung angewandter Informatik e.V., der mir die Möglichkeit gegeben hat, dieses spannende Thema im Rahmen meiner Bachelorarbeit zu bearbeiten und mich darüber hinaus auch ausgezeichnet dabei begleitet hat. 
\\\\
\noindent Mein Dank geht auch an Prof. Dr.-Ing. Jörg Schlingheider, Prof. Dr.-Ing. Frank Neumann, Prof. Dr. Nils Siebel, Prof. Dr. Frank Burghardt und Frau Prof. Dr. Christina Papenfuß von denen ich wirklich viel lernen konnte, um für meinen weiteren beruflichen Lebensweg bestens vorbereitet zu sein.
\\\\
\noindent Von ganzem Herzen möchte ich schließlich auch meiner lieben Frau Inna danken, die mich während meines Studiums in jeder Hinsicht unterstützt hat und auf die ich mich, auch in schwierigen Situation, immer verlassen kann.

\newpage\null\thispagestyle{empty} 

\newpage
\tableofcontents
\newpage
\pagenumbering{Roman}
\listoffigures
\addcontentsline{toc}{chapter}{Abbildungsverzeichnis}
\newpage
\listoftables
\addcontentsline{toc}{chapter}{Tabellenverzeichnis}
\newpage
\pagenumbering{arabic}
\newpage

\chapter{Einleitung}
\label{ch:Einleitung}

Bei Gusseisen handelt es sich um eine Eisen-Kohlenstoff-Legierung die, verglichen mit Stahl, einen wesentlich höheren Kohlenstoffgehalt von mehr als 2 bis 3,8 \% aufweist. Weitere Legierungsbestandteile sind Silicium und Mangan, womit durch sogenanntes Impfen oder der Veränderung der Abkühlgeschwindigkeit, die Werkstoffeigenschaften während des Herstellungsprozesses in großer Breite variiert werden können \cite{BDGuss02}. Aufgrund der hervorragenden gieß-technischen Eigenschaften des Werkstoffes (geringer Schmelzpunkt sowie dünnflüssige Schmelze) ergeben sich für den Konstrukteur außerdem hohe Freiheitsgrade in der Formgebung, wodurch eine besonders wirtschaftliche Fertigung ermöglicht wird. Der sich daraus ergebende immense Bedarf an Gusserzeugnissen wird besonders deutlich, wenn man sich die weltweite Gussproduktion anschaut, die sich beispielsweise im Jahr 2019 auf über 98 Millionen Tonnen belief. Der größte Produzent ist China mit 48,75 und Deutschland liegt mit 4,95 Millionen Tonnen auf Platz 5 \cite{Statista2021}.
\\\\
Der bei Herstellung des Werkstoffes zugeführte Kohlenstoff führt zu Graphiteinlagerungen, deren Form- und Gefügeausbildung sich durch die Schmelz- und Abkühlungsbedingungen bei der Herstellung ganz wesentlich beeinflussen lassen. Demnach unterscheidet man grundsätzlich die in Abbildung \ref{fig:GraphitTypes} dargestellten Arten von Gusseisen:

\begin{figure}[h]
	\begin{subfigure}{1.0\textwidth}
		\begin{subfigure}{0.33\textwidth}
			\centering
			\includegraphics[scale=0.25]{pics/graphit_form1}
			\label{fig:LamellenGraphit}
			\caption*{\textbf{Form \uproman{1}}\\Lamellengraphit}
		\end{subfigure}\hfill
		\begin{subfigure}{0.33\textwidth}
			\centering
			\includegraphics[scale=0.25]{pics/graphit_form2}
			\caption*{\textbf{Form \uproman{2}}\\Krabbengraphit}
		\end{subfigure}\hfill
		\begin{subfigure}{0.33\textwidth}
			\centering
			\includegraphics[scale=0.25]{pics/graphit_form3}
			\caption*{\textbf{Form \uproman{3}}\\Vermiculargraphit}
		\end{subfigure}
	\end{subfigure}
	\\\\\\
	\begin{subfigure}{1.0\linewidth}
		\begin{subfigure}[t]{0.33\textwidth}
			\centering
			\includegraphics[scale=0.25]{pics/graphit_form4}
			\caption*{\textbf{Form \uproman{4}}\\ungleichf. \\ Kugelgraphit}
		\end{subfigure}
		\begin{subfigure}[t]{0.33\textwidth}
			\centering
			\includegraphics[scale=0.25]{pics/graphit_form5}
			\caption*{\textbf{Form \uproman{5}}\\gering ungleichf.\\Kugelgraphit }
		\end{subfigure}
		\begin{subfigure}[t]{0.33\textwidth}
			\centering
			\includegraphics[scale=0.25]{pics/graphit_form6}
			\caption*{\textbf{Form \uproman{6}}\\Kugelgraphit}
		\end{subfigure}
	\end{subfigure}
	\caption{Typen von Gusseisen nach Form (\uproman{1} - \uproman{6}) der Graphitpartikelausbildung \cite[S.7]{ISO945}}
	\label{fig:GraphitTypes}
\end{figure}

\noindent Doch die Gusseisenwerkstoffe unterscheiden sich nicht nur hinsichtlich der Form und Struktur ihrer Graphiteinlagerungen, sondern vor allem auch in Bezug auf ihre damit unmittelbar in Zusammenhang stehenden mechanischen Eigenschaften, wie Zug- und Druckfestigkeit, dem Elastizitätsmodul, der Scherfestigkeit und weiterer Werkstoffkenndaten die, in Verbindung mit der Konstruktionsgeometrie die Qualität eines solchen Erzeugnisses ganz wesentlich beeinflussen. Das bedeutet, dass bei Kräfteeinwirkung je nach Art unterschiedliche Spannungskonzentrationen in den Graphiteinschlüssen entstehen und sich Gusseisenwerkstoffe unterschiedlicher Formklassen daher mehr oder weniger für bestimmte Verwendungszwecke eignen können. 
\\\\
\noindent Die wichtigsten Abnehmer von Gusseisenwerkstoffen sind der Straßenfahrzeugbau mit fast 60~\% sowie der Maschinenbau mit 25 bis 30~\% der gesamten Gusslieferungen \cite{BDGuss01}. Die Qualitätsanforderungen in diesen Branchen sind allgemein sehr hoch, jedoch wird gleichzeitig der  durch Substitution hervorgerufene Kostendruck auf die Hersteller stetig erhöht und zwingt diese zu Kostensenkungen im Bereich der Herstellkosten, bei gleichzeitiger Beibehaltung der Qualität. Ein weiterer Aspekt, der vor diesem Hintergrund für die Verwendung von Gusseisenwerkstoffen (vor allem Gusseisen mit Lamellengraphit) spricht, ist die Möglichkeit, diesen Zielkonflikt auf elegante Art und Weise zu lösen. Und zwar deshalb, weil der Kohlenstoffgehalt von 3 bis 4 Masseprozent im Gefüge zu wesentlichen Teilen als ausgeschiedener Graphit vorliegt und wegen der unterschiedlichen Dichte von Eisen und Graphit, 3,5 Masseprozent Kohlenstoff bis zu 11 Volumenprozent Graphit entsprechen können. Im Vergleich zu Stahl resultiert daraus eine immense Gewichts- und Kosteneinsparung, wobei die Unterschiede in den mechanischen Eigenschaften im Bedarfsfall durch entsprechende Werkstoffzugaben gewichtsneutral eliminiert werden können \cite{BDGuss02}.
\\\\
Die zur Beurteilung der Qualität von Gusseisenerzeugnissen maßgebliche Norm ist die DIN EN ISO 945-1. Genauso wie, je nach Art der Graphiteinlagerungen, eine grundsätzliche Einteilung in unterschiedliche Formklassen vorgenommen wird (siehe Abb. \ref{fig:GraphitTypes}), werden Gusseisenwerkstoffe mit Lamellengraphit zusätzlich in unterschiedliche Anordnungsklassen eingeteilt (siehe dazu auch Kapitel \ref{subsec:MethodenBestAnordnungsklassen} auf Seite \pageref{subsec:MethodenBestAnordnungsklassen}).
\\\\
Die Klassifizierung und Auswertung einer vorliegenden Werkstoffprobe erfolgt gemäß dieser Norm jedoch durch visuelle Auswertung, die von einer fachkundigen und entsprechend geschulten Fachkraft durchzuführen ist. Diese Vorgehensweise hat sich in der Praxis zwar über viele Jahre bewährt, ist aber dennoch fehleranfällig. Das ist sicherlich einer der Gründe dafür, weshalb moderne software-basierte Verfahren zur quantitativen Gefügeanalyse sich zu einem festen Bestandteil der heutigen metallographischen Praxis entwickelt haben, wenn auch diese Entwicklung noch lange nicht als abgeschlossen gelten kann \cite{schumann_oettel_2016}.
\\\\
Vor diesem Hintergrund ist auch die Software AMGuss entstanden, die von Gießereien zur schnellen und effizienten quantitativen Analyse der Mikrostruktur von Gusseisenwerkstoffen eingesetzt wird (vgl. dazu auch Kapitel \ref{sec:BestimmungMikrostrukturAMGuss}). Die Software  unterstützt  damit den Metallographen bei der Umsetzung der in \cite{ISO945} spezifizierten Anforderungen zur Analyse der Mikrostruktur von Gusseisenwerkstoffen und leistet somit einen wertvollen Beitrag zur Qualitätssicherung in Eisengießereien. Während das Programm Funktionalitäten zur Auswertung aller Formklassen (siehe Abb. \ref{fig:GraphitTypes}) anbietet, beschäftigt sich die vorliegend Arbeit ausschließlich mit Typ \uproman{1}, also der Analyse von Gusseisen mit Lamellengraphit oder GJL, wie der Werkstoff nach DIN EN 1561 auch bezeichnet wird. 




\chapter{Grundlagen}
\label{ch:Grundlagen}

\section{Klassifikation und Analyse von Gusseisenwerkstoffen mit Lamellengraphit}
\label{subsec:MethodenBestAnordnungsklassen}
Werden Gusseisenwerkstoffe mit Lamellengraphit nach DIN EN ISO 945-1 untersucht, müssen die darin enthaltenen Graphitbestandteile nach dem Bezeichnungssystem für die Klassifizierung von Graphit in Gusseisen klassifiziert  werden \cite[Seite 6]{ISO945}. Dazu bietet die Norm dem Metallographen sogenannte Richtreihenbilder an, anhand derer eine solche Klassifizierung durch visuelle Beurteilung durchzuführen ist.
\\\\
In der Formklasse Lamellengraphit (vgl. Abbildung \ref{fig:GraphitTypes} auf Seite \pageref{fig:GraphitTypes}) werden je nach Lage der Lamellen zueinander 5 verschiedene Anordnungsklassen (A - E) unterschiedenen, die in der folgenden Abbildung \ref{fig:Anordnungsklassen} dargestellt sind.

\begin{figure}[H]
	\centering
	\includegraphics[scale=0.5]{pics/Anordnungsklassen}
	\caption{Richtreihenbilder für die Graphitanordnung\\ (Form \uproman{1} / Lamellengraphit) \cite{ISO945}}
	\label{fig:Anordnungsklassen}
\end{figure}

\noindent Diese Richtreihenbilder dienen also den entsprechenden Fachleuten in metallographischen Laboren dazu, die in einer Probe vorliegende morphologische Graphitstruktur durch visuellen Vergleich zu bestimmen. Somit ist das Ergebnis einer solchen Bestimmung niemals eindeutig, sondern hängt zwangsläufig von der subjektiven Einschätzung Metallographen ab \cite{ISO945}.
\\\\
\noindent Das hat insbesondere zur Folge, dass es eine Diskrepanz in der Beurteilung des Gefüges zwischen Lieferant und Abnehmer geben kann und dann Qualitätssicherungsvorgaben nicht eindeutig interpretiert werden.

\section{Bildskalierung und Interpolationsverfahren}
\label{subsec:SkalierungUndInterpolation}

Unter Bildskalierung versteht man die Vergrößerung bzw. Verkleinerung von Bildern. Es handelt sich dabei um ein Standardverfahren, welches sowohl im Grafik- und Fotobereich als auch in der digitalen Bildverarbeitung sehr häufig zum Einsatz kommt und von Bildverarbeitungsprogrammen standardmäßig angeboten wird. Da die algorithmische Funktionsweise einer Bildskalierung für das Verständnis der in dieser Arbeit zu lösenden Problemstellung (siehe auch Kapitel \ref{sec:Problemstellung} auf Seite \pageref{sec:Problemstellung}) von zentraler Bedeutung ist, wird diese nun hier genauer Untersucht und beschrieben.
\\\\
Das grundsätzliche Problem bei der Skalierung von Bildern soll durch Abbildung \ref{fig:ScalingProblem} verdeutlicht werden. Zu sehen ist links ein Bild mit ($4\times 4$) Pixeln und rechts das durch Skalierung erzeugte Bild. Die neue Position eines beliebigen Originalpixels im neu erzeugten (skalierten) Bild ergibt sich durch Multiplikation der ursprünglichen Position mit dem Skalierungsfaktor ($\lambda$).
\begin{figure}[H]
	\centering
		\begin{tikzpicture}[x=0.75pt,y=0.75pt,yscale=-1,xscale=1, scale=1.2, every node/.style={scale=0.9}]
			%uncomment if require: \path (0,300); %set diagram left start at 0, 	and has height of 300
			
			%Shape: Rectangle [id:dp33493272980948086] 
			\draw  [color={rgb, 255:red, 255; green, 255; blue, 255 }  ,draw opacity=1 ][fill={rgb, 255:red, 155; green, 155; blue, 155 }  ,fill opacity=1 ] (240,53.85) -- (400,53.85) -- (400,213.85) -- (240,213.85) -- cycle ;
			%Shape: Grid [id:dp5166847032058127] 
			\draw  [draw opacity=0][fill={rgb, 255:red, 126; green, 211; blue, 33 }  ,fill opacity=1 ][line width=1.5]  (35.38,93.85) -- (116.38,93.85) -- (116.38,174.85) -- (35.38,174.85) -- cycle ; \draw  [line width=1.5]  (35.38,93.85) -- (35.38,174.85)(55.38,93.85) -- (55.38,174.85)(75.38,93.85) -- (75.38,174.85)(95.38,93.85) -- (95.38,174.85)(115.38,93.85) -- (115.38,174.85) ; \draw  [line width=1.5]  (35.38,93.85) -- (116.38,93.85)(35.38,113.85) -- (116.38,113.85)(35.38,133.85) -- (116.38,133.85)(35.38,153.85) -- (116.38,153.85)(35.38,173.85) -- (116.38,173.85) ; \draw  [line width=1.5]   ;
			%Right Arrow [id:dp7963578241503881] 
			\draw  [line width=1.5]  (139.38,125) -- (172.98,125) -- (172.98,116) -- (195.38,134) -- (172.98,152) -- (172.98,143) -- (139.38,143) -- cycle ;
			%Shape: Grid [id:dp813346916234577] 
			\draw  [draw opacity=0][line width=1.5]  (240,53.85) -- (400.38,53.85) -- (400.38,214.85) -- (240,214.85) -- cycle ; \draw  [color={rgb, 255:red, 0; green, 0; blue, 0 }  ,draw opacity=1 ][line width=1.5]  (240,53.85) -- (240,214.85)(260,53.85) -- (260,214.85)(280,53.85) -- (280,214.85)(300,53.85) -- (300,214.85)(320,53.85) -- (320,214.85)(340,53.85) -- (340,214.85)(360,53.85) -- (360,214.85)(380,53.85) -- (380,214.85)(400,53.85) -- (400,214.85) ; \draw  [color={rgb, 255:red, 0; green, 0; blue, 0 }  ,draw opacity=1 ][line width=1.5]  (240,53.85) -- (400.38,53.85)(240,73.85) -- (400.38,73.85)(240,93.85) -- (400.38,93.85)(240,113.85) -- (400.38,113.85)(240,133.85) -- (400.38,133.85)(240,153.85) -- (400.38,153.85)(240,173.85) -- (400.38,173.85)(240,193.85) -- (400.38,193.85)(240,213.85) -- (400.38,213.85) ; \draw  [color={rgb, 255:red, 0; green, 0; blue, 0 }  ,draw opacity=1 ][line width=1.5]   ;
			%Shape: Rectangle [id:dp7852221432702648] 
			\draw  [fill={rgb, 255:red, 126; green, 211; blue, 33 }  ,fill opacity=1 ][line width=1.5]  (240,53.85) -- (260,53.85) -- (260,73.85) -- (240,73.85) -- cycle ;
			%Shape: Rectangle [id:dp316493263782738] 
			\draw  [fill={rgb, 255:red, 126; green, 211; blue, 33 }  ,fill opacity=1 ][line width=1.5]  (360,53.85) -- (380,53.85) -- (380,73.85) -- (360,73.85) -- cycle ;
			%Shape: Rectangle [id:dp9150879931583693] 
			\draw  [fill={rgb, 255:red, 126; green, 211; blue, 33 }  ,fill opacity=1 ][line width=1.5]  (280,53.85) -- (300,53.85) -- (300,73.85) -- (280,73.85) -- cycle ;
			%Shape: Rectangle [id:dp21254213121480348] 
			\draw  [fill={rgb, 255:red, 126; green, 211; blue, 33 }  ,fill opacity=1 ][line width=1.5]  (320,53.85) -- (340,53.85) -- (340,73.85) -- (320,73.85) -- cycle ;
			%Shape: Rectangle [id:dp9138059724854111] 
			\draw  [fill={rgb, 255:red, 126; green, 211; blue, 33 }  ,fill opacity=1 ][line width=1.5]  (240,93.85) -- (260,93.85) -- (260,113.85) -- (240,113.85) -- cycle ;
			%Shape: Rectangle [id:dp8385734790799879] 
			\draw  [fill={rgb, 255:red, 126; green, 211; blue, 33 }  ,fill opacity=1 ][line width=1.5]  (360,93.85) -- (380,93.85) -- (380,113.85) -- (360,113.85) -- cycle ;
			%Shape: Rectangle [id:dp0266783762407925] 
			\draw  [fill={rgb, 255:red, 126; green, 211; blue, 33 }  ,fill opacity=1 ][line width=1.5]  (280,93.85) -- (300,93.85) -- (300,113.85) -- (280,113.85) -- cycle ;
			%Shape: Rectangle [id:dp8765432738424186] 
			\draw  [fill={rgb, 255:red, 126; green, 211; blue, 33 }  ,fill opacity=1 ][line width=1.5]  (320,93.85) -- (340,93.85) -- (340,113.85) -- (320,113.85) -- cycle ;
			%Shape: Rectangle [id:dp3683961028684497] 
			\draw  [fill={rgb, 255:red, 126; green, 211; blue, 33 }  ,fill opacity=1 ][line width=1.5]  (240,133.85) -- (260,133.85) -- (260,153.85) -- (240,153.85) -- cycle ;
			%Shape: Rectangle [id:dp17023134792901806] 
			\draw  [fill={rgb, 255:red, 126; green, 211; blue, 33 }  ,fill opacity=1 ][line width=1.5]  (360,133.85) -- (380,133.85) -- (380,153.85) -- (360,153.85) -- cycle ;
			%Shape: Rectangle [id:dp30885139351862023] 
			\draw  [fill={rgb, 255:red, 126; green, 211; blue, 33 }  ,fill opacity=1 ][line width=1.5]  (280,133.85) -- (300,133.85) -- (300,153.85) -- (280,153.85) -- cycle ;
			%Shape: Rectangle [id:dp2063497064425942] 
			\draw  [fill={rgb, 255:red, 126; green, 211; blue, 33 }  ,fill opacity=1 ][line width=1.5]  (320,133.85) -- (340,133.85) -- (340,153.85) -- (320,153.85) -- cycle ;
			%Shape: Rectangle [id:dp7893916008859733] 
			\draw  [fill={rgb, 255:red, 126; green, 211; blue, 33 }  ,fill opacity=1 ][line width=1.5]  (240,173.85) -- (260,173.85) -- (260,193.85) -- (240,193.85) -- cycle ;
			%Shape: Rectangle [id:dp03560102427440881] 
			\draw  [fill={rgb, 255:red, 126; green, 211; blue, 33 }  ,fill opacity=1 ][line width=1.5]  (360,173.85) -- (380,173.85) -- (380,193.85) -- (360,193.85) -- cycle ;
			%Shape: Rectangle [id:dp5475282364153556] 
			\draw  [fill={rgb, 255:red, 126; green, 211; blue, 33 }  ,fill opacity=1 ][line width=1.5]  (280,173.85) -- (300,173.85) -- (300,193.85) -- (280,193.85) -- cycle ;
			%Shape: Rectangle [id:dp34756519033919797] 
			\draw  [fill={rgb, 255:red, 126; green, 211; blue, 33 }  ,fill opacity=1 ][line width=1.5]  (320,173.85) -- (340,173.85) -- (340,193.85) -- (320,193.85) -- cycle ;
			%Straight Lines [id:da27131202915211117] 
			\draw [color={rgb, 255:red, 0; green, 0; blue, 0 }  ,draw opacity=1 ][line width=1.5]    (17.38,69.85) -- (46,69.85) -- (88,69.85) -- (142.38,69.85) ;
			\draw [shift={(145.38,69.85)}, rotate = 180] [color={rgb, 255:red, 0; green, 0; blue, 0 }  ,draw opacity=1 ][line width=1.5]    (14.21,-4.28) .. controls (9.04,-1.82) and (4.3,-0.39) .. (0,0) .. controls (4.3,0.39) and (9.04,1.82) .. (14.21,4.28)   ;
			%Straight Lines [id:da0003577742395167727] 
			\draw [color={rgb, 255:red, 0; green, 0; blue, 0 }  ,draw opacity=1 ][line width=1.5]    (17.38,69.85) -- (17.38,197.85) ;
			\draw [shift={(17.38,200.85)}, rotate = 270] [color={rgb, 255:red, 0; green, 0; blue, 0 }  ,draw opacity=1 ][line width=1.5]    (14.21,-4.28) .. controls (9.04,-1.82) and (4.3,-0.39) .. (0,0) .. controls (4.3,0.39) and (9.04,1.82) .. (14.21,4.28)   ;
			%Straight Lines [id:da8167500493181958] 
			\draw [fill={rgb, 255:red, 0; green, 0; blue, 0 }  ,fill opacity=1 ][line width=1.5]    (218.38,26.85) -- (446.38,26.85) ;
			\draw [shift={(449.38,26.85)}, rotate = 180] [color={rgb, 255:red, 0; green, 0; blue, 0 }  ][line width=1.5]    (14.21,-4.28) .. controls (9.04,-1.82) and (4.3,-0.39) .. (0,0) .. controls (4.3,0.39) and (9.04,1.82) .. (14.21,4.28)   ;
			%Straight Lines [id:da5628538454694642] 
			\draw [color={rgb, 255:red, 0; green, 0; blue, 0 }  ,draw opacity=1 ][line width=1.5]    (218.38,26.85) -- (218.38,243.85) ;
			\draw [shift={(218.38,246.85)}, rotate = 270] [color={rgb, 255:red, 0; green, 0; blue, 0 }  ,draw opacity=1 ][line width=1.5]    (14.21,-4.28) .. controls (9.04,-1.82) and (4.3,-0.39) .. (0,0) .. controls (4.3,0.39) and (9.04,1.82) .. (14.21,4.28)   ;
			
			% Text Node
			\draw (143,129) node [anchor=north west][inner sep=0.75pt]  [font=\normalsize,xslant=0.03] [align=left] {$\displaystyle \lambda $\textbf{ = 2}};
			% Text Node
			\draw (122,168) node [anchor=north west][inner sep=0.75pt]   [align=left] {{\footnotesize \textcolor[rgb]{0,0,0}{$\displaystyle \lambda $}\textcolor[rgb]{0,0,0}{\textbf{: Skalierungsfaktor}}}};
			% Text Node
			\draw (23,96) node [anchor=north west][inner sep=0.75pt]   [align=left] {0};
			% Text Node
			\draw (23,116) node [anchor=north west][inner sep=0.75pt]   [align=left] {1};
			% Text Node
			\draw (23,136) node [anchor=north west][inner sep=0.75pt]   [align=left] {2};
			% Text Node
			\draw (23,157) node [anchor=north west][inner sep=0.75pt]   [align=left] {3};
			% Text Node
			\draw (40,78) node [anchor=north west][inner sep=0.75pt]   [align=left] {0};
			% Text Node
			\draw (60,78) node [anchor=north west][inner sep=0.75pt]   [align=left] {1};
			% Text Node
			\draw (80,78) node [anchor=north west][inner sep=0.75pt]   [align=left] {2};
			% Text Node
			\draw (101,78) node [anchor=north west][inner sep=0.75pt]   [align=left] {3};
			% Text Node
			\draw (155,67) node [anchor=north west][inner sep=0.75pt]   [align=left] {\textbf{x}};
			% Text Node
			\draw (13,210) node [anchor=north west][inner sep=0.75pt]   [align=left] {\textbf{y}};
			% Text Node
			\draw (244,37) node [anchor=north west][inner sep=0.75pt]   [align=left] {0};
			% Text Node
			\draw (264,37) node [anchor=north west][inner sep=0.75pt]   [align=left] {1};
			% Text Node
			\draw (284,37) node [anchor=north west][inner sep=0.75pt]   [align=left] {2};
			% Text Node
			\draw (305,37) node [anchor=north west][inner sep=0.75pt]   [align=left] {3};
			% Text Node
			\draw (325,37) node [anchor=north west][inner sep=0.75pt]   [align=left] {4};
			% Text Node
			\draw (345,37) node [anchor=north west][inner sep=0.75pt]   [align=left] {5};
			% Text Node
			\draw (365,37) node [anchor=north west][inner sep=0.75pt]   [align=left] {6};
			% Text Node
			\draw (384,37) node [anchor=north west][inner sep=0.75pt]   [align=left] {7};
			% Text Node
			\draw (227,56) node [anchor=north west][inner sep=0.75pt]   [align=left] {0};
			% Text Node
			\draw (227,76) node [anchor=north west][inner sep=0.75pt]   [align=left] {1};
			% Text Node
			\draw (227,96) node [anchor=north west][inner sep=0.75pt]   [align=left] {2};
			% Text Node
			\draw (227,117) node [anchor=north west][inner sep=0.75pt]   [align=left] {3};
			% Text Node
			\draw (227,137) node [anchor=north west][inner sep=0.75pt]   [align=left] {4};
			% Text Node
			\draw (227,157) node [anchor=north west][inner sep=0.75pt]   [align=left] {5};
			% Text Node
			\draw (227,176) node [anchor=north west][inner sep=0.75pt]   [align=left] {6};
			% Text Node
			\draw (227,197) node [anchor=north west][inner sep=0.75pt]   [align=left] {7};
			% Text Node
			\draw (458,24) node [anchor=north west][inner sep=0.75pt]   [align=left] {\textbf{x}};
			% Text Node
			\draw (214,255) node [anchor=north west][inner sep=0.75pt]   [align=left] {\textbf{y}};
		\end{tikzpicture}
	\caption{Darstellung der Notwendigkeit von\\ Interpolationsverfahren bei der Bildskalierung}
	\label{fig:ScalingProblem}
\end{figure}

\noindent Genauer ausgedrückt handelt es sich bei den Lücken um undefinierte Pixel, da sich die Pixelmenge proportional zum Skalierungsfaktor ebenfalls verändert. Bei Skalierungsfaktoren kleiner 1 verhält es sich genau umgekehrt und die Anzahl der Pixel im Ergebnisbild verringert sich entsprechend.
\\\\
\noindent Zur Lösung dieses Problems wurden verschiedene Interpolationsalgorithmen entwickelt. Diese verfolgen alle das Ziel, die sich ergebende Differenz der Pixelanzahl auszugleichen und die Bildqualität dadurch möglichst gut zu erhalten. Dabei werden also die Farbwerte der grauen (undefinierten) Pixel aus den Farbwerten der umliegenden Pixel auf jeweils unterschiedliche Art approximiert. Es liegt jedoch auf der Hand, das ein solchen Verfahren die Details eines gegebenen Originalbildes nie 100\%-ig erhalten kann. So können zwar für viele praktische Anwendungsfälle im Grafik- oder Fotobereich sehr gute Ergebnisse erzielt werden, weil das menschliche Auge diese Abweichungen nicht erkennen kann. Allerdings erweist sich die Interpolation im Bereich der bildbasierten Materialstrukturanalyse als problematisch, da sich  die zu analysierenden Strukturen durch die Interpolation verändern und somit die Messergebnisse verfälscht werden. 
\\\\
\noindent Dieser Zusammenhang wird nun anhand der bilinearen Interpolation einmal genauer erklärt, damit sich der Leser ein besseres Bild davon machen kann. Dabei handelt es sich um ein sehr einfaches Verfahren, bei dem zur Bestimmung unbestimmter Pixel nach der Skalierung, deren Farbwerte (RGB) durch lineare Interpolation in X- und Y-Richtung berechnet werden. Als Beispiel dient folgende Situation, die in Abbildung \ref{fig:BilinearInterpolation1} dargestellt ist.

\begin{figure}[H]
	\centering
	\begin{tikzpicture}[x=0.75pt,y=0.75pt,yscale=-1,xscale=1, scale=1.2, every node/.style={scale=1.2}]
		%uncomment if require: \path (0,374); %set diagram left start at 0, and has height of 374
		
		%Shape: Grid [id:dp6031282918603735] 
		\draw  [draw opacity=0][line width=1.5]  (300.38,50.85) -- (481.38,50.85) -- (481.38,231.85) -- (300.38,231.85) -- cycle ; \draw  [color={rgb, 255:red, 0; green, 0; blue, 0 }  ,draw opacity=1 ][line width=1.5]  (300.38,50.85) -- (300.38,231.85)(330.38,50.85) -- (330.38,231.85)(360.38,50.85) -- (360.38,231.85)(390.38,50.85) -- (390.38,231.85)(420.38,50.85) -- (420.38,231.85)(450.38,50.85) -- (450.38,231.85)(480.38,50.85) -- (480.38,231.85) ; \draw  [color={rgb, 255:red, 0; green, 0; blue, 0 }  ,draw opacity=1 ][line width=1.5]  (300.38,50.85) -- (481.38,50.85)(300.38,80.85) -- (481.38,80.85)(300.38,110.85) -- (481.38,110.85)(300.38,140.85) -- (481.38,140.85)(300.38,170.85) -- (481.38,170.85)(300.38,200.85) -- (481.38,200.85)(300.38,230.85) -- (481.38,230.85) ; \draw  [color={rgb, 255:red, 0; green, 0; blue, 0 }  ,draw opacity=1 ][line width=1.5]   ;
		%Shape: Axis 2D [id:dp12421444962330219] 
		\draw  (283.56,4.16) -- (283.56,260.85)(506.38,29.83) -- (258.81,29.83) (288.56,253.85) -- (283.56,260.85) -- (278.56,253.85) (499.38,24.83) -- (506.38,29.83) -- (499.38,34.83)  ;
		%Shape: Axis 2D [id:dp5582126406856516] 
		\draw  (43.86,14.16) -- (43.86,166.85)(188.38,29.43) -- (27.81,29.43) (48.86,159.85) -- (43.86,166.85) -- (38.86,159.85) (181.38,24.43) -- (188.38,29.43) -- (181.38,34.43)  ;
		%Shape: Grid [id:dp6426300130896345] 
		\draw  [draw opacity=0][line width=2.25]  (62,51) -- (152.38,51) -- (152.38,141.85) -- (62,141.85) -- cycle ; \draw  [line width=2.25]  (62,51) -- (62,141.85)(92,51) -- (92,141.85)(122,51) -- (122,141.85)(152,51) -- (152,141.85) ; \draw  [line width=2.25]  (62,51) -- (152.38,51)(62,81) -- (152.38,81)(62,111) -- (152.38,111)(62,141) -- (152.38,141) ; \draw  [line width=2.25]   ;
		%Shape: Rectangle [id:dp13871600656081773] 
		\draw  [fill={rgb, 255:red, 245; green, 166; blue, 35 }  ,fill opacity=1 ][line width=1.5]  (92,51) -- (122,51) -- (122,81) -- (92,81) -- cycle ;
		%Shape: Rectangle [id:dp09914050619225745] 
		\draw  [fill={rgb, 255:red, 245; green, 166; blue, 35 }  ,fill opacity=1 ][line width=1.5]  (122,81) -- (152,81) -- (152,111) -- (122,111) -- cycle ;
		%Shape: Rectangle [id:dp7354425170625416] 
		\draw  [fill={rgb, 255:red, 245; green, 166; blue, 35 }  ,fill opacity=1 ][line width=1.5]  (122,111) -- (152,111) -- (152,141) -- (122,141) -- cycle ;
		%Shape: Rectangle [id:dp8022069903462297] 
		\draw  [fill={rgb, 255:red, 245; green, 166; blue, 35 }  ,fill opacity=1 ][line width=1.5]  (62,111) -- (92,111) -- (92,141) -- (62,141) -- cycle ;
		%Shape: Rectangle [id:dp5675136532554508] 
		\draw  [fill={rgb, 255:red, 245; green, 166; blue, 35 }  ,fill opacity=1 ][line width=1.5]  (92,111) -- (122,111) -- (122,141) -- (92,141) -- cycle ;
		%Shape: Rectangle [id:dp26546030668596743] 
		\draw  [fill={rgb, 255:red, 126; green, 211; blue, 33 }  ,fill opacity=1 ][line width=1.5]  (122,51) -- (152,51) -- (152,81) -- (122,81) -- cycle ;
		%Shape: Rectangle [id:dp31022250660119444] 
		\draw  [fill={rgb, 255:red, 126; green, 211; blue, 33 }  ,fill opacity=1 ][line width=1.5]  (92,81) -- (122,81) -- (122,111) -- (92,111) -- cycle ;
		%Shape: Rectangle [id:dp1632256531496281] 
		\draw  [fill={rgb, 255:red, 126; green, 211; blue, 33 }  ,fill opacity=1 ][line width=1.5]  (62,51) -- (92,51) -- (92,81) -- (62,81) -- cycle ;
		%Shape: Rectangle [id:dp36586491405776467] 
		\draw  [fill={rgb, 255:red, 126; green, 211; blue, 33 }  ,fill opacity=1 ][line width=1.5]  (62,81) -- (92,81) -- (92,111) -- (62,111) -- cycle ;
		%Shape: Rectangle [id:dp09483486001396602] 
		\draw  [fill={rgb, 255:red, 245; green, 166; blue, 35 }  ,fill opacity=1 ][line width=1.5]  (360.38,50.85) -- (390.38,50.85) -- (390.38,80.85) -- (360.38,80.85) -- cycle ;
		%Shape: Rectangle [id:dp8722790379045016] 
		\draw  [fill={rgb, 255:red, 245; green, 166; blue, 35 }  ,fill opacity=1 ][line width=1.5]  (420.38,110.85) -- (450.38,110.85) -- (450.38,140.85) -- (420.38,140.85) -- cycle ;
		%Shape: Rectangle [id:dp3090077325311579] 
		\draw  [fill={rgb, 255:red, 245; green, 166; blue, 35 }  ,fill opacity=1 ][line width=1.5]  (420.38,170.85) -- (450.38,170.85) -- (450.38,200.85) -- (420.38,200.85) -- cycle ;
		%Shape: Rectangle [id:dp39265423932322685] 
		\draw  [fill={rgb, 255:red, 245; green, 166; blue, 35 }  ,fill opacity=1 ][line width=1.5]  (300.38,170.85) -- (330.38,170.85) -- (330.38,200.85) -- (300.38,200.85) -- cycle ;
		%Shape: Rectangle [id:dp309527828572457] 
		\draw  [fill={rgb, 255:red, 245; green, 166; blue, 35 }  ,fill opacity=1 ][line width=1.5]  (360.38,170.85) -- (390.38,170.85) -- (390.38,200.85) -- (360.38,200.85) -- cycle ;
		%Shape: Rectangle [id:dp43382954399056994] 
		\draw  [fill={rgb, 255:red, 126; green, 211; blue, 33 }  ,fill opacity=1 ][line width=1.5]  (420.38,50.85) -- (450.38,50.85) -- (450.38,80.85) -- (420.38,80.85) -- cycle ;
		%Shape: Rectangle [id:dp5113204315472576] 
		\draw  [fill={rgb, 255:red, 126; green, 211; blue, 33 }  ,fill opacity=1 ][line width=1.5]  (360.38,110.85) -- (390.38,110.85) -- (390.38,140.85) -- (360.38,140.85) -- cycle ;
		%Shape: Rectangle [id:dp06279947117329954] 
		\draw  [fill={rgb, 255:red, 126; green, 211; blue, 33 }  ,fill opacity=1 ][line width=1.5]  (300.38,50.85) -- (330.38,50.85) -- (330.38,80.85) -- (300.38,80.85) -- cycle ;
		%Shape: Rectangle [id:dp7495389450561212] 
		\draw  [fill={rgb, 255:red, 126; green, 211; blue, 33 }  ,fill opacity=1 ][line width=1.5]  (300.38,110.85) -- (330.38,110.85) -- (330.38,140.85) -- (300.38,140.85) -- cycle ;
		%Shape: Rectangle [id:dp3179677452191165] 
		\draw  [color={rgb, 255:red, 0; green, 0; blue, 0 }  ,draw opacity=1 ][fill={rgb, 255:red, 185; green, 188; blue, 34 }  ,fill opacity=1 ][line width=1.5]  (330.38,50.85) -- (360.38,50.85) -- (360.38,80.85) -- (330.38,80.85) -- cycle ;
		%Shape: Rectangle [id:dp02520564590269525] 
		\draw  [color={rgb, 255:red, 0; green, 0; blue, 0 }  ,draw opacity=1 ][fill={rgb, 255:red, 185; green, 188; blue, 34 }  ,fill opacity=1 ][line width=1.5]  (390.38,50.85) -- (420.38,50.85) -- (420.38,80.85) -- (390.38,80.85) -- cycle ;
		%Shape: Rectangle [id:dp41456521365377197] 
		\draw  [color={rgb, 255:red, 0; green, 0; blue, 0 }  ,draw opacity=1 ][fill={rgb, 255:red, 185; green, 188; blue, 34 }  ,fill opacity=1 ][line width=1.5]  (360.38,80.85) -- (390.38,80.85) -- (390.38,110.85) -- (360.38,110.85) -- cycle ;
		%Shape: Rectangle [id:dp7715421317421746] 
		\draw  [color={rgb, 255:red, 0; green, 0; blue, 0 }  ,draw opacity=1 ][fill={rgb, 255:red, 185; green, 188; blue, 34 }  ,fill opacity=1 ][line width=1.5]  (420.38,80.85) -- (450.38,80.85) -- (450.38,110.85) -- (420.38,110.85) -- cycle ;
		%Shape: Rectangle [id:dp6537803831672764] 
		\draw  [color={rgb, 255:red, 0; green, 0; blue, 0 }  ,draw opacity=1 ][fill={rgb, 255:red, 185; green, 188; blue, 34 }  ,fill opacity=1 ][line width=1.5]  (390.38,110.85) -- (420.38,110.85) -- (420.38,140.85) -- (390.38,140.85) -- cycle ;
		%Shape: Rectangle [id:dp694492919060651] 
		\draw  [color={rgb, 255:red, 0; green, 0; blue, 0 }  ,draw opacity=1 ][fill={rgb, 255:red, 185; green, 188; blue, 34 }  ,fill opacity=1 ][line width=1.5]  (390.38,80.85) -- (420.38,80.85) -- (420.38,110.85) -- (390.38,110.85) -- cycle ;
		%Shape: Rectangle [id:dp8749183811880095] 
		\draw  [color={rgb, 255:red, 0; green, 0; blue, 0 }  ,draw opacity=1 ][fill={rgb, 255:red, 185; green, 188; blue, 34 }  ,fill opacity=1 ][line width=1.5]  (300.38,140.85) -- (330.38,140.85) -- (330.38,170.85) -- (300.38,170.85) -- cycle ;
		%Shape: Rectangle [id:dp15019771324119735] 
		\draw  [color={rgb, 255:red, 0; green, 0; blue, 0 }  ,draw opacity=1 ][fill={rgb, 255:red, 185; green, 188; blue, 34 }  ,fill opacity=1 ][line width=1.5]  (360.38,140.85) -- (390.38,140.85) -- (390.38,170.85) -- (360.38,170.85) -- cycle ;
		%Shape: Rectangle [id:dp8886413553682353] 
		\draw  [color={rgb, 255:red, 0; green, 0; blue, 0 }  ,draw opacity=1 ][fill={rgb, 255:red, 185; green, 188; blue, 34 }  ,fill opacity=1 ][line width=1.5]  (330.38,140.85) -- (360.38,140.85) -- (360.38,170.85) -- (330.38,170.85) -- cycle ;
		%Shape: Rectangle [id:dp18893376895528624] 
		\draw  [fill={rgb, 255:red, 245; green, 166; blue, 35 }  ,fill opacity=1 ][line width=1.5]  (420.38,140.85) -- (450.38,140.85) -- (450.38,170.85) -- (420.38,170.85) -- cycle ;
		%Shape: Rectangle [id:dp8401086030387102] 
		\draw  [fill={rgb, 255:red, 245; green, 166; blue, 35 }  ,fill opacity=1 ][line width=1.5]  (390.38,170.85) -- (420.38,170.85) -- (420.38,200.85) -- (390.38,200.85) -- cycle ;
		%Shape: Rectangle [id:dp5023664828243175] 
		\draw  [fill={rgb, 255:red, 245; green, 166; blue, 35 }  ,fill opacity=1 ][line width=1.5]  (330.38,170.85) -- (360.38,170.85) -- (360.38,200.85) -- (330.38,200.85) -- cycle ;
		%Shape: Rectangle [id:dp4926347479761316] 
		\draw  [fill={rgb, 255:red, 126; green, 211; blue, 33 }  ,fill opacity=1 ][line width=1.5]  (300.38,80.85) -- (330.38,80.85) -- (330.38,110.85) -- (300.38,110.85) -- cycle ;
		%Shape: Rectangle [id:dp03314094385373201] 
		\draw  [fill={rgb, 255:red, 126; green, 211; blue, 33 }  ,fill opacity=1 ][line width=1.5]  (330.38,110.85) -- (360.38,110.85) -- (360.38,140.85) -- (330.38,140.85) -- cycle ;
		%Shape: Rectangle [id:dp31700461022402926] 
		\draw  [color={rgb, 255:red, 0; green, 0; blue, 0 }  ,draw opacity=1 ][fill={rgb, 255:red, 155; green, 199; blue, 33 }  ,fill opacity=1 ][line width=1.5]  (330.38,80.85) -- (360.38,80.85) -- (360.38,110.85) -- (330.38,110.85) -- cycle ;
		%Shape: Rectangle [id:dp17649729390737434] 
		\draw  [color={rgb, 255:red, 0; green, 0; blue, 0 }  ,draw opacity=1 ][fill={rgb, 255:red, 215; green, 177; blue, 34 }  ,fill opacity=1 ][line width=1.5]  (390.38,140.85) -- (420.38,140.85) -- (420.38,170.85) -- (390.38,170.85) -- cycle ;
		%Right Arrow [id:dp33270781304422914] 
		\draw  [line width=1.5]  (185,86) -- (227,86) -- (227,76) -- (255,96) -- (227,116) -- (227,106) -- (185,106) -- cycle ;
		%Shape: Rectangle [id:dp24983247014857224] 
		\draw  [fill={rgb, 255:red, 126; green, 211; blue, 33 }  ,fill opacity=1 ][line width=1.5]  (450.38,50.85) -- (480.38,50.85) -- (480.38,80.85) -- (450.38,80.85) -- cycle ;
		%Shape: Rectangle [id:dp35849526261121356] 
		\draw  [fill={rgb, 255:red, 245; green, 166; blue, 35 }  ,fill opacity=1 ][line width=1.5]  (450.38,110.85) -- (480.38,110.85) -- (480.38,140.85) -- (450.38,140.85) -- cycle ;
		%Shape: Rectangle [id:dp1351111821765797] 
		\draw  [fill={rgb, 255:red, 245; green, 166; blue, 35 }  ,fill opacity=1 ][line width=1.5]  (450.38,170.85) -- (480.38,170.85) -- (480.38,200.85) -- (450.38,200.85) -- cycle ;
		%Shape: Rectangle [id:dp5617796085771564] 
		\draw  [color={rgb, 255:red, 0; green, 0; blue, 0 }  ,draw opacity=1 ][fill={rgb, 255:red, 185; green, 188; blue, 34 }  ,fill opacity=1 ][line width=1.5]  (450.38,80.85) -- (480.38,80.85) -- (480.38,110.85) -- (450.38,110.85) -- cycle ;
		%Shape: Rectangle [id:dp6849626334604124] 
		\draw  [fill={rgb, 255:red, 245; green, 166; blue, 35 }  ,fill opacity=1 ][line width=1.5]  (450.38,140.85) -- (480.38,140.85) -- (480.38,170.85) -- (450.38,170.85) -- cycle ;
		%Shape: Rectangle [id:dp4423643524339804] 
		\draw  [fill={rgb, 255:red, 245; green, 166; blue, 35 }  ,fill opacity=1 ][line width=1.5]  (420.38,200.85) -- (450.38,200.85) -- (450.38,230.85) -- (420.38,230.85) -- cycle ;
		%Shape: Rectangle [id:dp9844308475991588] 
		\draw  [fill={rgb, 255:red, 245; green, 166; blue, 35 }  ,fill opacity=1 ][line width=1.5]  (300.38,200.85) -- (330.38,200.85) -- (330.38,230.85) -- (300.38,230.85) -- cycle ;
		%Shape: Rectangle [id:dp19330105393002595] 
		\draw  [fill={rgb, 255:red, 245; green, 166; blue, 35 }  ,fill opacity=1 ][line width=1.5]  (360.38,200.85) -- (390.38,200.85) -- (390.38,230.85) -- (360.38,230.85) -- cycle ;
		%Shape: Rectangle [id:dp719400766119336] 
		\draw  [fill={rgb, 255:red, 245; green, 166; blue, 35 }  ,fill opacity=1 ][line width=1.5]  (390.38,200.85) -- (420.38,200.85) -- (420.38,230.85) -- (390.38,230.85) -- cycle ;
		%Shape: Rectangle [id:dp3277584194542974] 
		\draw  [fill={rgb, 255:red, 245; green, 166; blue, 35 }  ,fill opacity=1 ][line width=1.5]  (330.38,200.85) -- (360.38,200.85) -- (360.38,230.85) -- (330.38,230.85) -- cycle ;
		%Shape: Rectangle [id:dp5673130666239078] 
		\draw  [fill={rgb, 255:red, 245; green, 166; blue, 35 }  ,fill opacity=1 ][line width=1.5]  (450.38,200.85) -- (480.38,200.85) -- (480.38,230.85) -- (450.38,230.85) -- cycle ;
		%Left Right Arrow [id:dp10603987349893162] 
		\draw  [color={rgb, 255:red, 255; green, 255; blue, 255 }  ,draw opacity=1 ] (334.19,65.85) -- (339.79,58.43) -- (339.79,62.14) -- (350.98,62.14) -- (350.98,58.43) -- (356.57,65.85) -- (350.98,73.28) -- (350.98,69.56) -- (339.79,69.56) -- (339.79,73.28) -- cycle ;
		%Left Right Arrow [id:dp4851771954641114] 
		\draw  [color={rgb, 255:red, 255; green, 255; blue, 255 }  ,draw opacity=1 ] (394.19,65.85) -- (399.79,58.43) -- (399.79,62.14) -- (410.98,62.14) -- (410.98,58.43) -- (416.57,65.85) -- (410.98,73.28) -- (410.98,69.56) -- (399.79,69.56) -- (399.79,73.28) -- cycle ;
		%Left Right Arrow [id:dp46535515779583203] 
		\draw  [color={rgb, 255:red, 255; green, 255; blue, 255 }  ,draw opacity=1 ] (334.19,185.85) -- (339.79,178.43) -- (339.79,182.14) -- (350.98,182.14) -- (350.98,178.43) -- (356.57,185.85) -- (350.98,193.28) -- (350.98,189.56) -- (339.79,189.56) -- (339.79,193.28) -- cycle ;
		%Left Right Arrow [id:dp0388460015786245] 
		\draw  [color={rgb, 255:red, 255; green, 255; blue, 255 }  ,draw opacity=1 ] (394.19,185.85) -- (399.79,178.43) -- (399.79,182.14) -- (410.98,182.14) -- (410.98,178.43) -- (416.57,185.85) -- (410.98,193.28) -- (410.98,189.56) -- (399.79,189.56) -- (399.79,193.28) -- cycle ;
		%Left Right Arrow [id:dp7225211430319548] 
		\draw  [color={rgb, 255:red, 255; green, 255; blue, 255 }  ,draw opacity=1 ] (334.19,125.85) -- (339.79,118.43) -- (339.79,122.14) -- (350.98,122.14) -- (350.98,118.43) -- (356.57,125.85) -- (350.98,133.28) -- (350.98,129.56) -- (339.79,129.56) -- (339.79,133.28) -- cycle ;
		%Left Right Arrow [id:dp4516242307309136] 
		\draw  [color={rgb, 255:red, 255; green, 255; blue, 255 }  ,draw opacity=1 ] (394.19,125.85) -- (399.79,118.43) -- (399.79,122.14) -- (410.98,122.14) -- (410.98,118.43) -- (416.57,125.85) -- (410.98,133.28) -- (410.98,129.56) -- (399.79,129.56) -- (399.79,133.28) -- cycle ;
		%Left Right Arrow [id:dp2240592597082971] 
		\draw  [color={rgb, 255:red, 255; green, 255; blue, 255 }  ,draw opacity=1 ] (315.4,84.66) -- (322.82,90.26) -- (319.1,90.26) -- (319.09,101.45) -- (322.8,101.45) -- (315.37,107.04) -- (307.95,101.44) -- (311.66,101.44) -- (311.68,90.25) -- (307.97,90.25) -- cycle ;
		%Left Right Arrow [id:dp526728801760665] 
		\draw  [color={rgb, 255:red, 255; green, 255; blue, 255 }  ,draw opacity=1 ] (315.4,144.66) -- (322.82,150.26) -- (319.1,150.26) -- (319.09,161.45) -- (322.8,161.45) -- (315.37,167.04) -- (307.95,161.44) -- (311.66,161.44) -- (311.68,150.25) -- (307.97,150.25) -- cycle ;
		%Left Right Arrow [id:dp5265171418673112] 
		\draw  [color={rgb, 255:red, 255; green, 255; blue, 255 }  ,draw opacity=1 ] (435.4,84.66) -- (442.82,90.26) -- (439.1,90.26) -- (439.09,101.45) -- (442.8,101.45) -- (435.37,107.04) -- (427.95,101.44) -- (431.66,101.44) -- (431.68,90.25) -- (427.97,90.25) -- cycle ;
		%Left Right Arrow [id:dp30991370393310413] 
		\draw  [color={rgb, 255:red, 255; green, 255; blue, 255 }  ,draw opacity=1 ] (435.4,144.66) -- (442.82,150.26) -- (439.1,150.26) -- (439.09,161.45) -- (442.8,161.45) -- (435.37,167.04) -- (427.95,161.44) -- (431.66,161.44) -- (431.68,150.25) -- (427.97,150.25) -- cycle ;
		%Left Right Arrow [id:dp38972009417085385] 
		\draw  [color={rgb, 255:red, 255; green, 255; blue, 255 }  ,draw opacity=1 ] (374.4,84.66) -- (381.82,90.26) -- (378.1,90.26) -- (378.09,101.45) -- (381.8,101.45) -- (374.37,107.04) -- (366.95,101.44) -- (370.66,101.44) -- (370.68,90.25) -- (366.97,90.25) -- cycle ;
		%Left Right Arrow [id:dp7707043216642973] 
		\draw  [color={rgb, 255:red, 255; green, 255; blue, 255 }  ,draw opacity=1 ] (375.4,144.66) -- (382.82,150.26) -- (379.1,150.26) -- (379.09,161.45) -- (382.8,161.45) -- (375.37,167.04) -- (367.95,161.44) -- (371.66,161.44) -- (371.68,150.25) -- (367.97,150.25) -- cycle ;

		
		% Text Node
		\draw (262,156.99) node [anchor=north west][inner sep=0.75pt]  [font=\normalsize,rotate=-270.08] [align=left] {$\displaystyle \lambda \cdot \ y$};
		% Text Node
		\draw (372.56,11) node [anchor=north west][inner sep=0.75pt]  [font=\normalsize,rotate=-359.4] [align=left] {\textcolor[rgb]{0,0,0}{$\displaystyle \lambda \cdot \ x$}};
		% Text Node
		\draw (309,35) node [anchor=north west][inner sep=0.75pt]   [align=left] {0};
		% Text Node
		\draw (340,35) node [anchor=north west][inner sep=0.75pt]   [align=left] {1};
		% Text Node
		\draw (371,35) node [anchor=north west][inner sep=0.75pt]   [align=left] {2};
		% Text Node
		\draw (398,35) node [anchor=north west][inner sep=0.75pt]   [align=left] {3};
		% Text Node
		\draw (431,35) node [anchor=north west][inner sep=0.75pt]   [align=left] {4};
		% Text Node
		\draw (288,58) node [anchor=north west][inner sep=0.75pt]   [align=left] {0};
		% Text Node
		\draw (288,89) node [anchor=north west][inner sep=0.75pt]   [align=left] {1};
		% Text Node
		\draw (288,119) node [anchor=north west][inner sep=0.75pt]   [align=left] {2};
		% Text Node
		\draw (288,148) node [anchor=north west][inner sep=0.75pt]   [align=left] {3};
		% Text Node
		\draw (288,179) node [anchor=north west][inner sep=0.75pt]   [align=left] {4};
		% Text Node
		\draw (510,25.5) node [anchor=north west][inner sep=0.75pt]   [align=left] {x};
		% Text Node
		\draw (40,170) node [anchor=north west][inner sep=0.75pt]   [align=left] {y};
		% Text Node
		\draw (71,35) node [anchor=north west][inner sep=0.75pt]   [align=left] {0};
		% Text Node
		\draw (102,35) node [anchor=north west][inner sep=0.75pt]   [align=left] {1};
		% Text Node
		\draw (133,35) node [anchor=north west][inner sep=0.75pt]   [align=left] {2};
		% Text Node
		\draw (49,58) node [anchor=north west][inner sep=0.75pt]   [align=left] {0};
		% Text Node
		\draw (49,89) node [anchor=north west][inner sep=0.75pt]   [align=left] {1};
		% Text Node
		\draw (49,119) node [anchor=north west][inner sep=0.75pt]   [align=left] {2};
		% Text Node
		\draw (192,25.5) node [anchor=north west][inner sep=0.75pt]   [align=left] {x};
		% Text Node
		\draw (279,265) node [anchor=north west][inner sep=0.75pt]   [align=left] {y};
		% Text Node
		\draw (458,35) node [anchor=north west][inner sep=0.75pt]   [align=left] {5};
		% Text Node
		\draw (288,209) node [anchor=north west][inner sep=0.75pt]   [align=left] {5};
		% Text Node
		\draw (62,60) node [anchor=north west][inner sep=0.75pt]  [font=\scriptsize] [align=left] {$\displaystyle P_{( 0,0)}$};
		% Text Node
		\draw (93,90) node [anchor=north west][inner sep=0.75pt]  [font=\scriptsize] [align=left] {$\displaystyle P_{( 1,1)}$};
		% Text Node
		\draw (62,121) node [anchor=north west][inner sep=0.75pt]  [font=\scriptsize] [align=left] {$\displaystyle P_{( 2,0)}$};
		% Text Node
		\draw (123,121) node [anchor=north west][inner sep=0.75pt]  [font=\scriptsize] [align=left] {$\displaystyle P_{( 2,2)}$};
		% Text Node
		\draw (93,121) node [anchor=north west][inner sep=0.75pt]  [font=\scriptsize] [align=left] {$\displaystyle P_{( 2,1)}$};
		% Text Node
		\draw (123,90) node [anchor=north west][inner sep=0.75pt]  [font=\scriptsize] [align=left] {$\displaystyle P_{( 1,2)}$};
		% Text Node
		\draw (62,90) node [anchor=north west][inner sep=0.75pt]  [font=\scriptsize] [align=left] {$\displaystyle P_{( 1,0)}$};
		% Text Node
		\draw (123,60) node [anchor=north west][inner sep=0.75pt]  [font=\scriptsize] [align=left] {$\displaystyle P_{( 0,2)}$};
		% Text Node
		\draw (93,60) node [anchor=north west][inner sep=0.75pt]  [font=\scriptsize] [align=left] {$\displaystyle P_{( 0,1)}$};
		% Text Node
		\draw (301,60) node [anchor=north west][inner sep=0.75pt]  [font=\scriptsize] [align=left] {$\displaystyle P_{( 0,0)}$};
		% Text Node
		\draw (361,60) node [anchor=north west][inner sep=0.75pt]  [font=\scriptsize] [align=left] {$\displaystyle P_{( 0,2)}$};
		% Text Node
		\draw (421,60) node [anchor=north west][inner sep=0.75pt]  [font=\scriptsize] [align=left] {$\displaystyle P_{( 0,4)}$};
		% Text Node
		\draw (301,120) node [anchor=north west][inner sep=0.75pt]  [font=\scriptsize] [align=left] {$\displaystyle P_{( 2,0)}$};
		% Text Node
		\draw (361,120) node [anchor=north west][inner sep=0.75pt]  [font=\scriptsize] [align=left] {$\displaystyle P_{( 2,2)}$};
		% Text Node
		\draw (421,120) node [anchor=north west][inner sep=0.75pt]  [font=\scriptsize] [align=left] {$\displaystyle P_{( 2,4)}$};
		% Text Node
		\draw (301,180) node [anchor=north west][inner sep=0.75pt]  [font=\scriptsize] [align=left] {$\displaystyle P_{( 4,0)}$};
		% Text Node
		\draw (361,180) node [anchor=north west][inner sep=0.75pt]  [font=\scriptsize] [align=left] {$\displaystyle P_{( 4,2)}$};
		% Text Node
		\draw (421,180) node [anchor=north west][inner sep=0.75pt]  [font=\scriptsize] [align=left] {$\displaystyle P_{( 4,4)}$};
		% Text Node
		\draw (192,90) node [anchor=north west][inner sep=0.75pt]   [align=left] {$\displaystyle \lambda =2$};
		% Text Node
		\draw (339,88) node [anchor=north west][inner sep=0.75pt]   [align=left] {$\displaystyle \dot{P}$};
	\end{tikzpicture}
	\caption{Lineare Interpolation (Beispiel)}
	\label{fig:BilinearInterpolation1}
\end{figure}
\noindent Die hier im Bild (links) dargestellten RGB-Farbwerte betragen für die grünen $(126, 211, 33)$ und für die orangenen Pixel $(245, 166, 35)$. Das rechte Bild repräsentiert das Ergebnisbild nach Skalierung des Ausgangsbildes (links) um den Faktor 2. Die Farbwerte der Pixel aus dem Ausgangsbild wurden unverändert in das Ergebnisbild übernommen, jedoch veränderte sich deren Position in Abhängigkeit vom Skalierungsfaktor $\lambda$. Diese Pixel sind im Ergebnisbild an der Beschriftung $P_{(m,n)}$ zu erkennen. Die Farbwerte aller anderen Pixel müssen berechnet werden, wie im Bild rechts bereits geschehen (Pfeilrichtung = Interpolationsrichtung). Ein Algorithmus zur Berechnung der fehlenden Pixel würde dann also, je nach konkreter Implementierung in etwa folgendermaßen arbeiten (Algorithmus \ref{alg:BiLinPseudo}):
\\\\
\begin{algorithm}[H]
	\SetKwInOut{Input}{input}\SetKwInOut{Output}{output}
	\BlankLine
	\Input{ImageOriginal, $\lambda$}
	\Output{ImageScaled}
	\BlankLine
	\emph{Create new pixel array to store the scaled image}\;
	ImageScaled[~][~][~] $\gets$ [ImageOriginal.Height $\times\lambda$] [ImageOriginal.width $\times\lambda$][3]\\
	\BlankLine
	\For{$i\leftarrow 0$ \KwTo ImageScaled.Height}{
		\For{$j\leftarrow 0$ \KwTo ImageScaled.Width}{
			\BlankLine
			\If{$i\mod\lambda == 0$ and $j\mod\lambda == 0$}{
				\BlankLine
				\emph{Set pixel from the original image (as they are)}\;
				ImageScaled[i][j] $\gets$ ImageOriginal[i][j]
			}
			\Else{
				ImageScaled[i][j] $\gets$ \textbf{CalculatePixelValues(ImageOriginal, i, j, $\mathbf{\lambda}$)}
			}
		}
		\BlankLine
		\Return{ImageScaled}
	}
	\BlankLine
	\caption{Pseudocode zur algorithmischen Beschreibung bilinearer Interpolation bei Bildskalierung}
	\label{alg:BiLinPseudo}
\end{algorithm}
\bigskip
\noindent Beispielsweise lässt sich der Farbwert des Pixels $\dot{P}$  im Ergebnisbild (Abbildung \ref{fig:BilinearInterpolation1}, oberer linker Quadrant) wie folgt berechnen:
\\\\
\noindent Zunächst werden die Farbwerte der Pixel $P_1$ und $P_2$ zwischen $P_{0,0}$ und $P_{0,2}$ bzw. $P_{2,0}$ und $P_{2,2}$ durch Interpolation in X-Richtung berechnet:
\begin{equation}
	\begin{split}
		P_1	&=	\frac{1}{2} \left(P_{0,0} + P_{0,2}\right) =
				\frac{1}{2}
				\left[
				\begin{pmatrix}
					126\\
					211\\
					33
				\end{pmatrix}
				+
				\begin{pmatrix}
					245\\
					166\\
					35
				\end{pmatrix}
				\right]
				=
				\begin{pmatrix}
					185,5\\
					188,5\\
					34
				\end{pmatrix} \\\\
		P_2	&=	\frac{1}{2} \left(P_{2,0} + P_{2,2}\right) =
				\begin{pmatrix}
					126\\
					211\\
					33
				\end{pmatrix}
	\end{split}
\end{equation}

\noindent Aus den Ergebnissen (2.1) werden dann die Farbwerte des Pixels $\dot{P}$ auf analoge Art und Weise ebenfalls durch lineare Interpolation in Y-Richtung berechnet:

\begin{equation}
	\dot{P} =	\frac{1}{2} \cdot \left(P_1 + P_2\right) =
				\frac{1}{2}\cdot
				\left[
				\begin{pmatrix}
					185,5\\
					188,5\\
					34
				\end{pmatrix}
				+
				\begin{pmatrix}
					126\\
					211\\
					33
				\end{pmatrix}
				\right]
				=
				\begin{pmatrix}
					155,75\\
					199,75\\
					33,5
				\end{pmatrix}
\end{equation}
\\\\
\noindent Um zu zeigen, wie sich bilineare Interpolation bei Skalierung auf ein reales Bild auswirkt, wurde ein Ausgangsbild mit einer Größe von $640\times 426~px$ mit $\lambda = 20$ skaliert. Dabei wurde der Skalierungsfaktor absichtlich so groß gewählt, damit die Effekte für das menschliche Auge sichtbar und hier entsprechend dargestellt werden können Das Ergebnis ist in Abbildung \ref{fig:MosaicBilinear} (b) dargestellt.
\\
\begin{figure}[H]
	\begin{subfigure}[t]{0.49\textwidth}
		\centering
		\boxed{\includegraphics[scale=1.3]{pics/mosaic.jpg}}
		\caption{Mosaikbild (www.pixabay.com,\\ Fotograf: Michael Gaida)}
	\end{subfigure}\hfill
	\begin{subfigure}[t]{0.49\textwidth}
		\centering
		\boxed{\includegraphics[scale=1.3]{pics/mosaic-zoom-20.jpg}}
		\caption{Mosaikbild bei 20-facher Vergrößerung}
	\end{subfigure}
	\caption{Bildskalierung unter Anwendung bilinearer Interpolation}
	\label{fig:MosaicBilinear}
\end{figure}
\noindent Wie zu erkennen, führt die bilineare Interpolation dazu, das Kanten sehr unscharf werden und neue Farbverläufe entstehen. Im Bereich der Materialstrukturanalyse, wie später in Kapitel \ref{sec:Problemstellung} auf Seite \pageref{sec:Problemstellung} noch gezeigt wird, kann das (auch bei wesentlich kleineren Skalierungsfaktoren) zu sehr unerwünschten Einflüssen führen und Messergebnisse verfälschen.
\\\\
\noindent Abgesehen von diesen sehr einfachen Beispiel der bilinearen Interpolation gibt es jedoch auch noch einige weitere Interpolationsverfahren, bei denen die Modelle zur Approximation der fehlenden Bildinformationen auf wesentlich komplexere Art und Weise berechnet werden. Im Rahmen dieser Arbeit kommen, außer der bereits beschriebenen linearen Interpolation, die folgenden Interpolationsverfahren zum Einsatz:
\begin{minipage}[t]{0.45\linewidth}\vspace{0pt}
	\begin{itemize}
		\item \textbf{Nächster Nachbar}\\
		Dabei wird jedem neu zu berechnenden Pixel der Farbwert des Pixels zugewiesen,, der die geringste euklidische Distanz zum betrachteten Pixel hat.  
		\item \textbf{Nächster Nachbar Exak}t\\
		Dabei handelt es sich um eine bit-exakte Nächster-Nachbar-Interpol.
		\item \textbf{Linear Exakt}\\
		Dabei handelt es sich um eine bit-exakte lineare Interpolation
	\end{itemize}
\end{minipage}~~~~
\begin{minipage}[t]{0.45\linewidth}\vspace{0pt}
	\begin{itemize}
		\item \textbf{Kubisch}\\
		Bei der kubischen Interpolation wird über einen Ausschnitt von 4 $\times$ 4 Pixeln interpoliert.
		\item \textbf{Flächen-basiert}\\
		Hierbei werden Pixelflächenbeziehungen zur Berechnung fehlender Pixel verwendet.
		\item \textbf{Lanczos4}\\
		Dabei wird über einen Ausschnitt von 8 $\times$ 8 Pixeln interpoliert.
	\end{itemize}
\end{minipage}
\\\\\\
\noindent Je komplexer die Algorithmen, umso genauer arbeiten sie natürlich, aber umso größer ist die Laufzeitkomplexität. Es hängt also vom konkreten Anwendungsfall ab, welcher Algorithmus zu bevorzugen ist.
\\\\
\noindent In den Kapiteln \ref{ch:Konzept} (ab Seite \pageref{ch:Konzept}) und \ref{ch:Umsetzung} (ab Seite \pageref{ch:Umsetzung}) wird das Thema wieder eine bedeutende Rolle spielen wenn es darum geht, all die benötigten Bilder mit unterschiedlichen Ausgangskalibrierungen zu erzeugen und auszuwerten, um letztlich die zentrale mit dieser Arbeit verbundenen Fragestellung beantworten zu können.

\section{Statistische Versuchsplanung}
\label{sec:StatVersPlanung}
Statistische Versuchsplanung wird häufig in der Industrie eingesetzt, um Prozesse zu optimieren. Folgende Gründe können dafür ausschlaggebend sein \cite{kleppmann_2020}:

\begin{itemize}
	\item Der Funktionsumfang der Produkte muss erhöht werden, um die Anforderungen der Kunden immer besser erfüllen zu können.
	\item Die Kosten müssen gesenkt werden, z.B. durch geringere Materialkosten oder höhere
	Ausbeute.
	\item Die Entwicklungszeit neuer Produkte und ihre Durchlaufzeit in der Fertigung müssen
	immer weiter verkürzt werden.
\end{itemize}

\noindent Diese Verbesserungen können nicht allein durch Analyse von Daten aus der Fertigung und kritisches Nachdenken erreicht werden. Dazu sind die Zusammenhänge in Entwicklung, Fertigung und Qualitätsmanagement zu kompliziert und vielschichtig. Um den Einfluss von Designänderungen auf die Eigenschaften eines neuen Produktes oder den Einfluss von Änderungen von Prozessparametern auf das Prozessergebnis zu bestimmen,
sind gezielte Versuche notwendig \cite{kleppmann_2020}.
\\\\
\noindent Diese werden mit dem Ziel durchgeführt, die komplexen Ursache-/Wirkungszusammenhänge zwischen einer oder mehrerer Stör-/ Steuergrößen und verschiedener Zielgrößen untersuchen und verstehen zu können. Es geht also darum sichtbar zumachen, welche Auswirkungen die Veränderung der Einflussgrößen auf ein bestimmtes Ergebnis haben. Dadurch wird eine gezielte Ergebnisbeeinflussung ermöglicht. Das Verhältnis zwischen Eingabe und Ausgabe (Ergebnis/Zielgröße) wird auch als Prozess bezeichnet. Zur Veranschaulichung soll die nachstehende Abbildung \ref{fig:ProzessDoe} dienen.

\begin{figure}[H]
	\centering
	\begin{tikzpicture}[x=0.75pt,y=0.75pt,yscale=-1,xscale=1, scale=1.0, every node/.style={scale=1.0}]
		%uncomment if require: \path (0,222); %set diagram left start at 0, and has height of 222
		
		%Shape: Rectangle [id:dp9587460691430016] 
		\draw  [color={rgb, 255:red, 74; green, 74; blue, 74 }  ,draw opacity=1 ][fill={rgb, 255:red, 255; green, 255; blue, 255 }  ,fill opacity=1 ] (80,60) -- (564,60) -- (564,280) -- (80,280) -- cycle ;
		%Shape: Rectangle [id:dp8844267253059725] 
		\draw  [color={rgb, 255:red, 248; green, 231; blue, 28 }  ,draw opacity=1 ][fill={rgb, 255:red, 248; green, 231; blue, 28 }  ,fill opacity=1 ] (220,172) -- (420,172) -- (420,252) -- (220,252) -- cycle ;
		%Straight Lines [id:da8411829174990819] 
		\draw [color={rgb, 255:red, 126; green, 211; blue, 33 }  ,draw opacity=1 ][line width=4.5]    (273,128) -- (273,160) ;
		\draw [shift={(273,168)}, rotate = 270] [fill={rgb, 255:red, 126; green, 211; blue, 33 }  ,fill opacity=1 ][line width=0.08]  [draw opacity=0] (24.11,-11.58) -- (0,0) -- (24.11,11.58) -- cycle    ;
		%Straight Lines [id:da7471836258073241] 
		\draw [color={rgb, 255:red, 126; green, 211; blue, 33 }  ,draw opacity=1 ][line width=4.5]    (374,128) -- (374,160) ;
		\draw [shift={(374,168)}, rotate = 270] [fill={rgb, 255:red, 126; green, 211; blue, 33 }  ,fill opacity=1 ][line width=0.08]  [draw opacity=0] (24.11,-11.58) -- (0,0) -- (24.11,11.58) -- cycle    ;
		%Notched Right Arrow [id:dp19574990057369157] 
		\draw  [color={rgb, 255:red, 126; green, 211; blue, 33 }  ,draw opacity=1 ][fill={rgb, 255:red, 126; green, 211; blue, 33 }  ,fill opacity=1 ] (432,195.75) -- (502.2,195.75) -- (502.2,181) -- (549,210.5) -- (502.2,240) -- (502.2,225.25) -- (432,225.25) -- (446.75,210.5) -- cycle ;
		%Notched Right Arrow [id:dp2222830717379538] 
		\draw  [color={rgb, 255:red, 126; green, 211; blue, 33 }  ,draw opacity=1 ][fill={rgb, 255:red, 126; green, 211; blue, 33 }  ,fill opacity=1 ] (93,195.75) -- (163.2,195.75) -- (163.2,181) -- (210,210.5) -- (163.2,240) -- (163.2,225.25) -- (93,225.25) -- (107.75,210.5) -- cycle ;
		
		% Text Node
		\draw (243.06,198.75) node [anchor=north west][inner sep=0.75pt]  [color={rgb, 255:red, 74; green, 74; blue, 74 }  ,opacity=1 ,rotate=-0.19] [align=left] {\begin{minipage}[lt]{111.75pt}\setlength\topsep{0pt}
				\begin{center}
					\textbf{{\large PROZESS}}\\{\scriptsize \textbf{\textit{Ursache-/Wirgungsbez.}}}
				\end{center}
				
		\end{minipage}};
		% Text Node
		\draw (322,76) node [anchor=north west][inner sep=0.75pt]  [color={rgb, 255:red, 74; green, 74; blue, 74 }  ,opacity=1 ] [align=left] {\begin{minipage}[lt]{72.94pt}\setlength\topsep{0pt}
				\begin{center}
					\textbf{Störgrößen}\\{\scriptsize \textbf{\textit{(nicht beeinflussbar)}}}\\
				\end{center}
				
		\end{minipage}};
		% Text Node
		\draw (222,76) node [anchor=north west][inner sep=0.75pt]  [color={rgb, 255:red, 74; green, 74; blue, 74 }  ,opacity=1 ] [align=left] {\begin{minipage}[lt]{69.03pt}\setlength\topsep{0pt}
				\begin{center}
					\textbf{Steuergrößen}\\{\scriptsize \textbf{\textit{(beinfussbar)}}}
				\end{center}
				
		\end{minipage}};
		% Text Node
		\draw (452,201) node [anchor=north west][inner sep=0.75pt]  [font=\scriptsize,color={rgb, 255:red, 74; green, 74; blue, 74 }  ,opacity=1 ] [align=left] {\textbf{AUSGABE/}\\\textbf{ERGEBNIS}};
		% Text Node
		\draw (113,206) node [anchor=north west][inner sep=0.75pt]  [font=\scriptsize,color={rgb, 255:red, 74; green, 74; blue, 74 }  ,opacity=1 ] [align=left] {\textbf{EINGABE}};
	\end{tikzpicture}
	\caption{Prozessmodell zur statistischen Versuchsplanung}
	\label{fig:ProzessDoe}
\end{figure}

\noindent Die Abbildung zeigt einen beliebigen Prozess der dadurch gekennzeichnet ist, dass eine Eingabe unter dem Einfluss von verschiedenen Steuer- und Störgrößen ein bestimmtes Ergebnis erzeugt wird. Außerdem zu erkennen ist eine Unterteilung der Einflussgrößen in Steuer- und Störgrößen. Das geschieht deshalb, weil es Einflussfaktoren geben kann, die während des Prozessverlaufs nicht beeinflusst werden können (wie bspw. die Außentemperatur, etc.).
\\\\
\noindent Was in diesem Modell noch fehlt, um zu einer ganzheitlichen Analyse zu gelangen, ist die Einbeziehung der Wechselwirkungen zwischen den Einflussfaktoren in das Modell. Denn nur durch die Betrachtung der Gesamtdynamik eines Systems ist es möglich, ein globales Optimum zu finden. Im Rahmen einer statistischen Versuchsplanung wir genau dieses Ziel verfolgt, wobei diese Vorgehensweise auf beliebige Systeme angewendet werden kann, bei denen es darum geht, eine oder mehrere Zielgrößen in Abhängigkeit von mehreren beeinflussenden Faktoren zu optimieren. 
\\\\
Nach \cite{schiefer_2018} kann zwischen den in nachstehender Abbildung \ref{fig:Versuchsplaene} dargestellten Versuchsplänen unterschieden werden:

\begin{figure}[H]
	\centering
	\includegraphics[scale=0.8]{pics/versuchsplaene}
	\caption{Arten von Versuchsplänen}
	\label{fig:Versuchsplaene}
\end{figure}

\noindent Welcher Versuchsplan der Richtige ist, hängt von der jeweiligen Aufgabenstellung ab. Da in der vorliegenden Arbeit eine statistische Auswertung anhand eines \textbf{vollständig faktoriellen Versuchsplans} erfolgt, beschränken sich die weiteren Ausführungen in diesem Kapitel auf dessen detaillierte Beschreibung.
\\\\
\noindent Von elementarer Bedeutung ist es zu wissen, wie viele \textbf{Einflussfaktoren (k)} im Rahmen einer Untersuchung vorliegen und im Modell entsprechend zu berücksichtigen sind und welche \textbf{Variationsstufen (n)} innerhalb eines Faktors möglich sind. Die Anzahl der durchzuführenden Versuche (z) kann dann allgemein wie folgt berechnet werden:
\\\\
\begin{equation}
	\mathbf{z = n^k}
\end{equation}
\\\\
\noindent Wird bspw. ein System mit 3 Einflussfaktoren untersucht, die jeweils in 2 Stufen $(+/-)$ variiert werden können, müssen bei einem vollständig faktoriellen Versuchsplan also $3^2 = 8$ Versuche durchgeführt werden. Der sich daraus ergebende Plan ist in folgender Tabelle \ref{tab:Versuchsplan} dargestellt.

\begin{minipage}[t]{0.4\linewidth}\vspace{0pt}
	\begin{table}[H]
		\centering
		\caption{Versuchsplan mit 3 Einflussfaktoren und jeweils 2 Faktorstufenkombinationen}
		\label{tab:Versuchsplan}
		\begin{tabular}{@{}cccc@{}}
			\toprule[3pt]
			\multicolumn{1}{l}{\textbf{\begin{tabular}[c]{@{}l@{}}Anzahl\\ Versuche\end{tabular}}} & \multicolumn{3}{l}{\textbf{\begin{tabular}[c]{@{}l@{}}Faktorstufen-\\ kombinationen\end{tabular}}} \\ \midrule[3pt]
			\multicolumn{1}{l}{\textbf{}}                                                          & \textbf{A}                      & \textbf{B}                      & \textbf{C}                     \\ \midrule
			\textbf{1}                                                                             & -                               & -                               & -                              \\ \midrule
			\textbf{2}                                                                             & -                               & -                               & +                              \\ \midrule
			\textbf{3}                                                                             & -                               & +                               & -                              \\ \midrule
			\textbf{4}                                                                             & -                               & +                               & +                              \\ \midrule
			\textbf{5}                                                                             & +                               & -                               & -                              \\ \midrule
			\textbf{6}                                                                             & +                               & -                               & +                              \\ \midrule
			\textbf{7}                                                                             & +                               & +                               & -                              \\ \midrule
			\textbf{8}                                                                             & +                               & +                               & +                              \\ \bottomrule[3pt]
		\end{tabular}
	\end{table}
\end{minipage}
\begin{minipage}[t]{0.59\linewidth}\vspace{0pt}
	\begin{figure}[H]
		\centering
		\medskip\medskip\medskip\medskip\medskip\medskip\medskip\medskip\smallskip
		\begin{tikzpicture}[x=0.75pt,y=0.75pt,yscale=-1,xscale=1, scale=0.6, every node/.style={scale=0.6}]
			%uncomment if require: \path (0,222); %set diagram left start at 0, and has height of 222
			
			%Straight Lines [id:da017743038240972897] 
			\draw [color={rgb, 255:red, 126; green, 211; blue, 33 }  ,draw opacity=1 ][line width=2.25]    (320,389) -- (320,409) ;
			%Straight Lines [id:da5646155797639212] 
			\draw [color={rgb, 255:red, 126; green, 211; blue, 33 }  ,draw opacity=1 ][line width=2.25]    (482,350) -- (499.8,350) ;
			%Straight Lines [id:da5139851724668463] 
			\draw [color={rgb, 255:red, 126; green, 211; blue, 33 }  ,draw opacity=1 ][line width=2.25]    (432,279) -- (449.8,279) ;
			
			% Text Node
			\draw (314,419) node [anchor=north west][inner sep=0.75pt]   [align=left] {\textbf{\textcolor[rgb]{0.49,0.83,0.13}{A}}};
			% Text Node
			\draw (520,360) node [anchor=north west][inner sep=0.75pt]   [align=left] {\textbf{\textcolor[rgb]{0.49,0.83,0.13}{B}}};
			% Text Node
			\draw (464,270) node [anchor=north west][inner sep=0.75pt]   [align=left] {\textbf{\textcolor[rgb]{0.49,0.83,0.13}{C}}};
			% Text Node
			\draw (174,385) node [anchor=north west][inner sep=0.75pt]   [align=left] {\begin{minipage}[lt]{38.5pt}\setlength\topsep{0pt}
					\begin{center}
						\textbf{\textcolor[rgb]{0.29,0.56,0.89}{1}}\\\textbf{\textcolor[rgb]{0.29,0.56,0.89}{(- - -)}}
					\end{center}
					
			\end{minipage}};
			% Text Node
			\draw (174,145) node [anchor=north west][inner sep=0.75pt]   [align=left] {\begin{minipage}[lt]{38.5pt}\setlength\topsep{0pt}
					\begin{center}
						\textbf{\textcolor[rgb]{0.29,0.56,0.89}{2}}\\\textbf{\textcolor[rgb]{0.29,0.56,0.89}{(- - +)}}
					\end{center}
					
			\end{minipage}};
			% Text Node
			\draw (274,288) node [anchor=north west][inner sep=0.75pt]   [align=left] {\begin{minipage}[lt]{38.5pt}\setlength\topsep{0pt}
					\begin{center}
						\textbf{\textcolor[rgb]{0.29,0.56,0.89}{3}}\\\textbf{\textcolor[rgb]{0.29,0.56,0.89}{(- + -)}}
					\end{center}
					
			\end{minipage}};
			% Text Node
			\draw (360,411) node [anchor=north west][inner sep=0.75pt]   [align=left] {\textbf{\textcolor[rgb]{0.49,0.83,0.13}{+}}};
			% Text Node
			\draw (266,414) node [anchor=north west][inner sep=0.75pt]   [align=left] {\begin{minipage}[lt]{8.67pt}\setlength\topsep{0pt}
					\begin{center}
						\textbf{\textcolor[rgb]{0.49,0.83,0.13}{-}}
					\end{center}
					
			\end{minipage}};
			% Text Node
			\draw (495,370) node [anchor=north west][inner sep=0.75pt]   [align=left] {\begin{minipage}[lt]{8.67pt}\setlength\topsep{0pt}
					\begin{center}
						\textbf{\textcolor[rgb]{0.49,0.83,0.13}{-}}
					\end{center}
					
			\end{minipage}};
			% Text Node
			\draw (512,342) node [anchor=north west][inner sep=0.75pt]   [align=left] {\textbf{\textcolor[rgb]{0.49,0.83,0.13}{+}}};
			% Text Node
			\draw (450,231) node [anchor=north west][inner sep=0.75pt]   [align=left] {\textbf{\textcolor[rgb]{0.49,0.83,0.13}{+}}};
			% Text Node
			\draw (451,323) node [anchor=north west][inner sep=0.75pt]   [align=left] {\begin{minipage}[lt]{8.67pt}\setlength\topsep{0pt}
					\begin{center}
						\textbf{\textcolor[rgb]{0.49,0.83,0.13}{-}}
					\end{center}
					
			\end{minipage}};
			% Text Node
			\draw (274,46) node [anchor=north west][inner sep=0.75pt]   [align=left] {\begin{minipage}[lt]{38.5pt}\setlength\topsep{0pt}
					\begin{center}
						\textbf{\textcolor[rgb]{0.29,0.56,0.89}{4}}\\\textbf{\textcolor[rgb]{0.29,0.56,0.89}{(- + +)}}
					\end{center}
					
			\end{minipage}};
			% Text Node
			\draw (417,385) node [anchor=north west][inner sep=0.75pt]   [align=left] {\begin{minipage}[lt]{33.94pt}\setlength\topsep{0pt}
					\begin{center}
						\textbf{\textcolor[rgb]{0.29,0.56,0.89}{5}}\\\textbf{\textcolor[rgb]{0.29,0.56,0.89}{(+ - -)}}
					\end{center}
					
			\end{minipage}};
			% Text Node
			\draw (415,145) node [anchor=north west][inner sep=0.75pt]   [align=left] {\begin{minipage}[lt]{38.5pt}\setlength\topsep{0pt}
					\begin{center}
						\textbf{\textcolor[rgb]{0.29,0.56,0.89}{6}}\\\textbf{\textcolor[rgb]{0.29,0.56,0.89}{(+ - +})}
					\end{center}
					
			\end{minipage}};
			% Text Node
			\draw (515,288) node [anchor=north west][inner sep=0.75pt]   [align=left] {\begin{minipage}[lt]{38.5pt}\setlength\topsep{0pt}
					\begin{center}
						\textbf{\textcolor[rgb]{0.29,0.56,0.89}{7}}\\\textbf{\textcolor[rgb]{0.29,0.56,0.89}{(+ + -)}}
					\end{center}
					
			\end{minipage}};
			% Text Node
			\draw (509,48) node [anchor=north west][inner sep=0.75pt]   [align=left] {\begin{minipage}[lt]{44.5pt}\setlength\topsep{0pt}
					\begin{center}
						\textbf{\textcolor[rgb]{0.29,0.56,0.89}{8}}\\\textbf{\textcolor[rgb]{0.29,0.56,0.89}{(+ + +)}}
					\end{center}
					
			\end{minipage}};
			% Text Node
			\draw  [color={rgb, 255:red, 74; green, 144; blue, 226 }  ,draw opacity=1 ][line width=1.5]   (382, 222.5) circle [x radius= 76.37, y radius= 14.85]   ;
			\draw (331,217) node [anchor=north west][inner sep=0.75pt]   [align=left] {\textbf{\textcolor[rgb]{0.29,0.56,0.89}{Versuchsraum}}};
			% Connection
			\draw [line width=1.5]    (464.87,375.38) -- (514.93,326.79) ;
			\draw [shift={(517.8,324)}, rotate = 495.86] [fill={rgb, 255:red, 0; green, 0; blue, 0 }  ][line width=0.08]  [draw opacity=0] (11.61,-5.58) -- (0,0) -- (11.61,5.58) -- cycle    ;
			\draw [shift={(462,378.16)}, rotate = 315.86] [fill={rgb, 255:red, 0; green, 0; blue, 0 }  ][line width=0.08]  [draw opacity=0] (11.61,-5.58) -- (0,0) -- (11.61,5.58) -- cycle    ;
			% Connection
			\draw [line width=1.5]    (541.39,274) -- (540.61,88) ;
			\draw [shift={(540.6,84)}, rotate = 449.76] [fill={rgb, 255:red, 0; green, 0; blue, 0 }  ][line width=0.08]  [draw opacity=0] (11.61,-5.58) -- (0,0) -- (11.61,5.58) -- cycle    ;
			\draw [shift={(541.4,278)}, rotate = 269.76] [fill={rgb, 255:red, 0; green, 0; blue, 0 }  ][line width=0.08]  [draw opacity=0] (11.61,-5.58) -- (0,0) -- (11.61,5.58) -- cycle    ;
			% Connection
			\draw [line width=1.5]    (439.72,373) -- (441.28,186) ;
			\draw [shift={(441.31,182)}, rotate = 450.48] [fill={rgb, 255:red, 0; green, 0; blue, 0 }  ][line width=0.08]  [draw opacity=0] (11.61,-5.58) -- (0,0) -- (11.61,5.58) -- cycle    ;
			\draw [shift={(439.69,377)}, rotate = 270.48] [fill={rgb, 255:red, 0; green, 0; blue, 0 }  ][line width=0.08]  [draw opacity=0] (11.61,-5.58) -- (0,0) -- (11.61,5.58) -- cycle    ;
			% Connection
			\draw [line width=1.5]    (225,399.1) -- (413,399.89) ;
			\draw [shift={(417,399.91)}, rotate = 180.24] [fill={rgb, 255:red, 0; green, 0; blue, 0 }  ][line width=0.08]  [draw opacity=0] (11.61,-5.58) -- (0,0) -- (11.61,5.58) -- cycle    ;
			\draw [shift={(221,399.09)}, rotate = 0.24] [fill={rgb, 255:red, 0; green, 0; blue, 0 }  ][line width=0.08]  [draw opacity=0] (11.61,-5.58) -- (0,0) -- (11.61,5.58) -- cycle    ;
			% Connection
			\draw [line width=1.5]    (200.61,372) -- (201.39,186) ;
			\draw [shift={(201.4,182)}, rotate = 450.24] [fill={rgb, 255:red, 0; green, 0; blue, 0 }  ][line width=0.08]  [draw opacity=0] (11.61,-5.58) -- (0,0) -- (11.61,5.58) -- cycle    ;
			\draw [shift={(200.6,376)}, rotate = 270.24] [fill={rgb, 255:red, 0; green, 0; blue, 0 }  ][line width=0.08]  [draw opacity=0] (11.61,-5.58) -- (0,0) -- (11.61,5.58) -- cycle    ;
			% Connection
			\draw [line width=1.5]    (228,159) -- (413,159) ;
			\draw [shift={(417,159)}, rotate = 180] [fill={rgb, 255:red, 0; green, 0; blue, 0 }  ][line width=0.08]  [draw opacity=0] (11.61,-5.58) -- (0,0) -- (11.61,5.58) -- cycle    ;
			\draw [shift={(224,159)}, rotate = 0] [fill={rgb, 255:red, 0; green, 0; blue, 0 }  ][line width=0.08]  [draw opacity=0] (11.61,-5.58) -- (0,0) -- (11.61,5.58) -- cycle    ;
			% Connection
			\draw [line width=1.5]    (226.87,134.38) -- (275.93,86.79) ;
			\draw [shift={(278.8,84)}, rotate = 495.86] [fill={rgb, 255:red, 0; green, 0; blue, 0 }  ][line width=0.08]  [draw opacity=0] (11.61,-5.58) -- (0,0) -- (11.61,5.58) -- cycle    ;
			\draw [shift={(224,137.17)}, rotate = 315.86] [fill={rgb, 255:red, 0; green, 0; blue, 0 }  ][line width=0.08]  [draw opacity=0] (11.61,-5.58) -- (0,0) -- (11.61,5.58) -- cycle    ;
			% Connection
			\draw [line width=1.5]    (467.58,133.19) -- (514.42,86.81) ;
			\draw [shift={(517.27,84)}, rotate = 495.29] [fill={rgb, 255:red, 0; green, 0; blue, 0 }  ][line width=0.08]  [draw opacity=0] (11.61,-5.58) -- (0,0) -- (11.61,5.58) -- cycle    ;
			\draw [shift={(464.73,136)}, rotate = 315.29] [fill={rgb, 255:red, 0; green, 0; blue, 0 }  ][line width=0.08]  [draw opacity=0] (11.61,-5.58) -- (0,0) -- (11.61,5.58) -- cycle    ;
			% Connection
			\draw [line width=1.5]    (331,61) -- (510,61) ;
			\draw [shift={(514,61)}, rotate = 180] [fill={rgb, 255:red, 0; green, 0; blue, 0 }  ][line width=0.08]  [draw opacity=0] (11.61,-5.58) -- (0,0) -- (11.61,5.58) -- cycle    ;
			\draw [shift={(327,61)}, rotate = 0] [fill={rgb, 255:red, 0; green, 0; blue, 0 }  ][line width=0.08]  [draw opacity=0] (11.61,-5.58) -- (0,0) -- (11.61,5.58) -- cycle    ;
			% Connection
			\draw [line width=1.5]  [dash pattern={on 5.63pt off 4.5pt}]  (302.27,88) -- (300.73,273) ;
			\draw [shift={(300.69,277)}, rotate = 270.48] [fill={rgb, 255:red, 0; green, 0; blue, 0 }  ][line width=0.08]  [draw opacity=0] (11.61,-5.58) -- (0,0) -- (11.61,5.58) -- cycle    ;
			\draw [shift={(302.31,84)}, rotate = 90.48] [fill={rgb, 255:red, 0; green, 0; blue, 0 }  ][line width=0.08]  [draw opacity=0] (11.61,-5.58) -- (0,0) -- (11.61,5.58) -- cycle    ;
			% Connection
			\draw [line width=1.5]  [dash pattern={on 5.63pt off 4.5pt}]  (327,300.11) -- (513,300.88) ;
			\draw [shift={(517,300.9)}, rotate = 180.24] [fill={rgb, 255:red, 0; green, 0; blue, 0 }  ][line width=0.08]  [draw opacity=0] (11.61,-5.58) -- (0,0) -- (11.61,5.58) -- cycle    ;
			\draw [shift={(323,300.09)}, rotate = 0.24] [fill={rgb, 255:red, 0; green, 0; blue, 0 }  ][line width=0.08]  [draw opacity=0] (11.61,-5.58) -- (0,0) -- (11.61,5.58) -- cycle    ;
			% Connection
			\draw [line width=1.5]  [dash pattern={on 5.63pt off 4.5pt}]  (223.84,375.89) -- (275.16,325.09) ;
			\draw [shift={(278,322.27)}, rotate = 495.29] [fill={rgb, 255:red, 0; green, 0; blue, 0 }  ][line width=0.08]  [draw opacity=0] (11.61,-5.58) -- (0,0) -- (11.61,5.58) -- cycle    ;
			\draw [shift={(221,378.71)}, rotate = 315.29] [fill={rgb, 255:red, 0; green, 0; blue, 0 }  ][line width=0.08]  [draw opacity=0] (11.61,-5.58) -- (0,0) -- (11.61,5.58) -- cycle    ;
			
		\end{tikzpicture}
		\caption{Darstellung des Versuchsraumes aus 3 Einflussfaktoren und Faktorstufenkombinationen $(+/-)$}
		\label{fig:Versuchsplan}
	\end{figure}
\end{minipage}
\\\\\\
\noindent Die sich ergebenden Faktorstufenkombinationen aus der Tabelle beschreiben, bei 3 Einflussfaktoren einen 3-Dimensionalen Raum, wie in Abbildung \ref{fig:Versuchsplan} dargestellt. Dieser wird jeweils durch die minimalen und maximalen Ausprägungen begrenzt und beschreibt den sog. Versuchsraum, der alle möglichen Wirkungen enthält. 
\\\\
\noindent Der Vorteil von vollständig faktoriellen Versuchsplänen ist der, dass wirklich alle möglichen Faktorausprägungen und Wirkungen in der Analyse berücksichtigt werden und somit der Erkenntnisgewinn am größten ist. Allerdings können diese Pläne sehr schnell auch extrem groß werden, wodurch die Durchführung in der Praxis oftmals aus Kosten- und/oder Zeitgründen unmöglich ist. In solchen  Fällen kann dann auf die Verwendung von teilfaktoriellen Plänen ausgewichen werden.
\\\\
\noindent In Kapitel \ref{ch:Konzept} ab Seite \pageref{ch:Konzept} wird das hier beschriebene Modell eines vollständig faktoriellen Versuchsplanes wieder aufgegriffen und dahingehend erweitert, das eine Anwendung zur Untersuchung der im Rahmen dieser Arbeit zu untersuchenden Fragestellung möglich wird. In Kapitel \ref{ch:Evaluation} ab Seite \pageref{ch:Evaluation} werden die Untersuchungsergebnisse dann anhand dieses Modells schließlich ausgewertet und interpretiert.

\section{Überblick über die Skriptsprache AutoIt}
\label{sec:UeberblickAutoIt}

Im Rahmen dieser Arbeit wird AutoIt einerseits dazu genutzt, um die zur Evaluation der Wirkungsweise von unterschiedlichen Interpolationsalgorithmen benötigten Daten zu generieren. Dazu ist es nötig, tausende von Lamellenguss-Messungen mit der Software AMGuss durchzuführen. Dies wäre ohne Automatisierung mit einem nicht vertretbaren Zeitaufwand verbunden. Die automatische Durchführung der Messungen mit AutoIt funktioniert dagegen nach einigen Tests sehr gut und der Zeitaufwand lässt sich somit auf etwa 20 Sekunden pro Messung reduzieren.
\\\\
\noindent Gemäß offizieller Dokumentation \cite{AutoItDocs} lässt sich AutoIt wie folgt beschreiben:
\\\\
\noindent Es handelt sich dabei um eine BASIC-ähnliche Skriptsprache, die zur Automatisierung von Windows-Benutzeroberflächen entwickelt wurde. Es verwendet eine Kombination aus simulierten Tastenanschlägen, Mausbewegungen sowie Fenster- und Steuerungsmanipulationen, um Aufgaben auf eine Weise zu automatisieren, die mit anderen Sprachen (z.B. VBScript und SendKeys) nicht möglich ist.
\\\\
\noindent AutoIt wurde ursprünglich für PC-Rollout-Situationen entwickelt, um Tausende von PCs zuverlässig zu automatisieren und zu konfigurieren. Im Laufe der Zeit hat es sich zu einer leistungsstarken Sprache entwickelt, die komplexe Ausdrücke, Benutzerfunktionen, Schleifen und alles andere unterstützt, was erfahrene Skripter erwarten würden. 
\\\\
\textbf{Die wichtigsten Eigenschaften kurz zusammengefasst:}
\begin{itemize}
	\item Einfach zu erlernende BASIC-ähnliche Syntax
	\item Simulation von Tastenanschlägen und Mausbewegungen
	\item Steuerung von Fenstern und Prozessen
	\item Interaktion mit Standard-Windows-Steuerelementen
	\item Kompilierung von Skripten zu eigenen ausführbaren Dateien
	\item Erstellung grafischer Benutzeroberflächen
	\item COM-Unterstützung
	\item Reguläre Ausdrücke
	\item Direkter Aufruf von Funktionen aus externen DLL's sowie von Windows-API-Funktionen
	\item etc.
\end{itemize}

\chapter{Ausgangssituation}
\label{ch:Ausgangssituation}
Ziel dieses Kapitels ist es, die Ausgangssituation, also die mit dieser Arbeit verbundenen Rahmenbedingungen im Detail darzulegen, um damit die Grundlage für eine schlüssige Präsentation des Lösungskonzeptes im darauf folgenden Kapitel \ref{ch:Konzept} zu schaffen.
\\\\
Im Abschnitt \ref{sec:BestimmungMikrostrukturAMGuss} wird zunächst auf die Bedeutung des Einsatzes digitaler Bildverarbeitung zur Analyse von Gusseisen eingegangen und in diesem Zusammenhang die Software AMGuss vorgestellt. Dabei wird auf die Bedeutung eines Anordnungsklassifikators eingegangen und die Erstellung eines solchen im Detail dargestellt (Abschnitt \ref{subsec: ErstellungAnordnungsklassifAMGuss}) und das Kapitel schließt mit einer ausführlichen Problembeschreibung in Abschnitt \ref{sec:Problemstellung}.

\section{Einsatz digitaler Bildverarbeitung zur Analyse der Mikrostruktur von Gusseisen (AMGuss)}
\label{sec:BestimmungMikrostrukturAMGuss}
Wegen der mit rein visueller Qualitätskontrolle verbundenen Probleme, die bereits in Kapitel \ref{subsec:MethodenBestAnordnungsklassen} auf Seite \pageref{subsec:MethodenBestAnordnungsklassen} dargelegt wurden, war ein automatisch ablaufendes Analyseverfahren zur Bestimmung der Mikrostruktur von Gusseisen, erweitert um Bilddokumentation und Bildarchivierung, ein großes Anliegen in den Laboren der Eisengießereien. Dadurch sollte eines der Grundanliegen des Qualitätsmanagements erfüllt werden, nämlich die quantitative Charakterisierung des Gussvorgangs durch Bereitstellung aussagekräftiger Messparameter und Objektivierung der Messergebnisse. Aus diesem Anliegen heraus entstand bei der GFaI e.V. bereits im Jahr 2005 ein Forschungsprojekt mit der Zielsetzung, die Grundlage zur Entwicklung eines marktreifen Software-Produktes zu schaffen, mit dem sich alle geforderten Normparameter der Mikrostruktur von Gusseisen automatisch (oder zumindest halbautomatisch) bestimmen lassen. \cite{AMGuss2007}.
\\\\
Das Ergebnis dieses Forschungsprojektes ist die Software AMGuss (\textbf{A}nalyse der \textbf{M}ikrostruktur von \textbf{Guss}eisen) die eine normgerechte (halb-) automatische Bestimmung der Mikrostruktur von Gusseisen ermöglicht. Da die vorliegende Arbeit nur der Weiterentwicklung einer Teilfunktionalität dieser Software gewidmet ist, nämlich der Lamellengraphit-Auswertung, beschränken sich die weiteren Ausführungen in diesem Abschnitt auf lediglich deren genauere Beschreibung. 

\subsection{Erstellung eines Anordnungsklassifikators für die Lamellengraphit-\\Auswertung}
\label{subsec: ErstellungAnordnungsklassifAMGuss}

\noindent Aufgrund unterschiedlicher Laborbedingungen können in der Praxis Probenbilder mit unterschiedlichen Kalibrierungen (in $\mu m/\text{Pixel}$) zur Messung vorliegen. Allerdings ist das für die eindeutige, exakte Analyse und Zuordnung (siehe Abbildung \ref{fig:Anordnungsklassen} auf Seite \pageref{fig:Anordnungsklassen}) problematisch, da die Unterscheidungsmerkmale der Anordnungsklassen verschwimmen und somit nicht mehr eindeutig identifiziert und zugeordnet werden können. Deshalb ist für die Durchführung einer Lamellenguss-Auswertung mit der Software AMGuss für jede Ausgangskalibrierung vorliegender Bilder ein eigener Anordnungsklassifikator manuell zu erstellen. Dabei werden in etwa 20-30 Probenbildern die darin vorliegenden lamellaren Graphitstrukturen gemäß der Anordnungsklassen markiert und auf der Grundlage dessen ein Klassifikator trainiert. Wichtig dabei ist zu beachten, dass alle hierfür verwendeten Bilder dieselbe Kalibrierung besitzen, in den Trainingsbildern auch alle Anordnungsklassen vertreten sind und diese richtig markiert werden.
\\\\
Dafür stellt das Programm AMGuss dem Nutzer eine entsprechende Funktionalität bereit, wie in Abbildung \ref{fig:ErstAnordnungsklass} dargestellt.


\begin{figure}[H]
	\centering
	\boxed{\includegraphics[scale=1.6]{pics/ErstAnordnungsklass}}
	\caption{Erstellung eines Anordnungsklassifikators in AMGuss}
	\label{fig:ErstAnordnungsklass}
\end{figure}

\noindent Grundlage für die Erstellung eines Klassifikators ist eine bereits geladene Bildserie, die zuvor erstellt worden sein muss. Dabei ist anzugeben, mit welcher Kalibrierung  die Bilder der Serie aufgenommen wurden. Diese ist abhängig von der Chipgröße der verwendeten Kamera (z.B. $\nicefrac{1}{2}$ Zoll in der Diagonale)  sowie der eingesetzten Optik bestehend aus Lichtweg, Objektiv und Zwischenadapter. Das in Abbildung \ref{fig:ErstAnordnungsklass} (links) dargestellte Bild einer Lamellenguss-Probe hat bspw. eine Kalibrierung von $1~\mu m/\text{Pixel}$. 
\\\\
Im kleinen Fenster (oben rechts) lässt sich dann eine Anordnungsklasse auswählen, der gewünschte Markierungsstil einstellen und die gemäß der gewählten Anordnungsklasse vorkommenden Strukturen im Bild entsprechend markieren. Im Bild links ist dies an den hellgrün markierten Strukturen zu erkennen. Hat der Nutzer dann in mehreren Bildern Markierungen für die Anordnungsklassen A bis E hinzugefügt, wird über die Funktion „Initialisieren“ (im mittleren Fenster oben) ein Anordnungsklassifikator erstellt. Dieser kann dann abgespeichert und für Vermessungen von Bildern verwendet werden, deren Kalibrierung dieselbe ist wie die der Bilder, die zur Erstellung des Anordnungsklassifikators verwendet wurden. 

\section{Problemstellung}
\label{sec:Problemstellung}

Die Problematik besteht darin, dass für in unterschiedlichen Vergrößerungen vorliegende Bilder kein universell anwendbarer Klassifikator existiert. Hätte man jedoch einen im System hinterlegten Standardklassifikator, der für Bilder mit einer definierten Kalibrierung anwendbar ist, so könnten durch eine entsprechende Umskalierung in der Kalibrierung abweichende Probenbilder trotzdem vermessen werden. Da sich die Kalibrierung umgekehrt Proportional zur Bildgröße verhält, müsste zum Beispiel ein mit einer Ausgangskalibrierung von $1,0~\nicefrac{\mu m}{\text{Pixel}}$ vorliegendes Bild also um den Faktor 2 hoch-skaliert (vergrößert) werden, um die Kalibrierung auf $0,5~\nicefrac{\mu m}{\text{Pixel}}$ zu verändern. Analog dazu wäre es dann natürlich auch möglich, ein mit hoher Auflösung aufgenommenes Bild  durch Verkleinerung entsprechend anzupassen. Geht man weiter davon aus, das der im System hinterlegte Standard-Klassifikator mit Bildern derselben Kalibrierung erstellt wurde, wäre nun eine korrekte Messung mit diesem Klassifikator möglich. Somit würde der Arbeitsschritt zur manuellen Erstellung und Verwaltung von Klassifikatoren für unterschiedliche Ausgangskalibrierungen wegfallen, die Forderung nach Allgemeingültigkeit wäre damit erfüllt und man würde mit nur einem einzigen Anordnungsklassifikator auskommen.
\\\\
Bei der Skalierung von Bildern ist jedoch zu beachten, dass zur Erhaltung der Bildqualität verschiedene Interpolationsalgorithmen zum Einsatz kommen, um die sich durch Vergrößerung/Verkleinerung ergebende Differenz der Pixelanzahl zwischen Eingangs- und Ausgangsbild auszugleichen. Im Falle der Vergrößerung ergibt sich, dass das Ausgangsbild entsprechend dem Vergrößerungsfaktor mehr Pixel besitzt als das Eingangsbild und die sich dadurch ergebenden Pixellücken durch Interpolation gefüllt werden müssen. Bei Verkleinerung verhält es sich genau umgekehrt, nur dass in diesem Fall die Pixelmenge verkleinert wird. Es liegt auf der Hand, dass eine solche Interpolation die Details des Originalbildes nie 100\%-ig erhalten kann, wie in Abschnitt \ref{subsec:SkalierungUndInterpolation} bereits beschrieben wurde. So können zwar für viele praktischen Anwendungsfälle im Grafik- oder Fotobereich sehr gute Ergebnisse erzielt werden, weil das menschliche Auge diese Abweichungen nicht erkennen kann. Allerdings erweist sich die Interpolation im Bereich der bildbasierten Materialstrukturanalyse als problematisch, da sich  die zu analysierenden Strukturen durch die Interpolation verändern und somit die Messergebnisse verfälscht werden. Welche unerwünschten Auswirkungen das im vorliegenden Fall hat, zeigen die folgenden beiden Abbildungen 3.2 und 3.3.

\begin{figure}[h]
	\label{fig:KlassErrorVisuell}
	\begin{subfigure}{0.33\textwidth}
		\centering
		\includegraphics[scale=0.5]{pics/ClassificationCalib10.png}
		\caption{Ausgangsbild mit Kalibrierung 1,0}
	\end{subfigure}\hfill
	\begin{subfigure}{0.33\textwidth}
		\centering
		\begin{tikzpicture}[x=0.75pt,y=0.75pt,yscale=-1,xscale=1]
			%uncomment if require: \path (0,300); %set diagram left start at 0, and has height of 300
			
			%Right Arrow [id:dp9517904043992672] 
			\draw  [color={rgb, 255:red, 208; green, 2; blue, 27 }  ,draw opacity=1 ][fill={rgb, 255:red, 255; green, 255; blue, 255 }  ,fill opacity=1 ][line width=1.5]  (192,87.35) -- (273.83,87.35) -- (273.83,65.85) -- (328.38,108.85) -- (273.83,151.85) -- (273.83,130.35) -- (192,130.35) -- cycle ;
			
			% Text Node
			\draw (198,97) node [anchor=north west][inner sep=0.75pt]  [font=\tiny] [align=left] {Änderung der Kalibrierung\\von 1,0 auf 0,5 = Skalierung\\um den Faktor 2};
		\end{tikzpicture}
	\end{subfigure}\hfill
	\begin{subfigure}{0.33\textwidth}
		\centering
		\includegraphics[scale=0.5]{pics/ClassificationCalib05.png}
		\caption{Transform. Bild mit Kalibrierung 0,5}
	\end{subfigure}
	\caption{Messfehler-Darstellung bzgl. der Graphitanordnung bei bikubischer Interpolation}
	\label{fig:ErrorAKL}
\end{figure}

\begin{figure}[h]
	\label{fig:KlassErrorHisto}
	\begin{subfigure}{0.33\textwidth}
		\centering
		\includegraphics[scale=0.5]{pics/HistoCalib10}
		\caption{Ausgangsbild mit Kalibrierung 1,0}
	\end{subfigure}\hfill
	\begin{subfigure}{0.33\textwidth}
		\centering
		\begin{tikzpicture}[x=0.75pt,y=0.75pt,yscale=-1,xscale=1]
			%uncomment if require: \path (0,300); %set diagram left start at 0, and has height of 300
			
			%Right Arrow [id:dp9517904043992672] 
			\draw  [color={rgb, 255:red, 208; green, 2; blue, 27 }  ,draw opacity=1 ][fill={rgb, 255:red, 255; green, 255; blue, 255 }  ,fill opacity=1 ][line width=1.5]  (192,87.35) -- (273.83,87.35) -- (273.83,65.85) -- (328.38,108.85) -- (273.83,151.85) -- (273.83,130.35) -- (192,130.35) -- cycle ;
			
			% Text Node
			\draw (198,97) node [anchor=north west][inner sep=0.75pt]  [font=\tiny] [align=left] {Änderung der Kalibrierung\\von 1,0 auf 0,5 = Skalierung\\um den Faktor 2};
		\end{tikzpicture}
	\end{subfigure}\hfill
	\begin{subfigure}{0.33\textwidth}
		\centering
		\includegraphics[scale=0.5]{pics/HistoCalib05.png}
		\caption{Transform. Bild mit Kalibrierung 0,5}
	\end{subfigure}
	\caption{Messfehler-Darstellung bzgl. der Größenklassen (Lamellenlängen) bei bikubischer Interpolation}
	\label{fig:ErrorGKL}
\end{figure}

\newpage

\noindent Es wurde also ein Bild mit einer Ausgangskalibrierung von $1,0~\nicefrac{\mu m}{\text{Pixel}}$ um den Faktor 2 mit dem Ziel skaliert, die Kalibrierung auf $0,5~\nicefrac{\mu m}{\text{Pixel}}$ zu verändern. Die beiden Abbildungen zeigen nun unterschiedliche Sichtweisen auf ein und dasselbe Problem. Die Fehler sind zwar im hier betrachteten Fall sehr gering, im Bereich der Materialstrukturanalyse jedoch sehr unerwünscht, da sich aus der Lage, Größe und Verteilung (also der Gesamtstruktur) mechanische Materialeigenschaften (wie bspw. Druck und Zugfestigkeit, Bruchsicherheit, etc.) ableiten lassen und somit diese Abweichungen zu Fehleinschätzungen bei der qualitativen Bewertung des Materials führen können.

\chapter{Konzept}
\label{ch:Konzept}

\section{Sollzustand/Anforderungen}
\label{sec:DefAnforderungenAnordnKlas}
Die Vorgehensweise zur Erstellung eines Anordnungsklassifikators wurde in Kapitel \ref{subsec: ErstellungAnordnungsklassifAMGuss} bereits beschrieben. Dies ist für den Nutzer mit einem nicht unerheblichen Aufwand verbunden. Hinzu kommt eine gewisse Fehleranfälligkeit, da für jede Messung der in Bezug auf die Bildkalibrierung richtige Klassifikator für die Messung ausgewählt werden muss.
\\\\
\noindent Das Gütekriterium an einen allgemeingültigen Klassifikator ist, die durch Skalierung (bzw. Interpolation) hervorgerufenen und in Kapitel \ref{sec:Problemstellung} bereits näher beschriebenen Messfehler auf ein \textbf{tolerierbares} Maß hin zu minimieren. Allerdings gibt es jedoch, nach den aktuellen allgemein anerkannten Regeln der Technik, keinen eindeutigen objektiven Maßstab, der zur Beurteilung angelegt werden könnte. Stattdessen beruht die Graphitklassifizierung auf einer visuellen Einschätzung der Spezialisten, welche die Beurteilung der Proben vornehmen. Die Norm DIN ISO 945-1 definiert dabei die Grundlagen, auf denen eine solche Beurteilung zu erfolgen hat. Was also als noch tolerierbar gilt, entscheidet der versierte Nutzer in gewissen Grenzen selbst und wie die Erfahrungen zeigen, existieren teils nicht unerhebliche Abweichungen bei der Einschätzung. Die im Rahmen dieser Arbeit zu erfüllenden Anforderungen lassen sich daher wie folgt beschreiben:
\\\\
\noindent Die Aufgabe besteht darin zu untersuchen und nachzuweisen, wie sich die Anwendung verschiedener Interpolationsalgorithmen bei der Bildtransformation bzw. Veränderung der Bildkalibrierung im Verhältnis zu einem Standard-Klassifikator mit einer Kalibrierungsfaktor von 0,5 $\nicefrac{\mu m}{\text{Pixel}}$ auf die Messgenauigkeit bei der Durchführung von Lamellengraphitauswertungen mit der Software AMGuss auswirken. 

\section{Erzeugung von Testbildern}
\label{sec:ErzeugungVonBildern}

Zur Durchführung der im Rahmen dieser Arbeit durchzuführenden Versuche, steht eine Menge von insgesamt 232 repräsentativer Probenbilder mit einer Kalibrierung von 0,5 $\nicefrac{\mu m}{\text{Pixel}}$ zur Verfügung, für die ein Standard-Klassifikator trainiert wurde. Abbildung \ref{fig:Planungsmodell(1)} zeigt den bereits in Abschnitt \ref{sec:StatVersPlanung} beschriebenen Zusammenhang bezogen darauf, dass zur Vorbereitung der Versuchsdurchführungen zunächst Testbilder in unterschiedlichen Kalibrierungen erzeugt werden müssen, damit diese dann mit AMGuss gemessen, mit den Originalbildern verglichen und die Ergebnisse dann schließlich ausgewertet werden können.

\begin{figure}[H]
	\centering
	\begin{tikzpicture}[x=0.75pt,y=0.75pt,yscale=-1,xscale=1, scale=0.7, every node/.style={scale=0.7}]
		%uncomment if require: \path (0,222); %set diagram left start at 0, and has height of 222
		
		%Straight Lines [id:da8513951726256912] 
		\draw [line width=0.75]  [dash pattern={on 0.84pt off 2.51pt}]  (252,80) -- (252,320) ;
		%Straight Lines [id:da5456588557389066] 
		\draw [line width=1.5]    (172,420) -- (412,420) ;
		%Straight Lines [id:da6601013014749395] 
		\draw [line width=1.5]    (172,180) -- (172,420) ;
		%Straight Lines [id:da3287587582519509] 
		\draw [line width=1.5]    (172,180) -- (412,180) ;
		%Straight Lines [id:da8891201835694393] 
		\draw [line width=1.5]    (252,80) -- (172,180) ;
		%Straight Lines [id:da8301479230800961] 
		\draw [line width=1.5]    (492,80) -- (412,180) ;
		%Straight Lines [id:da897285446062482] 
		\draw [line width=1.5]    (492,320) -- (412,420) ;
		%Straight Lines [id:da291618853913844] 
		\draw [line width=1.5]    (492,80) -- (492,320) ;
		%Straight Lines [id:da27301051400697607] 
		\draw [line width=1.5]    (252,80) -- (492,80) ;
		%Straight Lines [id:da03595133060933242] 
		\draw [line width=0.75]  [dash pattern={on 0.84pt off 2.51pt}]  (252,320) -- (492,320) ;
		%Straight Lines [id:da5805598360719648] 
		\draw [line width=0.75]  [dash pattern={on 0.84pt off 2.51pt}]  (252,320) -- (172,420) ;
		%Straight Lines [id:da6408604526403883] 
		\draw    (291,414) -- (291,427) ; 				% 1.10
		%Straight Lines [id:da5720641419192563] 
		\draw [color={rgb, 255:red, 208; green, 2; blue, 27 }  ,draw opacity=1 ]   (201,414) -- (201,427) ;			% 0.50
		%Straight Lines [id:da7586148570180449] 
		\draw    (231,414) -- (231,427) ; 				% 0.70
		%Straight Lines [id:da6213998766294877] 
		\draw    (412,414) -- (412,427) ; 				% 1.90
		%Straight Lines [id:da21868777946411622] 
		\draw    (172,414) -- (172,427) ;				% 0.30
		%Shape: Parallelogram [id:dp6855290487769707] 
		\draw  [fill={rgb, 255:red, 126; green, 211; blue, 33 }  ,fill opacity=0.83 ] (252,200) -- (492,200) -- (412,300) -- (172,300) -- cycle ;
		%Shape: Square [id:dp5568502450606261] 
		\draw  [fill={rgb, 255:red, 74; green, 74; blue, 74 }  ,fill opacity=1 ] (304,224) -- (374,224) -- (374,294) -- (304,294) -- cycle ;
		%Shape: Square [id:dp11497079395910248] 
		\draw  [fill={rgb, 255:red, 74; green, 74; blue, 74 }  ,fill opacity=1 ] (294,234) -- (364,234) -- (364,304) -- (294,304) -- cycle ;
		%Shape: Square [id:dp9507633897127776] 
		\draw  [fill={rgb, 255:red, 74; green, 74; blue, 74 }  ,fill opacity=1 ] (285,244) -- (355,244) -- (355,314) -- (285,314) -- cycle ;
		%Shape: Square [id:dp2885472363956254] 
		\draw  [fill={rgb, 255:red, 74; green, 74; blue, 74 }  ,fill opacity=1 ] (273,256) -- (343,256) -- (343,326) -- (273,326) -- cycle ;
		%Shape: Square [id:dp9167567864275155] 
		\draw  [fill={rgb, 255:red, 74; green, 74; blue, 74 }  ,fill opacity=1 ] (262,268) -- (332,268) -- (332,338) -- (262,338) -- cycle ;
		
		%Straight Lines [id:da21914875225668995] 
		\draw [line width=1.5]    (412,180) -- (412,420) ;
		
		% Text Node
		\draw (276,284) node [anchor=north west][inner sep=0.75pt]  [color={rgb, 255:red, 255; green, 255; blue, 255 }  ,opacity=1 ] [align=left] {\begin{minipage}[lt]{28.8pt}\setlength\topsep{0pt}
				\begin{center}
					232\\Bilder
				\end{center}
				
		\end{minipage}};
		% Text Node
		\draw (506,199.71) node [anchor=north] [inner sep=0.75pt]  [font=\Large,rotate=-270] [align=left] {\textbf{Interpolationsmodi}};
		% Text Node
		\draw (159,430) node [anchor=north west][inner sep=0.75pt]  [font=\small] [align=left] {0,30};
		% Text Node
		\draw (190,430) node [anchor=north west][inner sep=0.75pt]  [font=\small,color={rgb, 255:red, 208; green, 2; blue, 27 }  ,opacity=1 ] [align=left] {0,50};
		% Text Node
		\draw (220,430) node [anchor=north west][inner sep=0.75pt]  [font=\small] [align=left] {0,70};
		% Text Node
		\draw (280,430) node [anchor=north west][inner sep=0.75pt]  [font=\small] [align=left] {1,10};
		% Text Node
		\draw (400,430) node [anchor=north west][inner sep=0.75pt]  [font=\small] [align=left] {1,90};
		% Text Node
		\draw (197,408) node [anchor=north west][inner sep=0.75pt]  [font=\small, rotate=52.5,color={rgb, 255:red, 208; green, 2; blue, 27 }  ,opacity=1 ] [align=left] {Zielkalibrierung};
		% Text Node
		\draw (438,410) node [anchor=north west][inner sep=0.75pt]   [align=left] {{\small - Nearest Neighbor}};
		% Text Node
		\draw (461,380) node [anchor=north west][inner sep=0.75pt]  [font=\small] [align=left] {\mbox{-} Linear};
		% Text Node
		\draw (483,349) node [anchor=north west][inner sep=0.75pt]  [font=\small] [align=left] {\mbox{-} Cubic};
		% Text Node
		\draw (492,334) node [anchor=north west][inner sep=0.75pt]  [font=\small] [align=left] {\mbox{-} Area};
		% Text Node
		\draw (501,322) node [anchor=north west][inner sep=0.75pt]  [font=\small] [align=left] {\mbox{-} Lanczos};
		% Text Node
		\draw (450,394) node [anchor=north west][inner sep=0.75pt]   [align=left] {{\small - Nearest Neighbor} {\scriptsize / Exact}};
		% Text Node
		\draw (472,364) node [anchor=north west][inner sep=0.75pt]  [font=\small] [align=left] {\mbox{-} Linear / Exact};
		% Text Node
		\draw (178,450) node [anchor=north west][inner sep=0.75pt]  [font=\Large] [align=left] {\textbf{Ausgangskalibrierungen}};
		% Text Node
		\draw (125,200) node [anchor=north west][inner sep=0.75pt]  [font=\Large,rotate=-308.72] [align=left] {\textbf{Interpolationsverfahren}};
		% Text Node
		\draw (422.19,183.01) node [anchor=north west][inner sep=0.75pt]  [font=\small,color={rgb, 255:red, 65; green, 117; blue, 5 }  ,opacity=1 ,rotate=-308.57] [align=left] {\begin{minipage}[lt]{45.99pt}\setlength\topsep{0pt}
				\begin{center}
					\textbf{schrittweise }\\\textbf{Skalierung}
				\end{center}
				
		\end{minipage}};
		% Text Node
		\draw (423,310) node [anchor=north west][inner sep=0.75pt]  [font=\small,color={rgb, 255:red, 65; green, 117; blue, 5 }  ,opacity=1 ,rotate=-308.09] [align=left] {\textbf{Vollskalierung}};
	\end{tikzpicture}
	\caption{Planungsmodell (1) zur Erzeugung von Versuchsbildern}
	\label{fig:Planungsmodell(1)}
\end{figure}

\noindent Alle vorliegenden Probenbilder werden also unter Anwendung unterschiedlicher Interpolationsverfahren in unterschiedliche Kalibrierungsstufen sowohl voll als auch schrittweise transformiert (Abbildung \ref{fig:Planungsmodell(2)}), sodass für jedes Originalbild am Ende dieses Vorgangs insgesamt 56 ($7\times 2\times 4$) transformierte Kopien existieren, wodurch eine mehrdimensionale Analyse eines jeden Originalbildes ermöglicht wird.

\begin{figure}[H]
	\centering
	\begin{tikzpicture}[x=0.75pt,y=0.75pt,yscale=-1,xscale=1, scale=1.0, every node/.style={scale=1.0}]
		%uncomment if require: \path (0,222); %set diagram left start at 0, and has height of 222
		
		%Shape: Rectangle [id:dp6366412495734111] 
		\draw  [fill={rgb, 255:red, 74; green, 74; blue, 74 }  ,fill opacity=1 ] (53.75,200) -- (110,200) -- (110,255.26) -- (53.75,255.26) -- cycle ;
		%Shape: Rectangle [id:dp751718043193554] 
		\draw  [fill={rgb, 255:red, 74; green, 74; blue, 74 }  ,fill opacity=1 ] (45.71,207.89) -- (101.96,207.89) -- (101.96,263.16) -- (45.71,263.16) -- cycle ;
		%Shape: Rectangle [id:dp8955994284702644] 
		\draw  [fill={rgb, 255:red, 74; green, 74; blue, 74 }  ,fill opacity=1 ] (38.48,215.79) -- (94.73,215.79) -- (94.73,271.05) -- (38.48,271.05) -- cycle ;
		%Shape: Rectangle [id:dp5207514318052735] 
		\draw  [fill={rgb, 255:red, 74; green, 74; blue, 74 }  ,fill opacity=1 ] (28.84,225.26) -- (85.09,225.26) -- (85.09,280.53) -- (28.84,280.53) -- cycle ;
		%Shape: Rectangle [id:dp8155725825356159] 
		\draw  [fill={rgb, 255:red, 74; green, 74; blue, 74 }  ,fill opacity=1 ] (20,234.74) -- (76.25,234.74) -- (76.25,290) -- (20,290) -- cycle ;
		%Rounded Rect [id:dp8565749595273766] 
		\draw  [color={rgb, 255:red, 155; green, 155; blue, 155 }  ,draw opacity=1 ][fill={rgb, 255:red, 155; green, 155; blue, 155 }  ,fill opacity=0.12 ] (165,48) .. controls (165,43.58) and (168.58,40) .. (173,40) -- (252,40) .. controls (256.42,40) and (260,43.58) .. (260,48) -- (260,72) .. controls (260,76.42) and (256.42,80) .. (252,80) -- (173,80) .. controls (168.58,80) and (165,76.42) .. (165,72) -- cycle ;
		%Rounded Rect [id:dp6075836530317975] 
		\draw  [color={rgb, 255:red, 155; green, 155; blue, 155 }  ,draw opacity=1 ][fill={rgb, 255:red, 155; green, 155; blue, 155 }  ,fill opacity=0.12 ] (163,109) .. controls (163,104.58) and (166.58,101) .. (171,101) -- (252,101) .. controls (256.42,101) and (260,104.58) .. (260,109) -- (260,133) .. controls (260,137.42) and (256.42,141) .. (252,141) -- (171,141) .. controls (166.58,141) and (163,137.42) .. (163,133) -- cycle ;
		%Rounded Rect [id:dp0325477461297643] 
		\draw  [color={rgb, 255:red, 155; green, 155; blue, 155 }  ,draw opacity=1 ][fill={rgb, 255:red, 155; green, 155; blue, 155 }  ,fill opacity=0.12 ] (163,169) .. controls (163,164.58) and (166.58,161) .. (171,161) -- (252,161) .. controls (256.42,161) and (260,164.58) .. (260,169) -- (260,193) .. controls (260,197.42) and (256.42,201) .. (252,201) -- (171,201) .. controls (166.58,201) and (163,197.42) .. (163,193) -- cycle ;
		%Rounded Rect [id:dp3162904694467472] 
		\draw  [color={rgb, 255:red, 155; green, 155; blue, 155 }  ,draw opacity=1 ][fill={rgb, 255:red, 155; green, 155; blue, 155 }  ,fill opacity=0.12 ] (163,228) .. controls (163,223.58) and (166.58,220) .. (171,220) -- (252,220) .. controls (256.42,220) and (260,223.58) .. (260,228) -- (260,252) .. controls (260,256.42) and (256.42,260) .. (252,260) -- (171,260) .. controls (166.58,260) and (163,256.42) .. (163,252) -- cycle ;
		%Rounded Rect [id:dp3035992482144214] 
		\draw  [color={rgb, 255:red, 155; green, 155; blue, 155 }  ,draw opacity=1 ][fill={rgb, 255:red, 155; green, 155; blue, 155 }  ,fill opacity=0.12 ] (163,289) .. controls (163,284.58) and (166.58,281) .. (171,281) -- (252,281) .. controls (256.42,281) and (260,284.58) .. (260,289) -- (260,313) .. controls (260,317.42) and (256.42,321) .. (252,321) -- (171,321) .. controls (166.58,321) and (163,317.42) .. (163,313) -- cycle ;
		%Rounded Rect [id:dp2836577100042894] 
		\draw  [color={rgb, 255:red, 155; green, 155; blue, 155 }  ,draw opacity=1 ][fill={rgb, 255:red, 155; green, 155; blue, 155 }  ,fill opacity=0.12 ] (163,349) .. controls (163,344.58) and (166.58,341) .. (171,341) -- (252,341) .. controls (256.42,341) and (260,344.58) .. (260,349) -- (260,373) .. controls (260,377.42) and (256.42,381) .. (252,381) -- (171,381) .. controls (166.58,381) and (163,377.42) .. (163,373) -- cycle ;
		%Rounded Rect [id:dp9146993081319426] 
		\draw  [color={rgb, 255:red, 155; green, 155; blue, 155 }  ,draw opacity=1 ][fill={rgb, 255:red, 155; green, 155; blue, 155 }  ,fill opacity=0.12 ] (163,408) .. controls (163,403.58) and (166.58,400) .. (171,400) -- (252,400) .. controls (256.42,400) and (260,403.58) .. (260,408) -- (260,432) .. controls (260,436.42) and (256.42,440) .. (252,440) -- (171,440) .. controls (166.58,440) and (163,436.42) .. (163,432) -- cycle ;
		%Rounded Rect [id:dp5049349195701525] 
		\draw  [color={rgb, 255:red, 155; green, 155; blue, 155 }  ,draw opacity=1 ][fill={rgb, 255:red, 155; green, 155; blue, 155 }  ,fill opacity=0.12 ] (417,322) .. controls (417,317.58) and (420.58,314) .. (425,314) -- (501,314) .. controls (505.42,314) and (509,317.58) .. (509,322) -- (509,346) .. controls (509,350.42) and (505.42,354) .. (501,354) -- (425,354) .. controls (420.58,354) and (417,350.42) .. (417,346) -- cycle ;
		%Rounded Rect [id:dp7024977089857076] 
		\draw  [color={rgb, 255:red, 155; green, 155; blue, 155 }  ,draw opacity=1 ][fill={rgb, 255:red, 155; green, 155; blue, 155 }  ,fill opacity=0.12 ] (417,141) .. controls (417,136.58) and (420.58,133) .. (425,133) -- (501,133) .. controls (505.42,133) and (509,136.58) .. (509,141) -- (509,165) .. controls (509,169.42) and (505.42,173) .. (501,173) -- (425,173) .. controls (420.58,173) and (417,169.42) .. (417,165) -- cycle ;
		%Rounded Rect [id:dp7889794089989934] 
		\draw  [color={rgb, 255:red, 155; green, 155; blue, 155 }  ,draw opacity=1 ][fill={rgb, 255:red, 155; green, 155; blue, 155 }  ,fill opacity=0.12 ] (417,200) .. controls (417,195.58) and (420.58,192) .. (425,192) -- (501,192) .. controls (505.42,192) and (509,195.58) .. (509,200) -- (509,224) .. controls (509,228.42) and (505.42,232) .. (501,232) -- (425,232) .. controls (420.58,232) and (417,228.42) .. (417,224) -- cycle ;
		%Rounded Rect [id:dp17780665664929307] 
		\draw  [color={rgb, 255:red, 155; green, 155; blue, 155 }  ,draw opacity=1 ][fill={rgb, 255:red, 155; green, 155; blue, 155 }  ,fill opacity=0.12 ] (417,262) .. controls (417,257.58) and (420.58,254) .. (425,254) -- (501,254) .. controls (505.42,254) and (509,257.58) .. (509,262) -- (509,286) .. controls (509,290.42) and (505.42,294) .. (501,294) -- (425,294) .. controls (420.58,294) and (417,290.42) .. (417,286) -- cycle ;
		%Rounded Rect [id:dp08184078829998631] 
		\draw  [color={rgb, 255:red, 155; green, 155; blue, 155 }  ,draw opacity=1 ][fill={rgb, 255:red, 126; green, 211; blue, 33 }  ,fill opacity=1 ] (300,188) .. controls (300,183.58) and (303.58,180) .. (308,180) -- (372,180) .. controls (376.42,180) and (380,183.58) .. (380,188) -- (380,212) .. controls (380,216.42) and (376.42,220) .. (372,220) -- (308,220) .. controls (303.58,220) and (300,216.42) .. (300,212) -- cycle ;
		%Rounded Rect [id:dp9437055363891322] 
		\draw  [color={rgb, 255:red, 155; green, 155; blue, 155 }  ,draw opacity=1 ][fill={rgb, 255:red, 126; green, 211; blue, 33 }  ,fill opacity=1 ] (301,269) .. controls (301,264.58) and (304.58,261) .. (309,261) -- (374,261) .. controls (378.42,261) and (382,264.58) .. (382,269) -- (382,293) .. controls (382,297.42) and (378.42,301) .. (374,301) -- (309,301) .. controls (304.58,301) and (301,297.42) .. (301,293) -- cycle ;
		%Straight Lines [id:da4572029334299341] 
		\draw [color={rgb, 255:red, 155; green, 155; blue, 155 }  ,draw opacity=1 ][line width=1.5]    (80,180) -- (157.78,63.33) ;
		\draw [shift={(160,60)}, rotate = 483.69] [fill={rgb, 255:red, 155; green, 155; blue, 155 }  ,fill opacity=1 ][line width=0.08]  [draw opacity=0] (11.61,-5.58) -- (0,0) -- (11.61,5.58) -- cycle    ;
		%Straight Lines [id:da6541321425934448] 
		\draw [color={rgb, 255:red, 155; green, 155; blue, 155 }  ,draw opacity=1 ][line width=1.5]    (120,180) -- (157.78,123.33) ;
		\draw [shift={(160,120)}, rotate = 483.69] [fill={rgb, 255:red, 155; green, 155; blue, 155 }  ,fill opacity=1 ][line width=0.08]  [draw opacity=0] (11.61,-5.58) -- (0,0) -- (11.61,5.58) -- cycle    ;
		%Straight Lines [id:da08973193824238446] 
		\draw [color={rgb, 255:red, 155; green, 155; blue, 155 }  ,draw opacity=1 ][line width=1.5]    (120,240) -- (157.78,183.33) ;
		\draw [shift={(160,180)}, rotate = 483.69] [fill={rgb, 255:red, 155; green, 155; blue, 155 }  ,fill opacity=1 ][line width=0.08]  [draw opacity=0] (11.61,-5.58) -- (0,0) -- (11.61,5.58) -- cycle    ;
		%Straight Lines [id:da22636709326479676] 
		\draw [color={rgb, 255:red, 155; green, 155; blue, 155 }  ,draw opacity=1 ][line width=1.5]    (120,240) -- (156,240) ;
		\draw [shift={(160,240)}, rotate = 180] [fill={rgb, 255:red, 155; green, 155; blue, 155 }  ,fill opacity=1 ][line width=0.08]  [draw opacity=0] (11.61,-5.58) -- (0,0) -- (11.61,5.58) -- cycle    ;
		%Straight Lines [id:da5566318785301065] 
		\draw [color={rgb, 255:red, 155; green, 155; blue, 155 }  ,draw opacity=1 ][line width=1.5]    (120,240) -- (157.78,296.67) ;
		\draw [shift={(160,300)}, rotate = 236.31] [fill={rgb, 255:red, 155; green, 155; blue, 155 }  ,fill opacity=1 ][line width=0.08]  [draw opacity=0] (11.61,-5.58) -- (0,0) -- (11.61,5.58) -- cycle    ;
		%Straight Lines [id:da4869416201679271] 
		\draw [color={rgb, 255:red, 155; green, 155; blue, 155 }  ,draw opacity=1 ][line width=1.5]    (120,300) -- (157.78,356.67) ;
		\draw [shift={(160,360)}, rotate = 236.31] [fill={rgb, 255:red, 155; green, 155; blue, 155 }  ,fill opacity=1 ][line width=0.08]  [draw opacity=0] (11.61,-5.58) -- (0,0) -- (11.61,5.58) -- cycle    ;
		%Straight Lines [id:da7241225236235966] 
		\draw [color={rgb, 255:red, 155; green, 155; blue, 155 }  ,draw opacity=1 ][line width=1.5]    (80,300) -- (158.34,417.5) ;
		\draw [shift={(160,420)}, rotate = 236.31] [color={rgb, 255:red, 155; green, 155; blue, 155 }  ,draw opacity=1 ][line width=1.5]    (14.21,-4.28) .. controls (9.04,-1.82) and (4.3,-0.39) .. (0,0) .. controls (4.3,0.39) and (9.04,1.82) .. (14.21,4.28)   ;
		%Straight Lines [id:da33530746923265875] 
		\draw [color={rgb, 255:red, 126; green, 211; blue, 33 }  ,draw opacity=1 ][line width=1.5]    (260,60) -- (300,200) ;
		%Straight Lines [id:da40286884223752195] 
		\draw [color={rgb, 255:red, 126; green, 211; blue, 33 }  ,draw opacity=1 ][line width=1.5]    (260,120) -- (300,200) ;
		%Straight Lines [id:da11222683072358719] 
		\draw [color={rgb, 255:red, 126; green, 211; blue, 33 }  ,draw opacity=1 ][line width=1.5]    (260,180) -- (300,200) ;
		%Straight Lines [id:da39699658044921193] 
		\draw [color={rgb, 255:red, 126; green, 211; blue, 33 }  ,draw opacity=1 ][line width=1.5]    (260,240) -- (300,200) ;
		%Straight Lines [id:da6504122724693813] 
		\draw [color={rgb, 255:red, 126; green, 211; blue, 33 }  ,draw opacity=1 ][line width=1.5]    (260,300) -- (300,200) ;
		%Straight Lines [id:da19199893350929242] 
		\draw [color={rgb, 255:red, 126; green, 211; blue, 33 }  ,draw opacity=1 ][line width=1.5]    (260,360) -- (300,200) ;
		%Straight Lines [id:da8260837866362043] 
		\draw [color={rgb, 255:red, 126; green, 211; blue, 33 }  ,draw opacity=1 ][line width=1.5]    (260,420) -- (300,200) ;
		%Straight Lines [id:da6370346711010084] 
		\draw [color={rgb, 255:red, 126; green, 211; blue, 33 }  ,draw opacity=1 ][line width=1.5]    (261,420) -- (301,280) ;
		%Straight Lines [id:da20414839850127708] 
		\draw [color={rgb, 255:red, 126; green, 211; blue, 33 }  ,draw opacity=1 ][line width=1.5]    (260,360) -- (300,280) ;
		%Straight Lines [id:da18864076317673573] 
		\draw [color={rgb, 255:red, 126; green, 211; blue, 33 }  ,draw opacity=1 ][line width=1.5]    (260,300) -- (300,280) ;
		%Straight Lines [id:da9343508628518289] 
		\draw [color={rgb, 255:red, 126; green, 211; blue, 33 }  ,draw opacity=1 ][line width=1.5]    (260,240) -- (300,280) ;
		%Straight Lines [id:da04517305353552259] 
		\draw [color={rgb, 255:red, 126; green, 211; blue, 33 }  ,draw opacity=1 ][line width=1.5]    (260,180) -- (300,280) ;
		%Straight Lines [id:da24567568618653834] 
		\draw [color={rgb, 255:red, 126; green, 211; blue, 33 }  ,draw opacity=1 ][line width=1.5]    (260,120) -- (300,280) ;
		%Straight Lines [id:da8425395757812855] 
		\draw [color={rgb, 255:red, 126; green, 211; blue, 33 }  ,draw opacity=1 ][line width=1.5]    (260,60) -- (300,280) ;
		%Shape: Rectangle [id:dp9373282017478466] 
		\draw  [fill={rgb, 255:red, 74; green, 74; blue, 74 }  ,fill opacity=1 ] (547.5,200) -- (603.75,200) -- (603.75,255.26) -- (547.5,255.26) -- cycle ;
		%Shape: Rectangle [id:dp8258119442037437] 
		\draw  [fill={rgb, 255:red, 74; green, 74; blue, 74 }  ,fill opacity=1 ] (555.46,207.89) -- (611.71,207.89) -- (611.71,263.16) -- (555.46,263.16) -- cycle ;
		%Shape: Rectangle [id:dp040967609209594746] 
		\draw  [fill={rgb, 255:red, 74; green, 74; blue, 74 }  ,fill opacity=1 ] (564.23,215.79) -- (620.48,215.79) -- (620.48,271.05) -- (564.23,271.05) -- cycle ;
		%Shape: Rectangle [id:dp3042156197876449] 
		\draw  [fill={rgb, 255:red, 74; green, 74; blue, 74 }  ,fill opacity=1 ] (574.59,225.26) -- (630.84,225.26) -- (630.84,280.53) -- (574.59,280.53) -- cycle ;
		%Shape: Rectangle [id:dp22955626125957718] 
		\draw  [fill={rgb, 255:red, 74; green, 74; blue, 74 }  ,fill opacity=1 ] (583.75,234.74) -- (640,234.74) -- (640,290) -- (583.75,290) -- cycle ;
		%Straight Lines [id:da3121848936924858] 
		\draw [color={rgb, 255:red, 155; green, 155; blue, 155 }  ,draw opacity=1 ][line width=1.5]    (517,152) -- (572.76,185.16) ;
		\draw [shift={(576.2,187.2)}, rotate = 210.74] [fill={rgb, 255:red, 155; green, 155; blue, 155 }  ,fill opacity=1 ][line width=0.08]  [draw opacity=0] (11.61,-5.58) -- (0,0) -- (11.61,5.58) -- cycle    ;
		%Straight Lines [id:da9804292425008414] 
		\draw [color={rgb, 255:red, 155; green, 155; blue, 155 }  ,draw opacity=1 ][line width=1.5]    (516,330) -- (566.11,288.74) ;
		\draw [shift={(569.2,286.2)}, rotate = 500.54] [fill={rgb, 255:red, 155; green, 155; blue, 155 }  ,fill opacity=1 ][line width=0.08]  [draw opacity=0] (11.61,-5.58) -- (0,0) -- (11.61,5.58) -- cycle    ;
		%Straight Lines [id:da9498348064262745] 
		\draw [color={rgb, 255:red, 155; green, 155; blue, 155 }  ,draw opacity=1 ][line width=1.5]    (514.2,274.2) -- (538.84,258.36) ;
		\draw [shift={(542.2,256.2)}, rotate = 507.26] [fill={rgb, 255:red, 155; green, 155; blue, 155 }  ,fill opacity=1 ][line width=0.08]  [draw opacity=0] (11.61,-5.58) -- (0,0) -- (11.61,5.58) -- cycle    ;
		%Straight Lines [id:da553143017068322] 
		\draw [color={rgb, 255:red, 155; green, 155; blue, 155 }  ,draw opacity=1 ][line width=1.5]    (515.2,210.2) -- (539.32,220.44) ;
		\draw [shift={(543,222)}, rotate = 203] [fill={rgb, 255:red, 155; green, 155; blue, 155 }  ,fill opacity=1 ][line width=0.08]  [draw opacity=0] (11.61,-5.58) -- (0,0) -- (11.61,5.58) -- cycle    ;
		%Straight Lines [id:da012047266101314014] 
		\draw [color={rgb, 255:red, 126; green, 211; blue, 33 }  ,draw opacity=1 ][line width=1.5]    (380,200) -- (417.2,154.2) ;
		%Straight Lines [id:da07998854227546759] 
		\draw [color={rgb, 255:red, 126; green, 211; blue, 33 }  ,draw opacity=1 ][line width=1.5]    (380,200) -- (416.2,212.2) ;
		%Straight Lines [id:da9142355685686447] 
		\draw [color={rgb, 255:red, 126; green, 211; blue, 33 }  ,draw opacity=1 ][line width=1.5]    (380,200) -- (416.2,275.2) ;
		%Straight Lines [id:da27077761445957593] 
		\draw [color={rgb, 255:red, 126; green, 211; blue, 33 }  ,draw opacity=1 ][line width=1.5]    (380,200) -- (416.2,334.2) ;
		%Straight Lines [id:da28172695947658655] 
		\draw [color={rgb, 255:red, 126; green, 211; blue, 33 }  ,draw opacity=1 ][line width=1.5]    (382.2,282.2) -- (417.2,154.2) ;
		%Straight Lines [id:da3673944670219347] 
		\draw [color={rgb, 255:red, 126; green, 211; blue, 33 }  ,draw opacity=1 ][line width=1.5]    (382.2,282.2) -- (416.2,212.2) ;
		%Straight Lines [id:da7120289300410392] 
		\draw [color={rgb, 255:red, 126; green, 211; blue, 33 }  ,draw opacity=1 ][line width=1.5]    (382.2,282.2) -- (416.2,275.2) ;
		%Straight Lines [id:da8009655178230701] 
		\draw [color={rgb, 255:red, 126; green, 211; blue, 33 }  ,draw opacity=1 ][line width=1.5]    (382.2,282.2) -- (416.2,334.2) ;
		
		% Text Node
		\draw (170,57) node [anchor=north west][inner sep=0.75pt]  [font=\scriptsize] [align=left] {Nearest Neighbor};
		% Text Node
		\draw (172,106) node [anchor=north west][inner sep=0.75pt]  [font=\scriptsize] [align=left] {\begin{minipage}[lt]{56.68pt}\setlength\topsep{0pt}
				\begin{center}
					Neares Neighbor\\Exact
				\end{center}
				
		\end{minipage}};
		% Text Node
		\draw (192,177) node [anchor=north west][inner sep=0.75pt]   [align=left] {{\scriptsize Linear}};
		% Text Node
		\draw (179,234) node [anchor=north west][inner sep=0.75pt]   [align=left] {{\scriptsize Linear Exact}};
		% Text Node
		\draw (194,296) node [anchor=north west][inner sep=0.75pt]   [align=left] {{\scriptsize Cubic}};
		% Text Node
		\draw (196,357) node [anchor=north west][inner sep=0.75pt]   [align=left] {{\scriptsize Area}};
		% Text Node
		\draw (190,416) node [anchor=north west][inner sep=0.75pt]   [align=left] {{\scriptsize Lanczos}};
		% Text Node
		\draw (302,196) node [anchor=north west][inner sep=0.75pt]   [align=left] {{\scriptsize \textbf{Vollskalierung}}};
		% Text Node
		\draw (309,259) node [anchor=north west][inner sep=0.75pt]   [align=left] {\begin{minipage}[lt]{43.98pt}\setlength\topsep{0pt}
				\begin{center}
					\textbf{{\scriptsize schrittweise}}\\\textbf{{\scriptsize Skalierung}}
				\end{center}
				
		\end{minipage}};
		% Text Node
		\draw (432,149) node [anchor=north west][inner sep=0.75pt]  [font=\scriptsize] [align=left] {$\displaystyle 0,30\ \nicefrac{\mu m}{\text{Pixel}}$};
		% Text Node
		\draw (433,208) node [anchor=north west][inner sep=0.75pt]  [font=\scriptsize] [align=left] {$\displaystyle 0,70\ \nicefrac{\mu m}{\text{Pixel}}$};
		% Text Node
		\draw (431,269) node [anchor=north west][inner sep=0.75pt]  [font=\scriptsize] [align=left] {$\displaystyle 1,10\ \nicefrac{\mu m}{\text{Pixel}}$};
		% Text Node
		\draw (431,329) node [anchor=north west][inner sep=0.75pt]  [font=\scriptsize] [align=left] {$\displaystyle 1,90\ \nicefrac{\mu m}{\text{Pixel}}$};
		% Text Node
		\draw (588.36,250) node [anchor=north west][inner sep=0.75pt]  [color={rgb, 255:red, 255; green, 255; blue, 255 }  ,opacity=1 ] [align=left] {\begin{minipage}[lt]{33.92pt}\setlength\topsep{0pt}
				\begin{center}
					12.992\\Bilder
				\end{center}
				
		\end{minipage}};
		% Text Node
		\draw (27.32,250) node [anchor=north west][inner sep=0.75pt]  [color={rgb, 255:red, 255; green, 255; blue, 255 }  ,opacity=1 ] [align=left] {\begin{minipage}[lt]{28.8pt}\setlength\topsep{0pt}
				\begin{center}
					232\\Bilder
				\end{center}
				
		\end{minipage}};
		
		
	\end{tikzpicture}

	\caption{Planungsmodell (2) zur Erzeugung von Versuchsbildern}
	\label{fig:Planungsmodell(2)}
\end{figure}


\section{Das Analysekonzept}

Wie in Kapitel \ref{sec:StatVersPlanung} auf Seite \pageref{sec:StatVersPlanung} bereits umfänglich dargelegt, ist die statistische Versuchsplanung immer dann ein hervorragendes Werkzeug, wenn es darum geht die Wirkungsweise unterschiedlicher Einflussfaktoren auf eine Zielgröße zu untersuchen. Daher werden nun die dort beschriebenen Zusammenhänge auf den vorliegen Sachverhalt übertragen und angewendet.


\subsection{Systemanalyse}
Die Aufgabe, einen allgemeingültigen Anordnungsklassifikator zu entwickeln, der die beschriebenen Anforderungen erfüllt, ist im Grunde die Lösung eines Mehrgrößen-Optimierungsproblems. Dabei ist es erforderlich zu untersuchen, welche Abhängigkeiten zwischen den \textbf{Einflussgrößen} und Zielgrößen zu bestehen, was zunächst vereinfacht in Abbildung \ref{fig:Systemkontextmodell} dargestellt ist.
       
\begin{figure}[H]
	\centering
	\begin{tikzpicture}[x=0.75pt,y=0.75pt,yscale=-1,xscale=1, scale=1.2, every node/.style={scale=1.2}]
		%uncomment if require: \path (0,222); %set diagram left start at 0, and has height of 222
		
		%Shape: Rectangle [id:dp7648707069859968] 
		\draw  [color={rgb, 255:red, 74; green, 74; blue, 74 }  ,draw opacity=1 ][fill={rgb, 255:red, 255; green, 255; blue, 255 }  ,fill opacity=1 ] (60,40) -- (560,40) -- (560,260) -- (60,260) -- cycle ;
		%Shape: Rectangle [id:dp6660539300811295] 
		\draw  [color={rgb, 255:red, 248; green, 231; blue, 28 }  ,draw opacity=1 ][fill={rgb, 255:red, 248; green, 231; blue, 28 }  ,fill opacity=1 ] (236,152) -- (436,152) -- (436,232) -- (236,232) -- cycle ;
		%Straight Lines [id:da04096259435488303] 
		\draw [color={rgb, 255:red, 126; green, 211; blue, 33 }  ,draw opacity=1 ][line width=4.5]    (289,108) -- (289,140) ;
		\draw [shift={(289,148)}, rotate = 270] [fill={rgb, 255:red, 126; green, 211; blue, 33 }  ,fill opacity=1 ][line width=0.08]  [draw opacity=0] (24.11,-11.58) -- (0,0) -- (24.11,11.58) -- cycle    ;
		%Straight Lines [id:da9081230453690534] 
		\draw [color={rgb, 255:red, 126; green, 211; blue, 33 }  ,draw opacity=1 ][line width=4.5]    (390,108) -- (390,140) ;
		\draw [shift={(390,148)}, rotate = 270] [fill={rgb, 255:red, 126; green, 211; blue, 33 }  ,fill opacity=1 ][line width=0.08]  [draw opacity=0] (24.11,-11.58) -- (0,0) -- (24.11,11.58) -- cycle    ;
		%Shape: Rectangle [id:dp9833999336185018] 
		\draw  [fill={rgb, 255:red, 74; green, 74; blue, 74 }  ,fill opacity=1 ] (113.75,147) -- (170,147) -- (170,202.26) -- (113.75,202.26) -- cycle ;
		%Shape: Rectangle [id:dp6474898484381659] 
		\draw  [fill={rgb, 255:red, 74; green, 74; blue, 74 }  ,fill opacity=1 ] (105.71,154.89) -- (161.96,154.89) -- (161.96,210.16) -- (105.71,210.16) -- cycle ;
		%Shape: Rectangle [id:dp28214539132657324] 
		\draw  [fill={rgb, 255:red, 74; green, 74; blue, 74 }  ,fill opacity=1 ] (97.48,163.79) -- (153.73,163.79) -- (153.73,219.05) -- (97.48,219.05) -- cycle ;
		%Shape: Rectangle [id:dp8208799659588357] 
		\draw  [fill={rgb, 255:red, 74; green, 74; blue, 74 }  ,fill opacity=1 ] (87.84,173.26) -- (144.09,173.26) -- (144.09,228.53) -- (87.84,228.53) -- cycle ;
		%Shape: Rectangle [id:dp42497443725878625] 
		\draw  [fill={rgb, 255:red, 74; green, 74; blue, 74 }  ,fill opacity=1 ] (78,182.74) -- (134.25,182.74) -- (134.25,238) -- (78,238) -- cycle ;
		%Notched Right Arrow [id:dp7255518395939695] 
		\draw  [color={rgb, 255:red, 155; green, 155; blue, 155 }  ,draw opacity=0.52 ][fill={rgb, 255:red, 126; green, 211; blue, 33 }  ,fill opacity=1 ] (175,182) -- (205,182) -- (205,172) -- (225,192) -- (205,212) -- (205,202) -- (175,202) -- (185,192) -- cycle ;
		%Notched Right Arrow [id:dp4043296670143506] 
		\draw  [color={rgb, 255:red, 126; green, 211; blue, 33 }  ,draw opacity=1 ][fill={rgb, 255:red, 126; green, 211; blue, 33 }  ,fill opacity=1 ] (444,182) -- (501.6,182) -- (501.6,172) -- (540,192) -- (501.6,212) -- (501.6,202) -- (444,202) -- (454,192) -- cycle ;
		
		% Text Node
		\draw (260.06,175.75) node [anchor=north west][inner sep=0.75pt]  [color={rgb, 255:red, 74; green, 74; blue, 74 }  ,opacity=1 ,rotate=-0.19] [align=left] {\begin{minipage}[lt]{108.72pt}\setlength\topsep{0pt}
				\begin{center}
					\textbf{AMGuss (System)}\\{\scriptsize \textbf{\textit{Ursache-/Wirgungsbez.}}}
				\end{center}
				
		\end{minipage}};
		% Text Node
		\draw (347,48) node [anchor=north west][inner sep=0.75pt]  [color={rgb, 255:red, 74; green, 74; blue, 74 }  ,opacity=1 ] [align=left] {\begin{minipage}[lt]{57.69pt}\setlength\topsep{0pt}
				\begin{center}
					\textbf{Störgrößen}\\{\scriptsize \textbf{\textit{Ausgangs-}}}\\{\scriptsize \textbf{\textit{kalibrierungen}}}
				\end{center}
				
		\end{minipage}};
		% Text Node
		\draw (238,48) node [anchor=north west][inner sep=0.75pt]  [color={rgb, 255:red, 74; green, 74; blue, 74 }  ,opacity=1 ] [align=left] {\begin{minipage}[lt]{69.03pt}\setlength\topsep{0pt}
				\begin{center}
					\textbf{Steuergrößen}\\{\scriptsize \textbf{\textit{Interpolations-}}}\\{\scriptsize \textbf{\textit{algorithmen}}}
				\end{center}
				
		\end{minipage}};
		% Text Node
		\draw (83.32,196.37) node [anchor=north west][inner sep=0.75pt]  [color={rgb, 255:red, 255; green, 255; blue, 255 }  ,opacity=1 ] [align=left] {\begin{minipage}[lt]{32.9pt}\setlength\topsep{0pt}
				\begin{center}
					{\scriptsize Eingaben}\\{\scriptsize (Bilder)}
				\end{center}
				
		\end{minipage}};
		% Text Node
		\draw (459,188) node [anchor=north west][inner sep=0.75pt]  [font=\scriptsize,color={rgb, 255:red, 74; green, 74; blue, 74 }  ,opacity=1 ] [align=left] {\textbf{Zielgrößen}};
	\end{tikzpicture}
	\caption{Modell zur Darstellung der Ursache-/Wirkungsbeziehungen im Systemkontext}
	\label{fig:Systemkontextmodell}
\end{figure}

\noindent Im weiteren Verlauf dieses Kapitels werden nun die in der Abbildung dargestellten Steuer-, Stör- und Zielgrößen genauer beschrieben und definiert.

\subsection{Definition der Zielgrößen}
\label{subsec:DefZiel}

Die Definition von Zielgrößen stellt eine wichtige Grundlage für die in dieser Arbeit durchgeführten Untersuchung dar, da damit die Wirkungsweise der Auswirkungen von Interpolation bei der Bildskalierung (vgl. dazu auch Abschnitt \ref{sec:Problemstellung}) quantitativ dargestellt und dann im weiteren Verlauf mit statistischen Methoden weiter analysiert werden können, wie das dann auch in Kapitel \ref{ch:Evaluation} ab Seite \pageref{ch:Evaluation} getan wird.
\\\\
Bei den Zielgrößen handelt es sich also um Kennzahlen die angeben, wie groß ein Fehler ist. Gemeint sind damit die Messfehler, die dadurch entstehen, das ein Bild unter Anwendung eines Interpolationsverfahrens skaliert wird, um dadurch die Kalibrierung zu ändern. Jedes skalierte Bild kann dann also mit seinem un-skalierten Original verglichen und die Messfehler ermittelt werden. Nachstehende Gleichung \ref{eq:MFAkl} zeigt zunächst die Berechnung des Messfehlers im Bereich der Anordnungsklassen.

\begin{equation}\label{eq:MFAkl}
	\boxed{\mathbf{MF_{AK}	=	\sum_{i=1}^{5} |\left(AKO_i - AKV_i\right)}|}
\end{equation}
mit
\begin{conditions}
	AKO	&	Messwert der Anordnungsklasse i im Originalbild mit i $\in$ (a, b, c, d, e)\\
	AKV	&	Messwert der Anordnungsklasse i Vergleichsbild mit i $\in$ (a, b, c, d, e)
\end{conditions}

\noindent Wie man sieht, wird dabei einfach jede Anordnungsklasse des Originalbildes mit der des jeweiligen skalierten Vergleichsbildes verglichen und die Beträge der Differenzen über alle Klassen aufsummiert.
\\\\
\noindent Weitere wichtige Erkenntnisgewinne können dadurch erzielt werden, indem die Messfehler nicht nur bezogen auf die Anordnungsklassen, sondern auch in Bezug auf die Größenklassen (anzahl- und flächengewichtet) betrachtet werden, deren Berechnung in den Gleichungen \ref{eq:MFGKa} und \ref{eq:MFGKf} gezeigt wird.

\begin{equation}\label{eq:MFGKa}
	\boxed{\mathbf{MF_{GKA} = \sum_{j=1}^{8} |\left(GKAO_j - GKAV_j\right)|}}
\end{equation}
mit
\begin{conditions}
	GKAO	&	Messwert der Größenklasse (anzahlgewichtet) j im Originalbild\\
	GKAV	&	Messwert der Größenklasse (anzahlgewichtet) j im Vergleichsbild
\end{conditions}

\noindent \textbf{sowie}

\begin{equation}\label{eq:MFGKf}
	\boxed{\mathbf{MF_{GKF} = \sum_{k=1}^{8} |\left(GKFO_k - GKFV_k\right)|}}
\end{equation}
mit
\begin{conditions}
	GKFO	&	Messwert der Größenklasse (flächengewichtet) k im Originalbild\\
	GKFV	&	Messwert der Größenklasse (flächengewichtet) k im  Vergleichsbild
\end{conditions}

\noindent Da bei der multivariaten statistischen Analyse, welche im Rahmen der statistischen Versuchsplanung durchgeführt wird, jedoch nur eine Zielgröße möglich ist, werden die berechneten Einzelwerte nun auch noch zu einer Zielgröße $MF_{ges}$ zusammengefasst (Gleichung \ref{eq:MFges}).

\begin{equation}\label{eq:MFges}
	\boxed{\mathbf{MF_{ges} = MF_{AK} + MF_{GKA} + MF_{GKF}}}
\end{equation}

\noindent Im nächsten Abschnitt geht es nun darum zu bestimmen, welches die Stör- und welches die Einflussgrößen im zu analysierenden Kontext sind.

\subsection{Definition der Einfluss- und Störgrößen}
\label{subsec:DefEinflusStoer}

Hierbei geht es nun um die Definition der bereits in Abbildung \ref{fig:Systemkontextmodell} dargestellten Steuer- und Einflussgrößen. In der vorliegenden Arbeit besteht der Systemkontext aus zwei Steuer- und einer Störgröße, die wie folgt beschrieben werden:

\begin{itemize}
	\item \textbf{Interpolationsalgorithmus und Interpolationsmodus (Steuergrößen)} \\
	Dies sind sog. Steuergrößen, da sie direkt beeinflusst werden können. Es stehen insgesamt 7 verschiedene Interpolationsalgorithmen zur Verfügung und die Bilder können sowohl voll als auch schrittweise skaliert werden.
	\item \textbf{Kalibrierung} \\
	Diese Größe (in $\nicefrac{\mu m}{\text{Pixel}}$) kann dagegen nicht direkt beeinflusst werden und stellt somit eine Störgröße dar. Schließlich ist es ja das Ziel eines allgemeingültigen Anordnungsklassifikators, das Bilder, unabhängig davon mit welcher Kalibrierung sie aufgenommen wurden, unter Anwendung eines Standardklassifikators gemessen werden können.
\end{itemize}

\noindent Diese beiden Größen sind später vor allem für die Interpretation der Ergebnisse aus der statistischen Versuchsplanung relevant (vgl. dazu auch Kapitel \ref{ch:Evaluation} und dort insbesondere Abschnitt \ref{subsubsec:AuswertungStatVpl}).

\section{Versuchsdurchführung}

Das Hauptziel in dieser Arbeit besteht darin, spezifische Daten zu erfassen, auszuwerten und zu interpretieren. Abbildung \ref{fig:Versuchsdurchführung} veranschaulicht den Ablauf der Versuchsdurchführung.

\begin{figure}[H]
	\centering
	\begin{tikzpicture}[x=0.75pt,y=0.75pt,yscale=-1,xscale=1, scale=1.0, every node/.style={scale=1.0}]
		%uncomment if require: \path (0,222); %set diagram left start at 0, and has height of 222
		
	%Shape: Rectangle [id:dp03419872242718203] 
	\draw  [color={rgb, 255:red, 74; green, 74; blue, 74 }  ,draw opacity=1 ][fill={rgb, 255:red, 255; green, 255; blue, 255 }  ,fill opacity=1 ] (3,21) -- (656,21) -- (656,321) -- (3,321) -- cycle ;
	%Shape: Rectangle [id:dp218360140301183] 
	\draw  [fill={rgb, 255:red, 74; green, 74; blue, 74 }  ,fill opacity=1 ] (51.5,102) -- (107.75,102) -- (107.75,157.26) -- (51.5,157.26) -- cycle ;
	%Shape: Rectangle [id:dp3937315611309855] 
	\draw  [fill={rgb, 255:red, 74; green, 74; blue, 74 }  ,fill opacity=1 ] (43.46,109.89) -- (99.71,109.89) -- (99.71,165.16) -- (43.46,165.16) -- cycle ;
	%Shape: Rectangle [id:dp5265848440532068] 
	\draw  [fill={rgb, 255:red, 74; green, 74; blue, 74 }  ,fill opacity=1 ] (35.23,118.79) -- (91.48,118.79) -- (91.48,174.05) -- (35.23,174.05) -- cycle ;
	%Shape: Rectangle [id:dp8134408230981591] 
	\draw  [fill={rgb, 255:red, 74; green, 74; blue, 74 }  ,fill opacity=1 ] (25.59,128.26) -- (81.84,128.26) -- (81.84,183.53) -- (25.59,183.53) -- cycle ;
	%Shape: Rectangle [id:dp12628310483676164] 
	\draw  [fill={rgb, 255:red, 74; green, 74; blue, 74 }  ,fill opacity=1 ] (15.75,137.74) -- (72,137.74) -- (72,193) -- (15.75,193) -- cycle ;
	
	%Image [id:dp19023457749906703] 
	\draw (233.53,154.36) node  {\includegraphics[width=116.3pt,height=104.47pt]{pics/measure_1}};
	%Shape: Circle [id:dp728673139479594] 
	\draw  [color={rgb, 255:red, 126; green, 211; blue, 33 }  ,draw opacity=1 ][fill={rgb, 255:red, 126; green, 211; blue, 33 }  ,fill opacity=1 ] (219,46) .. controls (219,38.82) and (224.82,33) .. (232,33) .. controls (239.18,33) and (245,38.82) .. (245,46) .. controls (245,53.18) and (239.18,59) .. (232,59) .. controls (224.82,59) and (219,53.18) .. (219,46) -- cycle ;
	%Image [id:dp55818248258904] 
	\draw (565.53,154.36) node  {\includegraphics[width=116.3pt,height=104.47pt]{pics/measure_1}};
	%Shape: Circle [id:dp678145460257517] 
	\draw  [color={rgb, 255:red, 126; green, 211; blue, 33 }  ,draw opacity=1 ][fill={rgb, 255:red, 126; green, 211; blue, 33 }  ,fill opacity=1 ] (551,46) .. controls (551,38.82) and (556.82,33) .. (564,33) .. controls (571.18,33) and (577,38.82) .. (577,46) .. controls (577,53.18) and (571.18,59) .. (564,59) .. controls (556.82,59) and (551,53.18) .. (551,46) -- cycle ;
	%Shape: Rectangle [id:dp8368148107013256] 
	\draw  [color={rgb, 255:red, 248; green, 231; blue, 28 }  ,draw opacity=1 ][fill={rgb, 255:red, 248; green, 231; blue, 28 }  ,fill opacity=1 ][line width=0.75]  (348,120) -- (443,120) -- (443,160) -- (348,160) -- cycle ;
	%Shape: Circle [id:dp1828429723656111] 
	\draw  [color={rgb, 255:red, 126; green, 211; blue, 33 }  ,draw opacity=1 ][fill={rgb, 255:red, 126; green, 211; blue, 33 }  ,fill opacity=1 ] (382,94) .. controls (382,86.82) and (387.82,81) .. (395,81) .. controls (402.18,81) and (408,86.82) .. (408,94) .. controls (408,101.18) and (402.18,107) .. (395,107) .. controls (387.82,107) and (382,101.18) .. (382,94) -- cycle ;
	%Shape: Rectangle [id:dp6879768109790683] 
	\draw  [color={rgb, 255:red, 248; green, 231; blue, 28 }  ,draw opacity=1 ][fill={rgb, 255:red, 248; green, 231; blue, 28 }  ,fill opacity=1 ] (294,254) -- (499,254) -- (499,305) -- (294,305) -- cycle ;
	%Bend Up Arrow [id:dp5130011350568686] 
	\draw  [color={rgb, 255:red, 126; green, 211; blue, 33 }  ,draw opacity=1 ][fill={rgb, 255:red, 126; green, 211; blue, 33 }  ,fill opacity=1 ] (240,236) -- (240,269) -- (258,269) -- (258,260) -- (282,278) -- (258,296) -- (258,287) -- (222,287) -- (222,236) -- cycle ;
	%Bend Up Arrow [id:dp5493434724310982] 
	\draw  [color={rgb, 255:red, 126; green, 211; blue, 33 }  ,draw opacity=1 ][fill={rgb, 255:red, 126; green, 211; blue, 33 }  ,fill opacity=1 ] (562.7,235) -- (562.7,268.55) -- (536.4,268.55) -- (536.4,259.4) -- (512,277.7) -- (536.4,296) -- (536.4,286.85) -- (581,286.85) -- (581,235) -- cycle ;
	%Shape: Circle [id:dp4252487137635921] 
	\draw  [color={rgb, 255:red, 126; green, 211; blue, 33 }  ,draw opacity=1 ][fill={rgb, 255:red, 126; green, 211; blue, 33 }  ,fill opacity=1 ] (384,227) .. controls (384,219.82) and (389.82,214) .. (397,214) .. controls (404.18,214) and (410,219.82) .. (410,227) .. controls (410,234.18) and (404.18,240) .. (397,240) .. controls (389.82,240) and (384,234.18) .. (384,227) -- cycle ;
	%Chevron Arrow [id:dp36704668199013146] 
	\draw  [color={rgb, 255:red, 126; green, 211; blue, 33 }  ,draw opacity=1 ][fill={rgb, 255:red, 126; green, 211; blue, 33 }  ,fill opacity=1 ] (118,120) -- (135.4,120) -- (147,140) -- (135.4,160) -- (118,160) -- (129.6,140) -- cycle ;
	%Chevron Arrow [id:dp38485139245994215] 
	\draw  [color={rgb, 255:red, 126; green, 211; blue, 33 }  ,draw opacity=1 ][fill={rgb, 255:red, 126; green, 211; blue, 33 }  ,fill opacity=1 ] (307,120) -- (324.4,120) -- (336,140) -- (324.4,160) -- (307,160) -- (318.6,140) -- cycle ;
	%Chevron Arrow [id:dp7077292276832701] 
	\draw  [color={rgb, 255:red, 126; green, 211; blue, 33 }  ,draw opacity=1 ][fill={rgb, 255:red, 126; green, 211; blue, 33 }  ,fill opacity=1 ] (451,120) -- (468.4,120) -- (480,140) -- (468.4,160) -- (451,160) -- (462.6,140) -- cycle ;
	
	% Text Node
	\draw (21.07,149.37) node [anchor=north west][inner sep=0.75pt]  [color={rgb, 255:red, 255; green, 255; blue, 255 }  ,opacity=1 ] [align=left] {\begin{minipage}[lt]{33.93pt}\setlength\topsep{0pt}
			\begin{center}
				232\\Bilder
			\end{center}
			
	\end{minipage}};
	% Text Node
	\draw (355,128) node [anchor=north west][inner sep=0.75pt]  [font=\scriptsize,color={rgb, 255:red, 74; green, 74; blue, 74 }  ,opacity=1 ] [align=left] {\begin{minipage}[lt]{54.7pt}\setlength\topsep{0pt}
			\begin{center}
				\textbf{Transformation}\\ $\displaystyle \to 0,5\ \nicefrac{\mu m}{\text{Pixel}}$
			\end{center}
			
	\end{minipage}};
	% Text Node
	\draw (511.27,68.52) node [anchor=north west][inner sep=0.75pt]   [align=left] {\textbf{{\scriptsize Messung mit AMGuss}}};
	% Text Node
	\draw (559,41) node [anchor=north west][inner sep=0.75pt]  [color={rgb, 255:red, 74; green, 74; blue, 74 }  ,opacity=1 ] [align=left] {\textbf{3}};
	% Text Node
	\draw (390,89) node [anchor=north west][inner sep=0.75pt]  [color={rgb, 255:red, 74; green, 74; blue, 74 }  ,opacity=1 ] [align=left] {\textbf{2}};
	% Text Node
	\draw (179.27,68.52) node [anchor=north west][inner sep=0.75pt]   [align=left] {\textbf{{\scriptsize Messung mit AMGuss}}};
	% Text Node
	\draw (227,41) node [anchor=north west][inner sep=0.75pt]  [color={rgb, 255:red, 74; green, 74; blue, 74 }  ,opacity=1 ] [align=left] {\textbf{1}};
	% Text Node
	\draw (391,222) node [anchor=north west][inner sep=0.75pt]  [color={rgb, 255:red, 74; green, 74; blue, 74 }  ,opacity=1 ] [align=left] {\textbf{4}};
	% Text Node
	\draw (304,265) node [anchor=north west][inner sep=0.75pt]  [color={rgb, 255:red, 74; green, 74; blue, 74 }  ,opacity=1 ] [align=left] {\begin{minipage}[lt]{134.83pt}\setlength\topsep{0pt}
			\begin{center}
				Bestimmung der Abweichung\\\textbf{= Zielwert}
			\end{center}
			
	\end{minipage}};
	\end{tikzpicture}
	\caption{Schematische Darstellung des Modells zur Versuchsdurchführung}
	\label{fig:Versuchsdurchführung}
\end{figure}

\noindent Viele der folgenden Darstellungen in dieser Arbeit können vor dem Hintergrund des hier dargestellten Zusammenhangs sehr gut nachvollzogen werden.


\chapter{Umsetzung des Konzeptes}
\label{ch:Umsetzung}

Dieses Kapitel beschäftigt sich mit der Umsetzung, also dem Teil der vorliegenden Arbeit, bei dem es darum ging, die Aufgabe, nämlich insgesamt 12.992 Lamellengraphit-Messungen mit der Software AMGuss zur Datengenerierung, sowie 232 statistische Versuchsplanungen mir der Software Minitab (einer Statistik-Software) durchzuführen, umzusetzen. Da eine manuelle Durchführung der Messungen jeden zeitlichen Rahmen gesprengt hätte, wurde dieser Prozess weitestgehend automatisiert.
\\\\
\noindent Für die Implementierung wurde die Skriptsprache Python gewählt, da diese auch die Möglichkeit bietet, einen objektorientierten Programmierstil umzusetzen. Darüber hinaus war es durch die Einbindung der Bibliothek PyAutoIt möglich, die gesamte Funktionalität von AutoIt (siehe auch Abschnitt \ref{sec:UeberblickAutoIt} auf Seite \pageref{sec:UeberblickAutoIt}) verfügbar und nutzbar zu machen. Dadurch konnte die Automatisierung der durchzuführenden Messungen in Verbindung mit einem objektorientierten Programmierstil sehr gut umgesetzt werden.
\\\\
\noindent Nach einer Darstellung der Komponentenstruktur als Überblick über die Implementierung folgt dann in weiteren Abschnitten die Beschreibung der zu Implementierenden Funktionalität anhand von Aktivitätsdiagrammen und schließlich Klassendiagrammen, die zeigen, wie die Objektorientierung dabei geholfen hat, dass die einzelnen Aufgaben in Teilaufgaben zerlegt und separat durchgeführt werden konnten.

\section{Darstellung der Komponenten}
\label{sec:Komponentendarstellung}

Wie bereits angedeutet, wurde die Gesamtaufgabe auf verschiedene, aufeinander aufbauende Arbeitsschritte aufgeteilt. Diese Komponenten erzeugen Daten, speichern diese in einer dafür implementierten Datenstruktur, die dann schließlich serialisiert, gespeichert und so für die Nutzung weitere Komponenten verfügbar gemacht wird. Abbildung \ref{fig:Komponentendiagramm} zeigt die einzelnen Bestandteile, wobei anzumerken ist, dass diese nicht, wie im herkömmlichen Sinne der objektorientierten Programmierung über implementierte Schnittstellen verbunden sind. Vielmehr handelt es sich um völlig getrennte Funktionseinheiten. Die Pfeil-Konnektoren, welche die Komponenten miteinander verbinden, symbolisieren also Abhängigkeitsbeziehungen in dem Sinne, dass die Komponente, auf welche die Pfeilspitze zeigt, die Ergebnisse derjenigen am anderen Endes des Pfeils voraussetzt bzw. weiterverarbeitet.

\begin{figure}[H]
	\centering
	\boxed{\includegraphics[scale=0.6]{pics/components}}
	\caption{Darstellung der Systemkomponenten und Abhängigkeiten}
	\label{fig:Komponentendiagramm}
\end{figure}



\section{Aktivitätsdiagramme}

\label{sec:Workflows}
Aktivitätsdiagramme sind Teil von UML (Unifiend Modelling Language) zur Modellierung von Software-Systemen und ein geeignetes Werkzeug, die Kontroll- und Datenflüsse der für diese Arbeit entwickelten Skript-Bibliothek anschaulich darzustellen. Dadurch werden die zu entwickelnden funktionalen Abläufe eindeutig spezifiziert - und zwar unabhängig davon, wie dies später programmatisch und algorithmisch umgesetzt wird. In den folgenden Unterabschnitten werden nun, nach unterschiedlichen Funktionseinheiten gegliedert, diese anhand von Aktivitätsdiagrammen eindeutig beschrieben, um so dem Leser zunächst einen Überblick über die funktionalen Zusammenhänge zu geben. Die Reihenfolge der Darstellung folgt dabei den dargestellten Abhängigkeiten im Komponentendiagramm.

\subsection{Bildskalierung in unterschiedliche Kalibrierungsstufen}

Die Erzeugung von Bildern in unterschiedlichen Kalibrierungsstufen entspricht der im Abschnitt \ref{sec:ErzeugungVonBildern} auf Seite \pageref{sec:ErzeugungVonBildern} bereits eingehend beschriebenen Vorgehensweise und die nachstehende Abbildung \ref{fig:Bildskalierung} zeigt den funktionalen Ablauf.

\begin{figure}[H]
	\centering
	\boxed{\includegraphics[scale=0.56]{pics/flow_scaling_images}}
	\caption{Aktivitätsdiagramm zur Erstellung einer Bildserie mit AMGuss}
	\label{fig:Bildskalierung}
\end{figure}

\noindent Die Skalierung der Bilder erfolgt dabei jeweils in 4 Kalibrierungsstufen, mit 7 verschiedenen Interpolationsalgorithmen, wie in Kapitel \ref{sec:ErzeugungVonBildern} auf Seite \pageref{sec:ErzeugungVonBildern} bereits ausführlich beschrieben wurde. Am Ende des Vorgangs existieren also von jedem Bild insgesamt 56 skalierte Kopien. Die Pfade zu den jeweiligen Speicherorten werden in einer entsprechenden Datenstruktur gespeichert und so für die weitere Verwendung bereitgestellt.
\\\\
Es sei an dieser Stelle auf den Unterschied zwischen der schrittweisen und der Vollskalierung hingewiesen. Der Unterschied besteht darin, dass bei der Vollskalierung das Bild entsprechend dem berechneten Skalierungsfaktor im  einem Schritt skaliert wird. Bei der schrittweisen Skalierung, wird das Bild in mehreren gleichmäßigen Schritten skaliert. Der Skalierungsfaktor berechnet sich in beiden Fällen wie folgt:

\begin{equation}
	S_f = \frac{\text{Ausgangskalibrierung}}{Zielkalibrierung}
\end{equation}
mit
\begin{conditions}
	S_f & Skalierungsfaktor
\end{conditions}

\noindent Für die Berechnung der Skalierungsschritte gilt zunächst folgende Festlegung:
\begin{equation}
	F = 
	\begin{cases}
		 2,		&	\text{für } S_f \leq 1,67\\
		 3, 	&	\text{sonst}
	\end{cases}
\end{equation}
mit
\begin{conditions}
	F & Anzahl der Skalierungsschritte
\end{conditions}

\noindent Unter dieser Prämisse berechnet sich dann der Skalierungsfaktor für jeden durchzuführenden Skalierungsschritt wie folgt:

\begin{equation}
	S_s = \sqrt[F]{S_f}
\end{equation}
mit
\begin{conditions}
	S_{s} & Skalierungsfaktor / Schritt
\end{conditions}

\noindent Die auf diese Weise schrittweise skalierten Bilder sind im Ergebnis gleich groß und weisen bei gleicher Ausgangskalibrierung im Ergebnis eine äquivalente Zielkalibrierung (Bildgröße) auf.
\\\\
\noindent Analog dazu wird auch die Skalierung der Trainingsbilder durchgeführt, jedoch nur mit Nächster-Nachbar-Interpolation, da bei diesem Interpolationsverfahren die Kanten erhalten bleiben und durch die Skalierung nicht verändert werden. Diese sog. Trainingsbilder sind die Bilder, welche von AMGuss automatisch bei der Erstellung eines Anordnungsklassifikators erzeugt werden (siehe dazu auch Kapitel \ref{subsec: ErstellungAnordnungsklassifAMGuss} auf Seite \pageref{subsec: ErstellungAnordnungsklassifAMGuss}). Für die vorliegende Arbeit wurde ein sog. Standardklassifikator mit insgesamt 50 Bildern (Kalibrierung 0,5 $\nicefrac{\mu m}{\text{Pixel}}$) trainiert, die so ausgewählt wurden, dass die im Abschnitt \ref{subsec:MethodenBestAnordnungsklassen} auf Seite \pageref{subsec:MethodenBestAnordnungsklassen} dargestellten Anordnungsklassen etwa in gleichmäßiger Verteilung über alle Bilder vorhanden sind.

\subsection{Erstellung einer Bildserie mit AMGuss}
\label{subsec:FlowErstellungBildserieAMGuss}

Um eine Lamellenguss-Auswertung mit der Software AMGuss durchzuführen, sollte jedes zu messende Bild eine in einer Bildserie vorliegen. Es können grundsätzlich auch Serien mit mehreren Bildern erstellt und auf einmal gemessen werden, jedoch hat sich dies nach Durchführung mehrerer Versuche im vorliegenden Kontext als nicht zweckmäßig erwiesen. Daher wurde für jedes zu messende Bild eine eigene Bildserie erstellt. Hierbei geht es im Gegensatz zur Bildskalierung darum, die Durchführung der Messungen mit der Software AMGuss zu steuern. Der funktionale Ablauf hierbei ist in der folgenden Abbildung \ref{fig:FlowBildserieErstellen} dargestellt.


\begin{figure}[H]
	\centering
	\boxed{\includegraphics[scale=0.8]{pics/flow_create_picture_set_amguss}}
	\caption{Aktivitätsdiagramm zur Erstellung einer Bildserie mit AMGuss}
	\label{fig:FlowBildserieErstellen}
\end{figure}

\noindent Im Ergebnis wird dadurch von AMGuss eine Textdatei.cia erzeugt, welche Informationen über das Bild, also den Pfad zum Speicherort und ggf. noch weitere Bildinformationen enthält. Den Inhalt einer solchen Datei zeigt die nachstehende Abbildung \ref{fig:CiaContent}.

\begin{figure}[H]
	\centering
	\boxed{\includegraphics[scale=0.9]{pics/CiaContent}}
	\caption{Inhalt einer .cia-Datei für eine Bildserie}
	\label{fig:CiaContent}
\end{figure}

\noindent Am Ende wird die erzeugte cia-Datei gespeichert, sowie auch der Pfad zum Speicherort, sodass bei der späteren Verwendung darauf zugegriffen werden kann.

\subsection{Durchführung einer Lamellengraphit-Messung mit AMGuss}
\label{subsec:FlowLamellengussmessung}
Die Durchführung der Messungen mit der Software AMGuss bildet, neben der Evaluation in Kapitel \ref{ch:Evaluation} ab Seite \pageref{ch:Evaluation}, den Kern dieser Arbeit.  Denn dadurch werden die Daten erzeugt, auf der Grundlage dessen erst eine Bewertung der für die Skalierung angewendeten Interpolationsverfahren möglich wird (Abbildung \ref{fig:FlowLamellengussMessung}). Es geht also hierbei, wie auch schon bei der Erstellung der Bildserien, darum, den Messablauf der Software zu automatisieren.

\begin{figure}[H]
	\centering
	\boxed{\includegraphics[scale=0.8]{pics/flow_lamellar_graphite_evaluation}}
	\caption{Aktivitätsdiagramm zur Durchführung einer Lamellengraphit-Messung mit AMGuss}
	\label{fig:FlowLamellengussMessung}
\end{figure}

\noindent Nach jeder Einzelmessung werden die Messergebnisse auch hier wieder, wie bei der Bildskalierung und bei der Erstellung von Bildserien auch, in der im Ressourcenverzeichnis erstellten Ordnerstruktur gespeichert, genauso wie die Pfade zu den Dateien zur Verwendung für nachfolgende Aufgaben.

\subsection{Extraktion der Messergebnisse aus den mit AMGuss gespeicherten Messungen}
\label{subsec:FlowExtraktionMessergebnisse}

Bevor die Messdaten ausgewertet werden können, müssen sie in irgendeiner weiterverarbeitbaren Form bereitgestellt werden. AMGuss speichert die Messergebnisse in Form von reinen Textdateien, in denen neben vielen anderen Informationen zur jederzeitigen Wiederherstellung des Messvorgangs vor allem auch die für diese Arbeit wichtigen Messergebnisse zu finden sind. Dies sind die Messwerte zu den Anordnungsklassen A-E sowie den Größenklassen 1-8 (anzahlgewichtet und flächengewichtet), da diese Messwerte repräsentativ zur Beurteilung der Interpolationsverfahren im Hinblick auf die durch Skalierung induzierten Messungenauigkeit sind. Die folgenden beiden Abbildungen zeigen, in welcher Form diese Daten gespeichert sind und zwar für die Anordnungsklassen (Abbildung \ref{fig:KlassifikationsergebnisseAkl}) sowie für die Größenklassen (Abbildung \ref{fig:KlassifikationsergebnisseGK}).

\begin{figure}[H]
	\centering
	\boxed{\includegraphics[scale=0.9]{pics/classify_data}}
	\caption{Beispiel für Speicherung der Messergebnisse für die Anordnungsklassen}
	\label{fig:KlassifikationsergebnisseAkl}
\end{figure}

\begin{figure}[H]
	\centering
	\boxed{\includegraphics[scale=0.6]{pics/global_size}}
	\caption{Beispiel für Speicherung der Messergebnisse für die Größenklassen}
	\label{fig:KlassifikationsergebnisseGK}
\end{figure}

\noindent In diesem Arbeitsschritt besteht also die Aufgabe darin, diese Daten in 13.224 verschiedenen Messergebnisdateien (für jedes Bild eine) zu finden und so zu speichern, dass sie jedem einzelnen Bild eindeutig zugeordnet werden können. Für die Organisation der Datenspeicherung wird an dieser Stelle auf Kapitel \ref{sec:Klassendiagramme} verwiesen. Zunächst soll mit dem im folgenden abgebildeten Aktivitätsdiagramm die Aufgabe näher spezifiziert und dokumentiert werden (Abbildung \ref{fig:FlowDatenextraktion}).

\begin{figure}[H]
	\centering
	\boxed{\includegraphics[scale=0.8]{pics/flow_parsing_data}}
	\caption{Aktivitätsdiagramm zur Extraktion von Messdaten}
	\label{fig:FlowDatenextraktion}
\end{figure}

\subsection{Durchführung der statistischen Analyse mit Minitab}
\label{subsec:FlowDoeMinitab}
Wie an früherer Stelle bereits beschrieben, handelt es sich bei der statistischen Versuchsplanung um eine statistische Methode zur multivariaten Untersuchung von Einflussgrößen sowie der Bestimmung der Haupteffekte in einem System (vgl. dazu auch Abschnitt \ref{sec:StatVersPlanung} auf Seite \pageref{sec:StatVersPlanung}). Diese Analyse wurde mit einer Statistik-Software Namens Minitab ebenfalls automatisiert durchgeführt und das Aktivitätsdiagramm (Abbildung \ref{fig:FlowVersuchsplanung}) zeigt den zu implementierenden funktionalen Ablauf.

\begin{figure}[H]
	\centering
	\boxed{\includegraphics[scale=0.8]{pics/flow_doe}}
	\caption{Aktivitätsdiagramm zur Durchführung der statistischen Versuchsplanungen}
	\label{fig:FlowVersuchsplanung}
\end{figure}

\subsection{Datenbereitstellung in Excel}
\label{subsec:FlowDataToExcel}

Da nun alle Daten erfasst und aufbereitet wurden, erfolgt nun im letzten Schritt die Übergabe nach Excel, damit sie unter Nutzung der in diesen Programm bereitgestellten Funktionen in allen für die Herleitung eines Gesamtergebnisses relevanten Dimensionen ausgewertet werden können. Wie im Komponentendiagramm (Abbildung \ref{fig:Komponentendiagramm} auf Seite \pageref{fig:Komponentendiagramm}) bereits zu erkennen war, muss dieser Schritt nicht zwingend an letzter Stelle durchgeführt werden, sondern ist auch schon nach Extraktion der Messergebnisse (Abschnitt \ref{subsec:FlowExtraktionMessergebnisse}) möglich. Der funktionale Ablauf ist auch hier wieder in Form eines Aktivitätsdiagramms in nachstehender Abbildung \ref{fig:FlowDatenbereitstellung} dargestellt.

\begin{figure}[H]
	\centering
	\boxed{\includegraphics[scale=0.8]{pics/flow_data_to_excel}}
	\caption{Aktivitätsdiagramm zur Bereitstellung der Daten als Excel-Datei}
	\label{fig:FlowDatenbereitstellung}
\end{figure}

\section{Klassendiagramme}
\label{sec:Klassendiagramme}

In diesem Abschnitt soll nun beschrieben werden, wie bestimmte Aspekte der Gesamtaufgabe durch die Objektorientierung in der Programmierung sehr effektiv gelöst werden konnten. Das bezieht sich sowohl auf die gesamte Programm- und Datenflussteuerung als auch auf die Persistierung der Daten, ohne dafür eine Datenbank im herkömmlichen Sinne aufzusetzen. Die Datenbank ist somit eine Objektstruktur, welche am Anfang der Prozesskette erstellt (vgl dazu auch Abbildung \ref{fig:Komponentendiagramm}) und vor Abschluss eines jeden Prozessschrittes serialisiert und gespeichert wird (\textbf{Persistierung der Daten}). Bei der Durchführung jedes darauf folgenden Arbeitsschrittes  wird diese Objektstruktur wieder geladen (de-serialisiert) und die Daten werden innerhalb des Programmablaufes zum einen genutzt und zum anderen um neu generierte Daten ergänzt.
\\\\
\noindent Diese Objektstruktur umfasst 2 Klassen, die Prozesssteuerrungs- und die Bild-Klasse, welche im nachfolgenden dargestellt und beschrieben werden. Folgende Abbildung \ref{fig:KlassendiagrammUeberblick}  zeigt zunächst das Übersichtsdiagramm beider Klassen zum Überblick. An der Kardinalitäten sowie der eingezeichneten Beziehung wird bereits die Abhängigkeit der Bild-Klasse von der Prozessteuerungs-Klasse erkennbar. Die hier dargestellten Kardinalitäten bedeuten, dass jede Bild-Instanz zu genau einer Prozessteuerungs-Instanz gehört, jeder Prozessteuerungs-Instanz aber n-viele Bild-Instanzen beinhalten kann.

\begin{figure}[H]
	\centering
	\includegraphics[scale=0.8]{pics/class_diagram_overview}
	\caption{Darstellung der Abhängigkeiten zwischen der Prozesssteuerungs- und der Bild-Klasse}
	\label{fig:KlassendiagrammUeberblick}
\end{figure}

\noindent Im folgenden wird nun auf die Darstellung in der vorstehenden Abbildung nochmal konkret eingegangen und jede einzelne Klasse noch im Detail modelliert. Die Einzeldarstellungen und Beschreibungen i.V.m Abbildung \ref{fig:Komponentendiagramm} auf Seite \pageref{fig:Komponentendiagramm} bilden dann das Gesamtbild zum funktionalen Verständnis der Klassenstruktur in der vorliegenden Arbeit.


\subsection{Die Prozesssteuerungs-Klasse}

Die Idee dahinter ist, eine übergeordnete Datenstruktur bereitzustellen, die alle Informationen enthält, die ohnehin bei der Ausführung aller beschriebenen Arbeitsschritte mehr oder weniger immer benötigt werden. Das steigert die Benutzbarkeit enorm und ist auch weniger fehleranfällig. Außerdem werden darin alle Funktionalitäten gekapselt, die erforderlich sind, um die entsprechenden Aufgaben auszuführen, wie bspw. die Durchführung einer Lamellengraphit-Messung oder einer Versuchsplanung, wie in Abbildung \ref{fig:KlasseProzessteuerung} dargestellt.

\begin{figure}[H]
	\centering
	\includegraphics[width=\textwidth, height=\textheight, keepaspectratio]{pics/class_runner}
	\caption{Darstellung der Prozessteuerungs-Klasse}
	\label{fig:KlasseProzessteuerung}
\end{figure}

\noindent Da bereits bei der Instanziierung der Klasse unter anderem alle Pfade der Probenbilder übergeben werden und die Bildobjekte automatisch mit erzeugt werden, muss man danach nichts weiter tun, als in einem Skript die gewünschten Funktionen direkt auf das Objekt anzuwenden. Bei Betrachtung der Bildklasse im nächsten Abschnitt wird erkennbar, wie die Funktionalität dadurch in eine hierarchisch-logische Struktur gebracht wurde, die dadurch leicht nachvollziehbar und verstehbar wird. Das erleichtert die Anwendung, Wartbarkeit und Erweiterbarkeit.

\subsection{Die Bild-Klasse}

Die Aufgabe der Bild-Klasse, als Teil der Prozesssteuerungs-Klasse, ist es, sowohl alle bild-bezogenen Operativdaten (wie bspw. die absoluten Pfade zu den 56 für jedes Bild erzeugten skalierten Kopien) sowie auch alle Messergebnisse zu speichern und vorzuhalten. Diese Struktur ist eindeutig und widerspruchsfrei in dem Sinne, dass sie den Zugriff und die Weiterverarbeitung der ziemlich großen Menge an erzeugten Daten in einer Weise ermöglicht, die einfach und nachvollziehbar ist, da alle bild-spezifischen Informationen an einer Stelle vorliegen. Durch einen Blick auf den Aufbau der Klasse in Abbildung \ref{fig:KlasseBild} wird vieles von dem gesagten offensichtlich.

\begin{figure}[H]
	\centering
	\includegraphics[width=\textwidth, height=\textheight, keepaspectratio]{pics/class_image}
	\caption{Darstellung der Bild-Klasse}
	\label{fig:KlasseBild}
\end{figure}

\noindent Wie man leicht erkennen kann, sind alle Daten und Information, die zu einem Bild gehören auch, im jeweiligen Bild-Objekt gespeichert und abrufbar. Das erleichtert die anschließende Datenaufbereitung sehr.


\chapter{Evaluation}
\label{ch:Evaluation}

\section{Auswertung der Analysedaten}
\label{sec:AuswertungAnalysedaten}

\subsection{Auswertung der statistischen Versuchsplanung}
\label{subsubsec:AuswertungStatVpl}
Wie in Kapitel \ref{sec:StatVersPlanung} ab Seite \pageref{sec:StatVersPlanung} bereits beschrieben, dient die statistische Versuchsplanung dazu, sichtbar zu machen, welchen Einfluss die unterschiedlichen Faktoren (Steuer- und Störgrößen) auf den Zielwert haben. Das Ergebnis zeigt die nachstehende Abbildung \ref{fig:DAPareto} sowie Tabelle \ref{tab:DAPareto}, in der die für das Diagramm berechneten Werte tabellarisch dargestellt werden. 

\begin{figure}[H]
	\centering
	\includegraphics[width=14cm, height=\textheight, keepaspectratio]{pics/DA_Pareto}
	\caption{Pareto-Diagramm der standardisierten Effekte}
	\label{fig:DAPareto}
\end{figure}

\begin{table}[H]
	\caption{Berechnete Mittelwerte (über alle Bilder) zum Pareto-Diagramm der standardisierten Effekte}
	\label{tab:DAPareto}
	\centering
	\includegraphics[width=\textwidth,height=\textheight,keepaspectratio]{pics/Tab_DA_Pareto}
\end{table}

\noindent Das Diagramm zeigt die Absolutwerte der standardisierten Effekte, geordnet vom größten zum kleinsten. Ein wichtiger Indikator ist die dargestellte Referenzlinie. Nach \cite{minitab_pareto} ist die Methode, die Minitab zum Zeichnen des Pareto-Diagramms der standardisierten Effekte verwendet, abhängig von den Freiheitsgraden für den Fehlerterm. Wenn dieser einen oder mehrere Freiheitsgrade hat, wird die gelbe Linie im Pareto-Diagramm bei t gezeichnet, wobei $t$ das $1-\nicefrac{\alpha}{2}$ - Quantil einer t-Verteilung mit Freiheitsgraden gleich den Graden von Freiheit für den Fehlerterm ist. Minitab bezeichnet dieses Diagramm als Pareto-Diagramm der standardisierten Effekte. Wenn der Fehlerterm null Freiheitsgrade hat, identifiziert Minitab wichtige Effekte mithilfe des Lenth-Pseudostandardfehlers (PSE). Die gelbe Linie des Pareto-Diagramms wird dann an der Fehlergrenze gezeichnet, die wie folgt lautet:

\begin{equation}
	ME = t\cdot PSE
\end{equation}

\noindent Faktoren, die diese Linie überschreiten, gelten als statistisch signifikant. 
\\\\
Wie in Kapitel \ref{ch:Umsetzung} bereits beschrieben wurden Versuchsplanungen für alle 232 Probebilder separat durchgeführt. Um die im Diagramm dargestellten Werte zu berechnen wurde die Werte aller Einzel-Pareto-Diagramme aufsummiert, durch die Anzahl der Bilder geteilt, um auf dieser Grundlage ein Pareto-Diagramm der standardisierten Effekte zu erstellen, welches einen Querschnitt über  alle Probebilder liefert.
\\\\
\noindent Vor diesem Hintergrund kann also geschlossen werden, dass die Einflussfaktoren \textbf{Interpolationsmodus} und \textbf{Kalibrierung} sowie die Wechselwirkung beider Faktoren insgesamt betrachtet den größten Einfluss auf die Zielgröße haben.
\\\\
\noindent Wenngleich das Pareto-Diagramm dadurch wichtige Hinweise zum Verständnis der Daten liefert und die Haupteffekte sichtbar macht, kann damit nicht festgestellt werden, durch welche Effekte den Wert der Antwortvariablen vergrößert oder verkleinert wird. Um diese Fragestellung zu beantworten, wurden weitere Analysen durchgeführt, die im folgenden beschrieben und ausgewertet werden.


\subsection{Analyse der eingesetzten Interpolationsverfahren}

Wie im vorigen Abschnitt schon angedeutet, geht es in diesem vor allem darum herauszufinden, wie genau sich die bereits identifizierten Haupteffekte auf die durch Skalierung der Bilder induzierten Messfehler auswirken. Da sich der gesamte Messfehler aus unterschiedlichen Messfehlerkategorien zusammensetzt, wird dieser nun nach einer vorausgehenden Gesamtbetrachtung in seine einzelnen Komponenten zerlegt und mit Methoden der deskriptiven Statistik näher untersucht. Ausgehend von der Gesamtbetrachtung ist es so möglich Schritt für Schritt ein tieferes Verständnis über die Wirkungsweise der unterschiedlichen Einflussgrößen (insbesondere der Interpolationsalgorithmen) auf die Zielgröße zu gewinnen.

\subsubsection{Gesamtfehlerbetrachtung}
\label{subsubsec:AuswertungGesamtfehlerbetrachtung}
 In Abschnitt \ref{fig:DAPareto} wurde der \textbf{Interpolationsmodus} als Haupteffekt mit dem größten Einfluss auf die Zielvariable identifiziert. Wie jedoch die folgende Abbildung \ref{fig:DAGesamtAbsolut_Alle} in Verbindung mit Tabelle \ref{tab:DAGesamtAbsolutAlle} zeigt, hat diese Einflussgröße zwar ohne Zweifel den größten Effekt, jedoch nicht in die gewünschte Richtung, da es ja darum geht, den Messfehler zu minimieren. 

\begin{figure}[H]
	\centering
	\includegraphics[width=\textwidth, height=\textheight, keepaspectratio]{pics/DA_Gesamt_Absolut_Alle}
	\caption{Auswertung des Messfehlers (Gesamt)}
	\label{fig:DAGesamtAbsolut_Alle}
\end{figure}
	
\begin{table}[H]
	\centering
	\caption{Daten zur Auswertung des Messfehlers (Gesamt)}
	\label{tab:DAGesamtAbsolutAlle}
	\includegraphics[width=\textwidth,height=\textheight,keepaspectratio]{pics/Tab_DA_Gesamt_Absolut_Alle}
\end{table}

\noindent An den berechneten Mittelwerten in der Tabelle ist zu erkennen, dass LANCZOS4 über alle Kategorien den geringsten durchschnittlichen Fehler erzeugt. Allerdings zeigen die Daten auch, dass dies nicht repräsentativ für alle Kalibrierungsstufen ist. Man betrachte z.B. Kalibrierungsstufe 0,7 bei Vollskalierung - in diesem Bereich liegt der Fehlerwert von Nächster-Nachbar mit ca. 88,41 etwa 4 Einheiten unter dem von LANCZOS4 mit ca. 92,22. Darüber hinaus ist zu beobachten, dass der Fehler mit zunehmender Ausgangskalibrierung wächst. Vergleicht man etwa am Beispiel von LANCZOS4 die Werte bei Kalibrierung  0,3  und 1,9 $\nicefrac{\mu m}{\text{Pixel}}$, stellt man fest, dass der Messfehlerwert bei 1,9 $\nicefrac{\mu m}{\text{Pixel}}$ um fast 57 $\%$ über dem bei Kalibrierung 0,3 liegt. Dadurch zeigt sich, dass die Vergrößerung von Bildern (also Verringerung der Kalibrierung) wohl offensichtlich weniger gut funktioniert, als in die entgegengesetzte Richtung.
\\\\
Betrachtet man dagegen die Streuungsparameter Standardabweichung und Varianz, ergibt sich ein etwas anderes Bild. Während die Standardabweichung über alle Bereiche hinweg sehr niedrig ausfällt und kaum schwankt, ist die Varianz, also die Streuung in den Daten wesentlich höher und schwankt im Bereich von ca. 135,62 bei Kalibrierung 0,3 bis ca. 367,18 bei einer Ausgangskalibrierung von 1,9 $\nicefrac{\mu m}{\text{Pixel}}$ (hier am Beispiel von LANCZOS4) ziemlich stark. Ebenfalls ist auffällig, dass die kubische Interpolation in den Kalibrierungsbereichen 0,3 und 1,1 $\nicefrac{\mu m}{\text{Pixel}}$ die niedrigste Varianz, also auch im Vergleich zu LANCZOS4 aufweist.  Die nachstehende Abbildung \ref{fig:DAGesamtStreu1Alle} sowie die dazugehörige Datentabelle \ref{tab:DAGesamtStreu1Alle} geben diesen Zusammenhang sehr gut zu erkennen.

\begin{figure}[H]
	\centering
	\includegraphics[width=\textwidth, height=\textheight, keepaspectratio]{pics/DA_Gesamt_Streu1_Alle}
	\caption{Streuungsparameter (Standardabweichung u. Varianz) bezogen auf den gesamten Messfehler}
	\label{fig:DAGesamtStreu1Alle}
\end{figure}
	
\begin{table}[H]
	\centering
	\caption{Daten zu den berechneten Streuungsparametern (Standardabweichung u. Varianz) bezogen auf den gesamten Messfehler}
	\label{tab:DAGesamtStreu1Alle}
	\includegraphics[width=\textwidth,height=\textheight,keepaspectratio]{pics/Tab_DA_Gesamt_Streu1_Alle}
\end{table}

\noindent Hierbei wird deutlich, dass LANCZOS4, wenn auch den geringsten gesamten Messfehler über alle Bereiche, andererseits in einigen Bereichen eine sehr hohe Varianz besitzt (grüne Datenpunkte), was auf eine hohe Streuung hindeutet. Demgegenüber hat die kubische Interpolation über alle Kategorien die geringste Varianz (rote Datenpunkte). Die Frage, welcher Kandidat die höchste Varianz aufweist, kann nicht eindeutig beantwortet werden, da diese sich kategorial unterscheidet. Die entsprechenden Datenpunkte und Werte sind in der Abbildung in der dem entsprechenden Algorithmus jeweils zugeordneten Farbe ebenfalls eingetragen, jedoch nur für die Vollskalierung, um die Grafik nicht zu sehr zu überlagern. 
\\\\
Ein weiteres Streuungsmaß, das sich zur Beschreibung von Daten sehr gut eignet, ist die Variationsbreite, die zeigt, wie weit die Werte genau streuen. Abbildung \ref{fig:DAGesamtStreu2Alle} i. V. m. Tabelle \ref{tab:DAGesamtStreu2Alle} zeigt diesen Sachverhalt und beschreibt in absoluten Zahlen, wie weit die Messfehlerwerte je Interpolationsalgorithmus, Interpolationsmodi und Stufe der Ausgangskalibrierung streuen. 

\begin{figure}[H]
	\centering
	\includegraphics[width=11cm, height=\textheight, keepaspectratio]{pics/DA_Gesamt_Streu2_Alle}
	\caption{Streuungsparameter (Variationsbreite) bezogen auf den gesamten Messfehler}
	\label{fig:DAGesamtStreu2Alle}
\end{figure}

\begin{table}[H]
	\centering
	\caption{Daten zum berechneten Streuungsparameter (Variationsbreite) bezogen auf den gesamten Messfehler}
	\label{tab:DAGesamtStreu2Alle}
	\includegraphics[width=\textwidth, height=\textheight, keepaspectratio]{pics/Tab_DA_Gesamt_Streu2_Alle}
\end{table}


\noindent Wie zu erwarten ist auch die Variationsbreite bei LANCZOS4 am geringsten (bezogen auf den Mittelwert), jedoch nicht in allen Kalibrierungsstufen. Betrachtet man bspw. die Kalibrierungsstufe $0,3~\nicefrac{\mu m}{\text{Pixel}}$ ist zu erkennen, das LANCZOS4 im Vergleich zur flächenbasierten Interpolation eine minimal höhere Variationsbreite der Messfehlerwerte aufweist. 
\\\\
Die Ergebnisse aus diesem Abschnitt lassen sich wie folgt kurz zusammenfassen:
\\\\
\colorbox{gray!10}{
	\label{box:ErgebnisGesamtauswertung}
	\begin{minipage}{0.975\textwidth}
		\textbf{\underline{Ergebnis der Gesamtfehlerbetrachtung:}}\\\\
		Zusammenfassend kann für diesen Abschnitt festgehalten werden, dass die LANCZOS4-Interpolation zwar im Durchschnitt über alle Bereiche den kleinsten Fehler erzeugt, jedoch allerdings in einigen Bereichen auch eine höhere Streuung aufweist, als die kubische Interpolation, welche zwar die geringste durchschnittliche Varianz besitzt, aber nicht immer den geringsten Fehler erzeugt.
		\\\\
		\noindent Des Weiteren kann festgehalten werden, dass die schrittweise Skalierung tendenziell immer größere Fehler erzeugt, als Voll-Skalierung - sehr wahrscheinlich deshalb, weil sich die Fehler bei der mehrfach-Skalierung addieren. 
		\\\\
		\noindent Eine weitere wichtige Beobachtung ist die, dass bei der Transformation von Bildern die Verringerung des Kalibrierungsfaktors offensichtlich geringere Datenverluste entstehen lässt, wohingegen der Fehler bei zunehmender Ausgangskalibrierung stetig wächst, da ein eindeutiger Trend in diese Richtung zu erkennen ist.
	\end{minipage}
}
\\\\\\
Da die Gesamtbetrachtung noch keine eindeutige Interpretation zulässt, wird nun im weiteren Verlauf der Analyse der Gesamt-Messfehler in seine Bestandteile zerlegt und zunächst innerhalb der Größenklassen (anzahlgewichtet) genauer untersucht.

\subsubsection{Auswertung nach Größenklassen (anzahlgewichtet)}
\label{subsubsec:AuswertungGKanz}
Die Größenklassen (anzahlgewichtet) entsprechen den Histogramm-Werten, welche durch die Software AMGuss bei der Durchführung einer Lamellengraphitauswertung als Teil des Analyseergebnisses berechnet werden. Durch die getrennte Erfassung der Abweichungen in den Originalbildern ist es nun möglich, diese Fehlerkategorie separat zu untersuchen. Wie im vorherigen Abschnitt bereits geschehen, erfolgt auch hier zunächst einer Gesamtbetrachtung über die Gesamte Kategorie (bestehend aus 8 Klassen). Zusätzlich wird die Analyse noch durch eine Einzel-Fehleranalyse bezogen auf die 8 Größenklassen erweitert. Abbildung \ref{fig:DAGesamtAbsolutGKanzahl} und die damit verbundene Datenbasis in Tabelle \ref{tab:DAGesamtAbsolutGKanzahl} vergleichen wieder zunächst die verschiedenen Interpolationsalgorithmen je Interpolationsmodi und Kalibrierungsstufe.

\begin{figure}[H]
	\centering
	\includegraphics[width=\textwidth, height=\textheight, keepaspectratio]{pics/DA_Gesamt_Absolut_GKanzahl}
	\caption{Auswertung des Messfehlers (Gesamt) nach Größenklassen (anzahlgewichtet)}
	\label{fig:DAGesamtAbsolutGKanzahl}
\end{figure}

\begin{table}[H]
	\centering
	\caption{Daten zu den berechneten Messfehlern nach Größenklassen (anzahlgewichtet)}
	\label{tab:DAGesamtAbsolutGKanzahl}
	\includegraphics[width=\textwidth, height=\textheight, keepaspectratio]{pics/Tab_DA_Gesamt_Absolut_GKanzahl}
\end{table}


\noindent Was sofort auffällt ist, dass die Kurven sich sehr denen aus der Gesamt-Fehlerbetrachtung im letzten Abschnitt ähneln. Auch hier ist eindeutig der Trend zu erkennen, dass der Fehler sich mit steigender Ausgangskalibrierung vergrößert und dass die Fehler bei der schrittweisen Skalierung immer größer ausfallen, als bei Voll-Skalierung. 
\\\\
Wenn auch knapp, weist die LANCZOS4-Interpolation hier mit einem Durchschnittswert von ca. 26,12 den geringsten durchschnittlichen Fehler auf. Allerdings erzeugt dieses Interpolationsverfahren, wie im Diagramm zu erkennen, nur im Bereich von 1,1 $\nicefrac{\mu m}{\text{Pixel}}$ bei schrittweiser Skalierung auch tatsächlich den geringsten absoluten Messfehler, der allerdings höher ist als bei vergleichbarer Voll-Skalierung und daher vernachlässigbar. Die jeweils kleinsten Werte sind in der dem jeweiligen Algorithmus zugeordneten Diagrammfarbe entsprechend gekennzeichnet. Man kann daran gut erkennen, dass die Frage danach, welcher Algorithmus denn in dieser Größenklasse den kleinsten Fehler erzeugt, noch nicht hinreichend beantwortet werden kann.
\\\\
Anhand der Streuungsparameter erkennt man, dass die LANCZOS4-Interpolation, wenn auch den geringsten absoluten Fehler, in den meisten Fällen nicht auch die geringste Varianz besitzt wie man in der Abbildungen \ref{fig:DAGesamtStreu1GKanzahl} und \ref{fig:DAGesamtStreu2GKanzahl} i.V.m  den dazugehörigen Datentabellen \ref{tab:DAGesamtStreu1GKanzahl} und \ref{tab:DAGesamtStreu2GKanzahl} eindeutig ablesen kann.

\begin{figure}[H]
	\centering
	\includegraphics[width=\textwidth, height=\textheight, keepaspectratio]{pics/DA_Gesamt_Streu1_GKanzahl}
	\caption{Streuungsparameter (Standardabweichung und Varianz) nach Größenklassen (anzahlgewichtet)}
	\label{fig:DAGesamtStreu1GKanzahl}
\end{figure}

\begin{table}[H]
	\centering
	\caption{Daten zu den berechneten Streuungsparametern (Standardabweichung und Varianz) nach Größenklassen (anzahlgewichtet)}
	\label{tab:DAGesamtStreu1GKanzahl}
	\includegraphics[width=\textwidth, height=\textheight, keepaspectratio]{pics/Tab_DA_Gesamt_Streu1_GKanzahl}
\end{table} 

\begin{figure}[H]
	\centering
	\includegraphics[width=\textwidth, height=\textheight, keepaspectratio]{pics/DA_Gesamt_Streu2_GKanzahl}
	\caption{Streuungsparameter (Variationsbreite) nach Größenklassen (anzahlgewichtet)}
	\label{fig:DAGesamtStreu2GKanzahl}
\end{figure}

\begin{table}[H]
	\centering
	\caption{Daten zum berechneten Streuungsparameter (Variationsbreite) nach Größenklassen (anzahlgewichtet)}
	\label{tab:DAGesamtStreu2GKanzahl}
	\includegraphics[width=\textwidth, height=\textheight, keepaspectratio]{pics/Tab_DA_Gesamt_Streu2_GKanzahl}
\end{table}

\noindent Das bedeutet, dass die Fehlerstreuung über alle Bilder in den meisten Kalibrierungsstufen für andere Algorithmen geringer ausfällt. Im direkten Vergleich mit der Gesamtauswertung (vgl. Abbildung \ref{fig:DAGesamtStreu1Alle}) sind hier zwar wesentlich geringere Varianzen im Bereich von $0,3~-~1,1\nicefrac{\mu m}{\text{Pixel}}$ (bei Voll-Skalierung) die, unabhängig vom Interpolationsalgorithmus, schon fast vernachlässigbar erscheinen. In der Kalibrierungsstufe $1,9\nicefrac{\mu m}{\text{Pixel}}$ ist hingegen wieder eine sehr hohe Streuung gegeben.
\\\\
\noindent Wie zu Beginn des Abschnitts bereits angekündigt, erfolgt zur vollständigen Betrachtung dieser Größenklassenkategorie nun noch eine Einzelbetrachtung, die zeigt, wie sich der Messfehler in den einzelnen Klassen auswirkt (Abbildung \ref{fig:DAEinzelAbsolutGKanzahl}).

\begin{figure}[H]
	\centering
	\includegraphics[width=\textwidth, height=\textheight, keepaspectratio]{pics/DA_Einzel_Absolut_GKanzahl}
	\caption{Auswertung des Messfehlers je Größenklasse im Bereich Größenklassen (anzahlgewichtet)}
	\label{fig:DAEinzelAbsolutGKanzahl}
\end{figure}

\noindent Auch wenn sich die Interpolationsalgorithmen sowohl in der Datenbasis als auch visuell im Diagramm kaum unterscheiden lassen, können hier wichtige Zusammenhänge abgeleitet werden. Die wichtigsten Unterscheidungsmerkmale sind außerdem farblich gekennzeichnet. Im Durchschnitt erzeugt LANCZOS4 über alle Kategorien den kleinsten Messfehler, wobei sich ein etwas anderes Bild ergibt, wenn man nach Größenklassen differenziert.
\\\\
Beispielsweise fällt beim Vergleich von LANCZOS und flächenbasierter Interpolation auf, dass der Messfehler über alle Kalibrierungsstufen in den Größenklassen 1-3 nahezu gegen Null geht, wohingegen in den Kategorien 4-8 ein starker Anstieg beobachtbar ist.Die LANCZOS-Interpolation weist etwa in der Größenklasse 8 mit 19,52 einen mehr als 10 mal höheren Fehlerwert auf, als in der Größenklasse 5 mit ca. 1,88. Bei der flächenbasierten Interpolation  ist der Unterschied dagegen, mit Werten von ca. 3,71 in Größenklasse 5 sowie 23,62 in Klasse 8, etwas geringer, denn hier beträgt das Verhältnis mit ca. 6,4 nur etwas mehr als die Hälfte. 
\\\\
Da dieses Beispiel mit LANCZOS und flächenbasierter Interpolation nicht zufällig gewählt wurde, sondern auf der Grundlage einer vorhergehenden Datenanalyse, kann der hier dargestellte Zusammenhang auf alle Kalibrierungsstufen übertragen werden. Auch an der Wiederholung des Musters in der Kurve lässt sich dies sehr gut erkennen. 
\\\\
\colorbox{gray!10}{
	\label{box:ErgebnisGesamtauswertung}
	\begin{minipage}{0.975\textwidth}
		\textbf{\underline{Ergebnis der Auswertung nach Größenklassen (anzahlgewichtet):}}
		\\\\
		Zusammenfassend kann für diesen Abschnitt festgehalten werden, dass die LANCZOS4-Interpolation mit einem Wert von 26,12 zwar über alle Bereiche zwar den kleinsten durchschnittlichen Fehler erzeugt, jedoch nur im Kalibrierungsbereich von 1,1 $\nicefrac{\mu m}{\text{Pixel}}$ bei schrittweise Skalierung auch tatsächlich den kleinsten absoluten Messfehler. Die Streuungsparameter zeigen jedoch das die LANCZOS-Interpolation nicht immer die niedrigste Varianz in den Daten aufweist, sondern nur im Bereich von 0,7 $\nicefrac{\mu m}{\text{Pixel}}$.
		\\\\
		Auch bezogen auf die Größenklassen bestätigt sich die bereits im Abschnitt \ref{subsubsec:AuswertungGesamtfehlerbetrachtung} Beobachtung, dass durch Voll-Skalierung wesentlich geringere Fehler erzeugt werden, als bei vergleichbarer schrittweisen Skalierung.
		Ein signifikanter Anstieg ist ab einer Kalibrierungsstufe von 1,1 $\nicefrac{\mu m}{\text{Pixel}}$ erkennbar. Jedoch gilt dies nur für die Größenklassen 4-8, wobei die Fehler in den Klassen 1-3 über alle Kalibrierungsstufen und Interpolationsalgorithmen vergleichsweise vernachlässigbar gering ausfallen.
	\end{minipage}
}
\\\\\\

\subsubsection{Auswertung nach Größenklassen (flächengewichtet)}
\label{subsubsec:AuswertungGKflaeche}
Wie schon bei den Größenklassen im vorherigen Abschnitt, teilen sich diese wiederum auf in unterschiedliche Kategorien 1-8. Die Auswertung in diesem Abschnitt erfolgt mit denselben statistischen Methoden, um eine größtmögliche Vergleichbarkeit zu gewährleisten.
\\\\
\noindent Abbildung \ref{fig:DAGesamtAbsolutGKflaeche} sowie die Datentabelle \ref{tab:DAGesamtAbsolutGKflaeche} zeigen zunächst wieder das Gesamtbild der Datenverteilung in den unterschiedlichen Kalibrierungsstufen.

\begin{figure}[H]
	\centering
	\includegraphics[width=\textwidth, height=\textheight, keepaspectratio]{pics/DA_Gesamt_Absolut_GKfläche}
	\caption{Auswertung des Messfehlers (Gesamt) nach Größenklassen (flächengewichtet)}
	\label{fig:DAGesamtAbsolutGKflaeche}
\end{figure}

\begin{table}[H]
	\caption{Messdaten zur Auswertung des Messfehlers (Gesamt) nach Größenklassen (flächengewichtet)}
	\label{tab:DAGesamtAbsolutGKflaeche}
	\begin{figure}[H]
		\centering
		\includegraphics[width=\textwidth, height=\textheight, keepaspectratio]{pics/Tab_DA_Gesamt_Absolut_GKfläche}
	\end{figure}
\end{table}

\noindent Im Vergleich mit den Größenklassen (anzahlgewichtet) ist hier ein ganz ähnlicher Verlauf zu erkennen, allerdings ist es nun die kubische Interpolation (rote Datenpunkte), die mit einem gemittelten Wert von ca. 28,73 über alle Bereiche den geringsten Fehler erzeugt. Jedoch ergibt sich auch wieder ein etwas anderes Bild, wenn man untersucht, für welche Kategorien das letztendlich auch wirklich zutrifft. Die Abbildung zeigt daher jeweils auch die minimalen Werte an, außer in den Fällen, in denen die kubische Interpolation tatsächlich auch den geringsten Fehler aufweist. 
\\\\
\noindent Die Streuungsparameter Standardabweichung und Varianz zeigen auf den ersten Blick nahezu exakt den selben Verlauf wie in Abschnitt \ref{subsubsec:AuswertungGKanz}, allerdings sind auch unterschiedliche Algorithmen für die jeweils größten und kleinsten Fehler verantwortlich (Abbildung \ref{fig:DAGesamtStreu1GKfläche} i.V.m. Datentabelle \ref{tab:DAGesamtStreu1GKflaeche} sowie Abbildung \ref{fig:DAGesamtStreu2GKfläche} und Datentabelle \ref{tab:DAGesamtStreu2GKflaeche}).

\begin{figure}[H]
	\centering
	\includegraphics[width=\textwidth, height=\textheight, keepaspectratio]{pics/DA_Gesamt_Streu1_GKfläche}
	\caption{Streuungsparameter (Standardabweichung u. Varianz) nach Größenklassen (flächengewichtet)}
	\label{fig:DAGesamtStreu1GKfläche}
\end{figure}

\begin{table}[H]
	\caption{Daten zu den berechneten Streuungsparametern (Standardabweichung u. Varianz) nach Größenklassen (flächengewichtet)}
	\label{tab:DAGesamtStreu1GKflaeche}
	\begin{figure}[H]
		\centering
		\includegraphics[width=\textwidth, height=\textheight, keepaspectratio]{pics/Tab_DA_Gesamt_Streu1_GKfläche}
	\end{figure}
\end{table}

\begin{figure}[H]
	\centering
	\includegraphics[width=\textwidth, height=\textheight, keepaspectratio]{pics/DA_Gesamt_Streu2_GKfläche}
	\caption{Streuungsparameter (Variationsbreite) nach Größenklassen (flächengewichtet)}
	\label{fig:DAGesamtStreu2GKfläche}
\end{figure}

\begin{table}[H]
	\caption{Daten zum berechneten Streuungsparameter (Variationsbreite) nach Größenklassen (flächengewichtet)}
	\label{tab:DAGesamtStreu2GKflaeche}
	\begin{figure}[H]
		\centering
		\includegraphics[width=\textwidth, height=\textheight, keepaspectratio]{pics/Tab_DA_Gesamt_Streu2_GKfläche}
	\end{figure}
\end{table}

\noindent Während die Standardabweichung mit einem Wert von im Mittel ca. 4,35 und einer Variationsbreite von ca. 0,42 sehr gering ist und kaum schwankt, zeigen die Varianzen wieder, dass die kubische Interpolation in fast allen Bereichen nur mittelmäßig abschneidet. Das ist an den eingezeichneten minimalen und maximalen Datenpunkten zu erkennen, denn nur im Bereich von 1,1 $\nicefrac{\mu m}{\text{Pixel}}$ (schrittweise Skalierung) weist die kubische Interpolation auch tatsächlich die geringste Varianz auf.
\\\\
\noindent Zur Vervollständigung der Analyse in diesem Bereich folgt nun wieder eine Einzelbetrachtung der Größenklassen (flächengewichtet). In der nachstehenden Abbildung \ref{fig:DAEinzelAbsolutGKflaeche} ist zu erkennen, dass wiederum in den Klassen 4-8 ein signifikanter Anstieg der Fehlerwerte vorliegt.

\begin{figure}[H]
	\centering
	\includegraphics[width=\textwidth, height=\textheight, keepaspectratio]{pics/DA_Einzel_Absolut_GKfläche}
	\caption{Auswertung des Messfehlers (Einzel) nach Größenklassen (flächengewichtet)}
	\label{fig:DAEinzelAbsolutGKflaeche}
\end{figure}

\noindent Für den im Diagramm eingezeichneten Vergleich (rote Datenpunkte) wurde die kubische Interpolation gewählt, weil bei diesem Interpolationsverfahren über alle Größenklassen der geringste durchschnittliche Messfehler auftritt. Dabei ist festzustellen, dass bspw. wie hier in Kalibrierungsstufe 1,9 $\nicefrac{\mu m}{\text{Pixel}}$ (Vollskalierung), der Fehlerwert in Größenklasse 8 mit ca. 10,87 nur etwa 3,4 mal höher ist, als in Klasse 5 mit ca. 3,15. Außerdem ist sowohl bei Voll- als auch bei der schrittweisen Skalierung jeweils ein Knick in der im Bereich der Größenklasse 7 auffällig, bei dem die Fehlerwerte im Vergleich mit Klasse 6 zunächst absinken und dann wieder sehr stark ansteigen. 
\\\\
\colorbox{gray!10}{
	\label{box:Ergebnis der Auswertung nach Größenklassen (flächengewichtet):}
	\begin{minipage}{0.975\textwidth}
		\textbf{\underline{Ergebnis der Auswertung nach Größenklassen (flächengewichtet):}}
		\\\\
		Es wurde in diesem Teil der Analyse festgestellt, dass die kubische Interpolation mit einem  Fehler von 28,72 einen über alle Bereiche geringsten durchschnittlichen Wert aufweist, aber nur im bei Kalibrierung von 0,7 $\nicefrac{\mu m}{\text{Pixle}}$ bei Voll-Skalierung auch den tatsächlich kleinsten absoluten Wert. Die Streuungsparameter zeigen bei einer konstant sehr niedrigen Standardabweichung, dass bei der die kubischen Interpolation nur im Bereich 1,1 $\nicefrac{\mu m}{\text{Pixel}}$ auch die kleinste Varianz in den Daten gegeben ist.
		\\\\
		Wie auch an früherer Stelle schon festgestellt, gilt auch hier wieder, dass die Verringerung des Kalibrierungsfaktors geringere Fehler erzeugt. Bedeutende Anstiege des Fehlers sind auch hier ab Kalibrierungsstufe 1,1 $\nicefrac{\mu m}{\text{Pixel}}$ gegeben, was jedoch wieder nur für die Größenklassen 4-8 zutreffend ist.
	\end{minipage}
}
	
\subsubsection{Auswertung nach Anordnungsklassen}
\label{subsubsec:AuswertungAKL}
Die Bestimmung der Anordnungsklassen ist ebenfalls ein von der Software AMGuss bei Durchführung einer Lamellengraphitauswertung zurückgegebenes Ergebnis, bei dem die Häufigkeiten der gefundenen Anordnungsklassen (vgl. auch Abbildung \ref{fig:Anordnungsklassen} auf Seite \pageref{fig:Anordnungsklassen}) auch in einem Histogramm dargestellt werden.  Diese Kategorie soll nun abschließend ebenfalls eingehend untersucht werden und im Anschluss daran alle Teilergebnisse zu einer Gesamtauswertung zusammengefasst werden.
\\\\
\noindent Die nachstehende Abbildung \ref{fig:DAGesamtAbsolutAKL} i.V.m. Datentabelle \ref{tab:DAGesamtAbsolutAKL} zeigen an, dass die LANCZOS4-Interpolation mit einem durchschnittlichen absoluten Fehlerwert von ca. 28,73 über alle Bilder den geringster Fehler verursacht. Doch bei genauerer Betrachtung stellt sich heraus, dass dies in keinem Bereich dem tatsächlich geringsten absoluten Fehler entspricht. 

\begin{figure}[H]
	\centering
	\includegraphics[width=\textwidth, height=\textheight, keepaspectratio]{pics/DA_Gesamt_Absolut_AKL}
	\caption{Auswertung des Messfehler nach Anordnungsklassen}
	\label{fig:DAGesamtAbsolutAKL}
\end{figure}

\begin{table}[H]
	\caption{Messdaten zur Auswertung des Messfehler nach Anordnungsklassen}
	\centering
	\includegraphics[width=\textwidth, height=\textheight, keepaspectratio]{pics/Tab_DA_Gesamt_Absolut_GKfläche}
	\label{tab:DAGesamtAbsolutAKL}
\end{table}

\noindent Zu erkennen ist das wieder an den farbig markierten Datenpunkte,n die ebenfalls angeben, welcher Algorithmus in der jeweiligen Kategorie den geringsten Wert aufweist. Wie man unschwer erkennen kann, ergibt sich auch hier wieder ein sehr unklares Bild, da der Algorithmus, welcher über aller Bilder den geringsten durchschnittlichen Fehler in dieser Kategorie aufweist, nicht derselbe ist, der in den unterschiedlichen Kalibrierungsstufen und Interpolationsmodi auch zum geringsten Fehler führt - und zwar in keinem einzigen Bereich.
\\\\
Des Weiteren sind ebenfalls hohe Varianzen zu erkennen, wie in folgender Abbildung \ref{fig:DAGesamtStreu1AKL} (i.V.m. Datentabelle \ref{tab:DAGesamtStreu1AKL} dargestellt, wenngleich auch im Unterschied zu den anderen bereits beschriebenen Fehlerbereichen ein etwas andersartiger Zusammenhang vorliegt. Die Streuung in den Daten ist hier über alle Kategorien hinweg etwa gleichbleibend und weist nur in Bereich ab Kalibrierungsstufe 1,1 (bei schrittweiser Skalierung) einen erkennbaren Anstieg auf.

\begin{figure}[H]
	\centering
	\includegraphics[width=\textwidth, height=\textheight, keepaspectratio]{pics/DA_Gesamt_Streu1_AKL}
	\caption{Streuungsparameter (Standardabweichung u. Varianz) nach Anordnungsklassen}
	\label{fig:DAGesamtStreu1AKL}
\end{figure}

\begin{table}[H]
	\caption{Daten zu den berechneten Streuungsparametern (Standardabweichung u. Varianz) nach Anordnungsklassen}
	\centering
	\includegraphics[width=\textwidth, height=\textheight, keepaspectratio]{pics/Tab_DA_Gesamt_Streu1_AKL}
	\label{tab:DAGesamtStreu1AKL}
\end{table}

\noindent Die über alle Kalibrierungsstufen und Interpolationsmodi gemittelte Varianz ist mit einem Wert von ca. 521,82, allerdings deutlich geringer als bspw. verglichen mit dem Wert der Größenklassen (flächengewichtet) von ca. 8.942. Die auf die selbe Weise berechnet Varianz in den Größenklassen (anzahlgewichtet) weist jedoch mit nur ca. 38,54 den geringsten Wert in dieser Kategorie auf. 
\\\\
\noindent Die in Abbildung \ref{fig:DAGesamtStreu2AKL} und Tabelle \ref{tab:DAGesamtStreu2GAKL} dargestellten Variationsbreiten bestätigen diese Zusammenhänge. Jedoch ist hier noch eine Verringerung im Bereich von 1,1 $\nicefrac{\mu m}{\text{Pixel}}$ (bei Vollskalierung) zu erkennen.

\begin{figure}[H]
	\centering
	\includegraphics[width=\textwidth, height=\textheight, keepaspectratio]{pics/DA_Gesamt_Streu2_AKL}
	\caption{Streuungsparameter (Variationsbreite) nach Anordnungsklassen}
	\label{fig:DAGesamtStreu2AKL}
\end{figure}

\begin{table}[H]
	\caption{Daten zum berechneten Streuungsparameter (Variationsbreite) nach Anordnungsklassen}
	\centering
	\includegraphics[width=\textwidth, height=\textheight, keepaspectratio]{pics/Tab_DA_Gesamt_Streu2_AKL}
	\label{tab:DAGesamtStreu2GAKL}
\end{table}

\noindent Bei der Einzelwertbetrachtung (Abbildung \ref{fig:DAEinzelAbsolutAKL}) fällt auf, dass die Fehlerwerte in den Anordnungsklassen A und C jeweils rapide ansteigen und danach in etwa gleich steil wieder sinken, wohingegen zwischen Klasse D und E nur ein vergleichsweise geringer Anstieg zu erkennen ist.

\begin{figure}[H]
	\centering
	\includegraphics[width=\textwidth, height=\textheight, keepaspectratio]{pics/DA_Einzel_Absolut_AKL}
	\caption{Auswertung des Messfehlers (Einzel) nach Anordnungsklassen}
	\label{fig:DAEinzelAbsolutAKL}
\end{figure}

\noindent Abschließend folgt nun wieder eine kurze Zusammenfassung der Ergebnisse aus diesem Abschnitt.
\\\\
\colorbox{gray!10}{
	\label{box:Ergebnis der Auswertung nach Größenklassen (flächengewichtet):}
	\begin{minipage}{0.975\textwidth}
		\textbf{\underline{Ergebnis der Auswertung nach Größenklassen (flächengewichtet):}}
		\\\\
		Es wurde in diesem Teil der Analyse festgestellt, dass die kubische Interpolation mit einem  Fehler von 28,73 einen über alle Bereiche geringsten durchschnittlichen, jedoch aber in keinem Bereich auch tatsächlich den kleinsten absoluten Wert aufweist. Die Streuungsparameter zeigen in allen Bereichen  in etwa gleich hohe Werte für Standardabweichung und Varianz, wobei ab Kalibrierungsstufe 1,1 $\nicefrac{\mu m}{\text{Pixel}}$ (bei schrittweiser Skalierung) ein Anstieg erkennbar ist.
		\\\\
		Der zuvor festgestellte Zusammenhang, dass die Fehler mit zunehmender Ausgangskalibrierung stetig ansteigen, lässt sich hier jedoch nicht eindeutig bestätigen.
	\end{minipage}
}

\chapter{Zusammenfassung und Ausblick}

Das es die Zielsetzung dieser Arbeit ist, die Möglichkeiten zur Entwicklung einer allgemeingültigen Anordnungsklassifikation für Lamellengraphit zu untersuchen, werden nun die im Kapitel \ref{ch:Evaluation} beschriebenen Analyseergebnisse zu einem Gesamtergebnis zusammengefasst und auf der Grundlage dessen eine Gesamtaussage abgeleitet. Das Kapitel schließt dann mit einem Ausblick in Abschnitt \ref{sec:Ausblick}.

\section{Zusammenfassung}
\label{sec:Zusammenfassung}
Bei der vorausgegangenen Analyse wurden in Abschnitt \ref{subsubsec:AuswertungStatVpl} zunächst die Ergebnisse aus der statistischen Versuchsplanung ausgewertet. Dabei wurde bereits deutlich, dass die Faktoren Interpolationsmodus und Kalibrierung den größten Einfluss auf den Messfehler haben.
\\\\
\noindent Daran anschließend wurden  die durch Bildskalierung und Interpolation hervorgerufen Messfehler nach einer Gesamtfehlerbetrachtung in Abschnitt \ref{subsubsec:AuswertungGesamtfehlerbetrachtung} schrittweise in ihre Bestandteile zerlegt und die Fehler in den Größenklassen (anzahl- und flächengewichtet) in den Abschnitten \ref{subsubsec:AuswertungGKanz} und \ref{subsubsec:AuswertungGKflaeche} sowie in den Anordnungsklassen im Abschnitt \ref{subsubsec:AuswertungAKL} untersucht, wobei die Ergebnisse aus den genannten Abschnitten jeweils am Ende kurz zusammengefasst wurden.
\\\\
\noindent Dabei wurde festgestellt, dass ein Vergleich der Algorithmen nicht ohne weiteres möglich ist, da bspw. bei der Gesamtfehlerbetrachtung der nach Größenklassen (anzahlgewichtet) sowie der nach Anordnungsklassen jeweils die LANCZOS4- Interpolation über alle Kategorien den geringsten Messfehler erzeugt, jedoch nicht immer den geringsten Wert in den unterschiedlichen Kalibrierungsstufen. Bei den Größenklassen (flächengewichtet) trifft dasselbe für die kubische Interpolation zu. Das ist insofern problematisch, als dass die Kalibrierung, wie bereits in Abschnitt \ref{subsec:DefEinflusStoer} beschrieben wurde, eine Störgröße darstellt und nicht direkt beeinflusst werden kann.
\\\\
Allerdings ist das Ergebnis auch im Hinblick auf das Auswertungsinteresse eines Metallographen bzw. einer Metallographin zu bewerten, die sich vor allem für die Messergebnisse in den Größenklassen (flächengewichtet) interessieren, da diese Größe wesentlich bedeutendere Rückschlüsse auf die Materialeigenschaften eines Gusseisenwerkstoffes zulässt. Vor diesem Hintergrund wurden in einer weiteren Analyse die Größenklassen in Bezug auf die unterschiedlichen Kalibrierungsstufen noch einmal separat betrachtet und in Abbildung \ref{fig:DAGesamtKalibBenchmark} dargestellt. Daran ist zu erkennen, dass die Fehler um die Standardkalibrierung (0,3 und 0,7 $\nicefrac{\mu m}{\text{Pixel}}$ noch tolerierbar sind, wohingegen sie bei höheren Kalibrierungen über der Benchmark-Linie liegen. Weiterhin ist zu verzeichnen, dass bei einer anzahlgewichteten Messung der Fehler wesentlich höher ausfällt. Dies ist darauf zurückzuführen, dass durch die starke Verringerung der Bildgröße sich die Anzahl der kleinen Strukturen stark verringert und diese somit nicht mehr in der kleinsten Größenklasse erfasst werden können. Die eingezeichneten Benchmark-Linien ergeben sich aus den Mittelwerten in der entsprechenden Größenklassenkategorie und es wird vermutet, dass alle Werte, die unterhalb dieser Linie liegen die Anforderungen an einen Standard-Klassifikator, welche in Abschnitt \ref{sec:DefAnforderungenAnordnKlas} auf Seite \pageref{sec:DefAnforderungenAnordnKlas} beschrieben wurden, zu erfüllen. Allerdings ist die Beantwortung der Frage, was den genau ein tolerierbarer Fehlerwert ist, nicht Teil dieser Arbeit und wurde daher auch nicht untersucht.

\begin{figure}[H]
	\centering
	\includegraphics[width=\textwidth, height=\textheight, keepaspectratio]{pics/DA_Gesamt_GK_Kalib_Benchmark}
	\caption{Gegenüberstellung der Messfehler in den Größenklassen (anzahl- und flächengewichtet) je Kalibrierungsstufe}
	\label{fig:DAGesamtKalibBenchmark}
\end{figure}

\noindent Generell kann somit als Ergebnis in dieser Arbeit festgestellt werden, dass die Anwendung eines Standard-Klassifikators nur empfehlenswert bei Bildern mit hoher Auflösung ist, also $< 1,0 \nicefrac{\mu m}{Pixel}$. Das ist bei heutigen modernen Kameras mit hoher Auflösung immer gegeben, wobei die Kalibrierung bei Bildern die mit älteren Kameramodellen  aufgenommen wurden (z.B. PAL-D mit einem Standard-Signal von 768 $\times$ 576 Pixeln) bei > 1,0 $\nicefrac{\mu m}{\text{Pixel}}$ liegt. 
\\\\
\noindent Außerdem kann gesagt werden, dass sowohl die kubische als auch die LANCZOS4-Interpolation sich am besten zur Bildskalierung bei Anwendung eines Standard-Klassifikators eignen. Und zwar sollte die kubische Interpolation auf Bilder mit einer Ausgangskalibrierung von 0,7 $\nicefrac{\mu m}{\text{Pixel}}$ und die LANCZOS4-Interpolation auf Bilder mit kleineren Kalibrierungen angewendet werden, jeweils bei Voll-Skalierung.

\section{Ausblick}
\label{sec:Ausblick}

Die in dieser Arbeit durchgeführten Untersuchungen haben gezeigt, welche Interpolationsverfahren sich zur Bild-transformation bei der Verwendung eines allgemeingültigen Anordnungsklassifikators zur Durchführung von Lamellengraphit-Messungen mit der Software AMGuss am besten geeignet sind, weil dadurch die geringsten Messfehler erzeugt werden. Allerdings ist eine Bildskalierung nie fehlerfrei und kann es auch nie sein. Die Frage ist nur, wie groß denn der Fehler maximal sein darf, damit die aus Expertensicht geforderten qualitativen Anforderungen an ein Messergebnis erfüllt werden. Daher wäre zur abschließenden Bewertung eine weitere Studie wichtig, in der die Experten, also die Anwender dieser Software in metallographischen Laboren, diese Ergebnisse begutachten und bewerten. Nur auf diese Weise kann letztlich die Frage beantwortet werden, ob die durch Bildtransformation erzeugten Messfehler ein tolerierbares Maß überschreiten oder nicht. 
\\\\
\noindent Abhängig davon, wie diese Bewertungsergebnisse ausfallen wäre es ggf. auch lohnenswert zu untersuchen, ob es unter Anwendung von Methoden des maschinellen Lernens möglich wäre, die Datenverluste bei der Bildtransformation und somit die bei einer Messung mit AMGuss erzeugten Messfehler auf ein tolerierbares Maß hin zu minimieren. Oder stattdessen einen Klassifikationsalgorithmus, der auf alle Bildskalierungen anwendbar ist und man somit auf Bildtransformation verzichten kann.

\newpage

\appendix
\chapter{Inhalt des Projekt-Repositories}

Alle Daten sowie auch der Quellcode zu dieser Arbeit, befinden sich auf
\\\\ 
- \url{https://github.com/MichaelKaip/lamello-stat}.
\\\\
Ein Zugang zu diesem Repository kann, sofern dieser noch nicht vorliegt, unter
\\\\ 
- \href{mailto:michael.kaip@outlook.com}{\nolinkurl{michael.kaip@outlook.com}} 
\\\\
beantragt werden.

\section{Daten}
Diese befinden sich im o.g. Repository im Unterverzeichnis
\\\\ 
- \textbf{/resources/evaluation\_data.xslx}.

\section{Quellcode}
Dieser befindet sich im o.g. Repository im Unterverzeichnis
\\\\ 
- \textbf{/src}
\\\\
mit folgendem Inhalt:
\begin{enumerate}\bfseries
	\item amguss\_runner.py
	\item amguss\_runner\_helper.py
	\item doe\_stats\_job.py
	\item evaluation\_job.py
	\item image.py
	\item img\_set\_creation\_job.py
	\item job\_helper.py
	\item lamello\_stat\_main.py
	\item lamello\_stat\_main\_helper.py
	\item monitoring\_container.py
	\item parsing\_job.py
	\item refresh\_runner.py
	\item scaling\_job.py
	\item scaling\_train\_images\_job.py
	\item writing\_data\_to\_excel\_job.py
\end{enumerate}

\section{Dokumentation}
Eine vollständige Dokumentation des Quellcodes befindet sich im Unterverzeichnis
\\\\ 
- \textbf{/docs}
\\\\
Diese ist als Webseite verfügbar und kann in jedem verfügbaren Browser angezeigt werden, indem die Datei \textbf{index.html} im Unterverzeichnis
\\\\
- \textbf{/html}
\\\\
dort geöffnet wird.


\bibliographystyle{IEEEtran}
\bibliography{literature}



\end{document}