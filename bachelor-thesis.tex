\documentclass[
fontsize=10pt, 
listof = totoc,
parskip = half	
]{report}
\usepackage[utf8]{inputenc}
\usepackage[ngerman]{babel}
\usepackage[dvipsnames]{xcolor}
\usepackage[most]{tcolorbox}
\usepackage[]{acronym}
\usepackage[T1]{fontenc}
\usepackage[absolute]{textpos}
\usepackage{amsmath}
\usepackage{amsfonts}
\usepackage{tabu}
\usepackage{graphicx}
\usepackage{lipsum}  
\usepackage{float}
\usepackage{esvect}
\usepackage{booktabs} % table style
\usepackage{tabularx}
%\usepackage{longtable} % table style
\usepackage{multirow} % table style
\usepackage{amssymb}
\usepackage{nicefrac}
\usepackage{cancel}
\usepackage{polynom}
\usepackage{stmaryrd} 
\usepackage{caption}
\usepackage{subcaption}
\usepackage{paralist}



\newenvironment{tableEnum}{\begin{varwidth}[t]{\linewidth}\begin{compactenum}[1)]}{\end{compactenum}\end{varwidth}\vspace{0.08cm}}
\newenvironment{tableItem}{\begin{varwidth}[t]{\linewidth}\begin{compactitem}[-]}{\end{compactitem}\end{varwidth}\vspace{0.08cm}}

\usepackage[left=2cm,right=2cm,top=2cm,bottom=2cm]{geometry}
\author{Michael Kaip}

\date{}

% Extension for amsmath matrix environment - matrix | vector
\makeatletter
\renewcommand*\env@matrix[1][*\c@MaxMatrixCols c]{%
	\hskip -\arraycolsep
	\let\@ifnextchar\new@ifnextchar
	\array{#1}}
\makeatother

%Extension for roman numbers
\newcommand{\uproman}[1]{\uppercase\expandafter{\romannumeral#1}}
\newcommand{\lowroman}[1]{\romannumeral#1\relax}

%%%%%%%%%%%%%%%%%%%%%%%%%%%%%%%%%%%%%%%%

\begin{document}
	\begin{titlepage}
		\vspace*{-\headsep}\vspace{-\headheight}
		\noindent
		\includegraphics[scale=0.38]{logo}
		\hfill
		\textbf{GFaI-Logo hier...}\\[-1ex]
		\rule{\linewidth}{1pt}
		
		\vfill\vfill
		
		\begin{center}
			Bachelorarbeit
			\begin{huge}
				\\[2ex]
				Untersuchung der Möglichkeiten zur Entwicklung \\ einer allgemeingültigen Anordnungsklassifikation \\ von Lamellengraphit
				\\[6ex]
			\end{huge}
			Vorgelegt von:
			\\[2ex]
			\begin{huge}
				Michael Kaip
			\end{huge}
			\\[2ex]
			Studiengang Ingenieurinformatik
			\\[28ex]
			Erstgutachter:
			\\[2ex]
			Prof. Dr.-Ing. Mohammad Abuosba
			\\[4ex]
			Zweitgutachter:
			\\[2ex]
			Dipl.-Mathematiker Ulrich Sonntag
			\\[40ex]
			Berlin, den XX. April 2021
		\end{center}
	

	\end{titlepage}
	
\clearpage

\begingroup
\pagestyle{empty}
\null
\newpage
\endgroup

\pagenumbering{gobble}

\begin{abstract}
	\lipsum[1-4]   
\end{abstract}

\newpage
\tableofcontents
\newpage
\pagenumbering{Roman}
\listoffigures
\addcontentsline{toc}{chapter}{Abbildungsverzeichnis}
\newpage
\listoftables
\addcontentsline{toc}{chapter}{Tabellenverzeichnis}
\newpage
\pagenumbering{arabic}
\newpage

\chapter{Einleitung}

\section{Motivation}

\section{Problemstellung}

\newpage

\chapter{Grundlagen}

\section{Art und Umfang der einzelnen Aufgaben}

\section{Angewandte Methoden}

\section{Arbeitsergebnisse}

\section{Grad der Selbständigkeit / Weisungsgebundenheit}

\section{Art und Umfang der Unterstützung durch die Ausbildungsstelle}












\end{document}