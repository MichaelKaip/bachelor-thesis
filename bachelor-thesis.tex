\documentclass[
fontsize=10pt, 
listof = totoc,
parskip = half	
]{report}
\usepackage[utf8]{inputenc}
\usepackage[german]{babel}
\usepackage[fixlanguage]{babelbib}
\selectbiblanguage{german}
\usepackage[dvipsnames, table]{xcolor}
\usepackage{colortbl}
\usepackage[most]{tcolorbox}
\usepackage[]{acronym}
\usepackage[T1]{fontenc}
\usepackage[absolute]{textpos}
\usepackage{amsmath}
\usepackage{amsthm}
\usepackage{amsfonts}
\usepackage[ruled,vlined,linesnumbered,ngerman]{algorithm2e}
\usepackage{tabu}
\usepackage{graphicx}
\usepackage{mdframed}
\usepackage{float}
\usepackage{lipsum}
\usepackage{fontawesome}  
\usepackage{esvect}
\usepackage{booktabs} % table style
\usepackage{tabularx}
\usepackage{multirow} % table style
\usepackage{amssymb}
\usepackage{nicefrac}
\usepackage{cancel}
\usepackage{polynom}
\usepackage{stmaryrd} 
\usepackage{caption}
\usepackage{subcaption}
\usepackage{paralist}
% for graphics
\usepackage{physics}
\usepackage{amsmath}
\usepackage{tikz}
\usepackage{mathdots}
\usepackage{yhmath}
\usepackage{cancel}

\usepackage{chngcntr}
\counterwithout{figure}{chapter}
\counterwithout{table}{chapter}

\usepackage[hidelinks]{hyperref}


\newenvironment{tableEnum}{\begin{varwidth}[t]{\linewidth}\begin{compactenum}[1)]}{\end{compactenum}\end{varwidth}\vspace{0.08cm}}
\newenvironment{tableItem}{\begin{varwidth}[t]{\linewidth}\begin{compactitem}[-]}{\end{compactitem}\end{varwidth}\vspace{0.08cm}}

\usepackage[left=2cm,right=2cm,top=2cm,bottom=2cm]{geometry}
\author{Michael Kaip}

\date{}

% Extension for amsmath matrix environment - matrix | vector
\makeatletter
\renewcommand*\env@matrix[1][*\c@MaxMatrixCols c]{%
	\hskip -\arraycolsep
	\let\@ifnextchar\new@ifnextchar
	\array{#1}}
\makeatother

%Extension for roman numbers
\newcommand{\uproman}[1]{\uppercase\expandafter{\romannumeral#1}}
\newcommand{\lowroman}[1]{\romannumeral#1\relax}

\newtheorem{definition}{Definition}


%%%%%%%%%%%%%%%%%%%%%%%%%%%%%%%%%%%%%%%%

\begin{document}
\begin{titlepage}
	\vspace*{-\headsep}\vspace{-\headheight}
	\noindent
	\includegraphics[scale=0.38]{logo}
	\hfill
	\textcolor{white}{placeholder}\\[-1ex]
	\rule{\linewidth}{1pt}
	
	\vfill\vfill
	
	\begin{center}
		Bachelorarbeit
		\begin{huge}
			\\[2ex]
			Untersuchung der Möglichkeiten zur Entwicklung \\ einer allgemeingültigen Anordnungsklassifikation \\ von Lamellengraphit
			\\[6ex]
		\end{huge}
		Vorgelegt von:
		\\[2ex]
		\begin{huge}
			Michael Kaip
		\end{huge}
		\\[2ex]
		Studiengang Ingenieurinformatik
		\\[28ex]
		Erstgutachter:
		\\[2ex]
		Prof. Dr.-Ing. Mohammad Abuosba
		\\[4ex]
		Zweitgutachter:
		\\[2ex]
		Dipl.-Mathematiker Ulrich Sonntag
		\\[40ex]
		Berlin, den XX. April 2021
	\end{center}
\end{titlepage}
	
\clearpage

\begingroup
\pagestyle{empty}
\null
\newpage
\endgroup

\pagenumbering{gobble}

\begin{abstract}
	\lipsum[1-4]   
\end{abstract}

\newpage
\tableofcontents
\newpage
\pagenumbering{Roman}
\listoffigures
\addcontentsline{toc}{chapter}{Abbildungsverzeichnis}
\newpage
\listoftables
\addcontentsline{toc}{chapter}{Tabellenverzeichnis}
\newpage
\pagenumbering{arabic}
\newpage

\chapter{Einleitung}
\label{ch:Einleitung}

\boxed{\textcolor{blue}{\textbf{2-3 Seiten}}}
\\\\
\textcolor{blue}{\textbf{(1) Was ist Gusseisen und welche wirtschaftliche Bedeutung hat der Werkstoff?}}
\\\\
\textcolor{blue}{\textbf{(2) Wie wird Gusseisen hergestellt und wie entstehen dabei unterschiedliche Graphitstrukturen überhaupt?}}
\\\\
\textcolor{blue}{\textbf{(3) Welche Typen von Gusseisen gibt es?}}
\\\\
\textcolor{blue}{\textbf{(4) Welche Bedeutung hat die Qualität des Gusseisenwerkstoffes mit Bezug auf unterschiedliche typische Verwendungszwecke - mit Beispiel?}}
\\\\
\textcolor{blue}{\textbf{(5) Warum ist eine korrekte Klassifikation von Gusseisenwerkstoffen wichtig?}}
\\\\
\textcolor{blue}{\textbf{(6) Welche Graphitmorphologien müssen bei normgerechter Klassifizierung berücksichtigt werden?}}
\\\\
\textcolor{blue}{\textbf{(6) Wie erfolgt die Klassifikation heute in der Praxis und welche Nachteile sind damit verbunden?}}
\\\\
\textcolor{blue}{\textbf{(7) Was sind die Vorteile digitaler Bildverarbeitung in diesem Kontext und wo liegen die Grenzen?}}

\newpage

\chapter{Grundlagen}
\label{ch:Grundlagen}

\section{Graphitklassifizierung}
\label{sec:Graphitklassifizierung}

\subsection{Lamellengraphit}
\label{subsec:Lamellengraphit}

\subsection{Mechanische Eigenschaften von Lamellengraphit}
\label{subsec:MechanischeEigenschaften}

\subsection{Einteilung von Gusseisen mit Lamellengraphit entsprechend den mechanischen\\ Eigenschaften}
\label{subsec:EinteilungLamellengraphit}

\section{Grundlagen der Bildverarbeitung}
\label{GrundlagenBildverarbeitung}

\subsection{Bildrepräsentation und Farbräume}
\label{Bildrep}

\subsection{Skalierung und Interpolationsverfahren}
\label{subsec:SkalierungUndInterpolation}

\section{Beschreibung der verwendeten Methoden zur statistischen Analyse}
\label{sec:MethodenStatAnalyse}



\chapter{Ausgangssituation}
\label{ch:Ausgangssituation}

\section{Metallographie und Analytik}
\label{sec:MetallographieAnalytik}

\subsection{Lichtmikroskopie für die Erstellung von Bildproben}
\label{subsec:Lichtmikroskopie}

\subsection{Bildmaterial}
\label{subsec:Bildmaterial}

\section{Bestimmung der Mikrostruktur von Gusseisen mit AMGuss}
\label{sec:BestimmungMikrostrukturAMGuss}

\subsection{Kalibrierung}
\label{subsec:Kalibrierung}

\subsection{Erstellung eines Anordnungsklassifikators für die Lamellengraphit-Auswertung}
\label{subsec: ErstellungAnordnungsklassifAMGuss}

\subsection{Methoden zur Bestimmung der Anordnungstypen A-E von Lamellengraphit}
\label{subsec:AnordnungstypenLamellengraphit}

\subsection{Bewertungsergebnisse einer Lamellengraphit-Auswertung}
\label{subsec:ErgebnisseAMGuss}

\section{Problemstellung}
\label{sec:Problemstellung}

Bei einem allgemeingültigen Anordnungsklassifikator müsste der Nutzer lediglich die Kalibrierung angeben, mit der die Probenbilder aufgenommen wurden und das System wäre in der Lage, die Kalibrierung der Bilder automatisch an die eines im System hinterlegten Klassifikators durch Skalierung anzupassen. Somit würde der Arbeitsschritt, Klassifikatoren manuell erstellen und  verwalten zu müssen, entfallen. Fehler könnten dadurch vermieden und eine Einheitlichkeit der Messungen sichergestellt werden.

\chapter{Konzept}
\label{ch:Konzept}

\section{Sollzustand/Anforderungen}
\label{sec:DefAnforderungenAnordnKlas}
Die Vorgehensweise zur Erstellung eines Anordnungsklassifikators in Kapitel \ref{subsec: ErstellungAnordnungsklassifAMGuss} bereits beschrieben. Dies ist für den Nutzer mit einem nicht unerheblichen Aufwand verbunden. Hinzu kommt eine gewisse Fehleranfälligkeit, da für jede Messung der in Bezug auf die Bildkalibrierung richtige Klassifikator für die Messung ausgewählt werden muss.
\\\\
Das Gütekriterium an einen solchen Klassifikator ist, die durch Skalierung (bzw. Interpolation) hervorgerufenen und in Kapitel \ref{sec:Problemstellung} bereits näher beschriebenen Messfehler auf ein \textbf{tolerierbares} Maß hin zu minimieren. Allerdings gibt es jedoch, nach den aktuellen allgemein anerkannten Regeln der Technik (vgl. dazu \textcolor{blue}{auf Norm verweisen}) keinen eindeutigen objektiven Maßstab, der zur Beurteilung angelegt werden könnte. Stattdessen beruht die Graphitklassifizierung auf einer visuellen Einschätzung der Spezialisten, welche die Beurteilung der Proben vornehmen. Die Norm DIN ISO 945-1 definiert dabei die Grundlagen, auf denen eine solche Beurteilung zu erfolgen hat. Was also als noch tolerierbar gilt, entscheidet der versierte Nutzer in gewissen Grenzen selbst und wie die Erfahrungen zeigen, existieren teils nicht unerhebliche Abweichungen bei der Einschätzung. 

\section{Erzeugung von Bildern mit unterschiedlichen Ausgangskalibrierungen}


\section{Statistische Versuchsplanung}


\noindent\textcolor{red}{todo:  beschreiben...}  

\subsection{Systemanalyse}
Die Aufgabe, einen allgemeingültigen Anordnungsklassifikator zu entwickeln, der die beschriebenen Anforderungen erfüllt, ist im Grunde die Lösung eines Optimierungsproblems. Dabei ist es erforderlich zu untersuchen, welche Abhängigkeiten zwischen den \textbf{Einflussgrößen} und Zielgrößen zu bestehen, was zunächst vereinfacht in Abbildung \ref{fig:BlackBox} dargestellt is.

\tikzset{every picture/.style={line width=0.75pt}} %set default line width to 0.75pt        

\begin{figure}[h]
	\centering
	\begin{tikzpicture}[x=0.75pt,y=0.75pt,yscale=-1,xscale=1, scale=0.7, every node/.style={scale=0.7}]
		%uncomment if require: \path (0,222); %set diagram left start at 0, and has height of 222
		
		%Flowchart: Process [id:dp31407792865997985] 
		\draw  [fill={rgb, 255:red, 155; green, 155; blue, 155 }  ,fill opacity=0.45 ][line width=2.25]  (147,92) -- (356,92) -- (356,186.35) -- (147,186.35) -- cycle ;
		%Right Arrow [id:dp008081388677440904] 
		\draw  [fill={rgb, 255:red, 248; green, 231; blue, 28 }  ,fill opacity=1 ][line width=1.5]  (199,14) -- (199,56) -- (209,56) -- (189,84) -- (169,56) -- (179,56) -- (179,14) -- cycle ;
		%Right Arrow [id:dp584717372432416] 
		\draw  [fill={rgb, 255:red, 248; green, 231; blue, 28 }  ,fill opacity=1 ][line width=1.5]  (315,14) -- (315,56) -- (325,56) -- (305,84) -- (285,56) -- (295,56) -- (295,14) -- cycle ;
		%Right Arrow [id:dp8933598607947869] 
		\draw  [fill={rgb, 255:red, 184; green, 233; blue, 134 }  ,fill opacity=1 ][line width=1.5]  (70,130) -- (112,130) -- (112,120) -- (140,140) -- (112,160) -- (112,150) -- (70,150) -- cycle ;
		%Right Arrow [id:dp43249460780861937] 
		\draw  [fill={rgb, 255:red, 184; green, 233; blue, 134 }  ,fill opacity=1 ][line width=1.5]  (363,129) -- (405,129) -- (405,119) -- (433,139) -- (405,159) -- (405,149) -- (363,149) -- cycle ;
		%Shape: Rectangle [id:dp02676070817238141] 
		\draw  [line width=3.75]  (5.68,-39.65) -- (514.68,-39.65) -- (514.68,210.35) -- (5.68,210.35) -- cycle ;
		
		% Text Node
		\draw (156,-24) node [anchor=north west][inner sep=0.75pt]  [font=\normalsize] [align=left] {{\scriptsize \textbf{Steuergrößen }}};
		% Text Node
		\draw (278,-24) node [anchor=north west][inner sep=0.75pt]   [align=left] {{\scriptsize \textbf{Störgrößen}}};
		% Text Node
		\draw (266,224.4) node [anchor=north west][inner sep=0.75pt]    {$$};
		% Text Node
		\draw (151,-6.6) node [anchor=north west][inner sep=0.75pt]  [font=\scriptsize]  {$\{x_{1} ,\ x_{2} ,\ ...,\ x_{n}\}$};
		% Text Node
		\draw (267,-6.6) node [anchor=north west][inner sep=0.75pt]  [font=\scriptsize]  {$\{v_{1} ,\ v_{2} ,\ ...,\ v_{n}\}$};
		% Text Node
		\draw (439,127) node [anchor=north west][inner sep=0.75pt]  [font=\scriptsize] [align=left] {\textbf{Zielgrößen}};
		% Text Node
		\draw (439,137.4) node [anchor=north west][inner sep=0.75pt]  [font=\scriptsize]  {$y_{1} ,\ y_{2} ,\ ...,\ y_{3}$};
		% Text Node
		\draw (13,136) node [anchor=north west][inner sep=0.75pt]  [font=\scriptsize] [align=left] {\textbf{Eingaben}};
		% Text Node
		\draw (177,129) node [anchor=north west][inner sep=0.75pt]  [font=\scriptsize] [align=left] {\begin{minipage}[lt]{108.95844000000001pt}\setlength\topsep{0pt}
				\begin{center}
					\textbf{Versuchsraum}\\\textbf{Ursache-/Wirkunsbez.}
				\end{center}
		\end{minipage}};
	\end{tikzpicture}
	\label{fig:BlackBox}
	\caption{Usache-/Wirkungsbeziehungen als Black-Box-Modell}
\end{figure}

\noindent Daher werden im Folgenden sowohl die Ziel-/ und Einflussgrößen eingehend beschrieben.

\subsection{Definition der Zielgrößen}

\textcolor{blue}{Hinweis: Mehrgrößenoptimierungsproblem $\to$ Kombination der Einzelwerte zu einer gewichteten Summe...}



\subsection{Definition der Einflussgrößen}


\subsection{Modellbildung}

\subsection{Versuchsplanaufbau}

\section{Untersuchung der Auswirkungen von Bildskalierungen auf die Reproduzierbarkeit von Messergebnissen unter Anwendung verschiedener Interpolationsverfahren}

\subsection{Messung der Bilder mit AMGuss vor und nach der Skalierung}

\subsection{Anwendung \textit{einstufiger} Skalierungen auf die erzeugten Bilder}

\subsection{Anwendung \textit{mehrstufiger} Skalierungen auf die erzeugten Bilder}




\chapter{Umsetzung}

\section{Rahmenbedingungen}

\subsection{Technologie-Stack}

\section{Implementierung}

\subsection{Modellierung und algorithmische Beschreibung der Implementierung}

\section{Verwendung von Bildern mit verschiedenen Ausgangskalibrierungen}

\section{Skalierung der erzeugten Bilder (einstufig/mehrstufig)}

\subsection{Bilineare Interpolation}

\subsection{Bikubische Interpolation}

\subsection{Flächenbasierte Interpolation}

\subsection{Nearest-Neighbor-Interpolation}

\subsection{LANCZOS-Interpolation}



\chapter{Evaluierung}

\section{Vergleich angewendeten Interpolationsverfahren}

\subsection{Laufzeitkomplexität und Performance}

\subsection{Kennzahlenvergleich und Interpretation}




\chapter{Zusammenfassung und Auswertung}
\section{Zusammenfassung}
\section{Auswertung}




\end{document}