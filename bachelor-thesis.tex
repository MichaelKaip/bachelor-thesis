\documentclass[
fontsize=10pt, 
listof = totoc,
parskip = half	
]{report}
\usepackage[utf8]{inputenc}
\usepackage[german]{babel}
\usepackage[fixlanguage]{babelbib}
\selectbiblanguage{german}
\usepackage[dvipsnames, table]{xcolor}
\usepackage{colortbl}
\usepackage[most]{tcolorbox}
\usepackage[]{acronym}
\usepackage[T1]{fontenc}
\usepackage[absolute]{textpos}
\usepackage{amsmath}
\usepackage{amsfonts}
\usepackage[ruled,vlined,linesnumbered,ngerman]{algorithm2e}
\usepackage{tabu}
\usepackage{graphicx}
\usepackage{mdframed}
\usepackage{float}
\usepackage{lipsum}
\usepackage{fontawesome}  
\usepackage{esvect}
\usepackage{booktabs} % table style
\usepackage{tabularx}
\usepackage{multirow} % table style
\usepackage{amssymb}
\usepackage{nicefrac}
\usepackage{cancel}
\usepackage{polynom}
\usepackage{stmaryrd} 
\usepackage{caption}
\usepackage{subcaption}
\usepackage{paralist}

\usepackage{chngcntr}
\counterwithout{figure}{chapter}
\counterwithout{table}{chapter}

\usepackage[hidelinks]{hyperref}


\newenvironment{tableEnum}{\begin{varwidth}[t]{\linewidth}\begin{compactenum}[1)]}{\end{compactenum}\end{varwidth}\vspace{0.08cm}}
\newenvironment{tableItem}{\begin{varwidth}[t]{\linewidth}\begin{compactitem}[-]}{\end{compactitem}\end{varwidth}\vspace{0.08cm}}

\usepackage[left=2cm,right=2cm,top=2cm,bottom=2cm]{geometry}
\author{Michael Kaip}

\date{}

% Extension for amsmath matrix environment - matrix | vector
\makeatletter
\renewcommand*\env@matrix[1][*\c@MaxMatrixCols c]{%
	\hskip -\arraycolsep
	\let\@ifnextchar\new@ifnextchar
	\array{#1}}
\makeatother

%Extension for roman numbers
\newcommand{\uproman}[1]{\uppercase\expandafter{\romannumeral#1}}
\newcommand{\lowroman}[1]{\romannumeral#1\relax}

%%%%%%%%%%%%%%%%%%%%%%%%%%%%%%%%%%%%%%%%

\begin{document}
\begin{titlepage}
	\vspace*{-\headsep}\vspace{-\headheight}
	\noindent
	\includegraphics[scale=0.38]{logo}
	\hfill
	\textcolor{white}{placeholder}\\[-1ex]
	\rule{\linewidth}{1pt}
	
	\vfill\vfill
	
	\begin{center}
		Bachelorarbeit
		\begin{huge}
			\\[2ex]
			Untersuchung der Möglichkeiten zur Entwicklung \\ einer allgemeingültigen Anordnungsklassifikation \\ von Lamellengraphit
			\\[6ex]
		\end{huge}
		Vorgelegt von:
		\\[2ex]
		\begin{huge}
			Michael Kaip
		\end{huge}
		\\[2ex]
		Studiengang Ingenieurinformatik
		\\[28ex]
		Erstgutachter:
		\\[2ex]
		Prof. Dr.-Ing. Mohammad Abuosba
		\\[4ex]
		Zweitgutachter:
		\\[2ex]
		Dipl.-Mathematiker Ulrich Sonntag
		\\[40ex]
		Berlin, den XX. April 2021
	\end{center}
\end{titlepage}
	
\clearpage

\begingroup
\pagestyle{empty}
\null
\newpage
\endgroup

\pagenumbering{gobble}

\begin{abstract}
	\lipsum[1-4]   
\end{abstract}

\newpage
\tableofcontents
\newpage
\pagenumbering{Roman}
\listoffigures
\addcontentsline{toc}{chapter}{Abbildungsverzeichnis}
\newpage
\listoftables
\addcontentsline{toc}{chapter}{Tabellenverzeichnis}
\newpage
\pagenumbering{arabic}
\newpage

\chapter{Einleitung}
\label{ch:Einleitung}

\section{Motivation}
\label{sec:Motivation}

\section{Problemstellung}
\label{sec:Problemstellung}

\section{Aufbau der Arbeit}
\label{sec:AufbauDerArbeit}

\newpage

\chapter{Grundlagen}
\label{ch:Grundlagen}

\section{Graphitklassifizierung}
\label{sec:Graphitklassifizierung}

\subsection{Lamellengraphit}
\label{subsec:Lamellengraphit}

\subsection{Mechanische Eigenschaften von Lamellengraphit}
\label{subsec:MechanischeEigenschaften}

\subsection{Einteilung von Gusseisen mit Lamellengraphit entsprechend den mechanischen\\ Eigenschaften}
\label{subsec:EinteilungLamellengraphit}

\section{Grundlagen der Bildverarbeitung}
\label{GrundlagenBildverarbeitung}

\subsection{Bildrepräsentation und Farbräume}
\label{Bildrep}

\subsection{Skalierung und Interpolationsverfahren}
\label{subsec:SkalierungUndInterpolation}

\section{Beschreibung der verwendeten Methoden zur statistischen Analyse}
\label{sec:MethodenStatAnalyse}



\chapter{Ausgangssituation}
\label{ch:Ausgangssituation}

\section{Metallographie und Analytik}
\label{sec:MetallographieAnalytik}

\subsection{Lichtmikroskopie für die Erstellung von Bildproben}
\label{subsec:Lichtmikroskopie}

\subsection{Bildmaterial}
\label{subsec:Bildmaterial}

\section{Bestimmung der Mikrostruktur von Gusseisen mit AMGuss}
\label{sec:BestimmungMikrostrukturAMGuss}

\subsection{Kalibrierung}
\label{subsec:Kalibrierung}

\subsection{Erstellung eines Anordnungsklassifikators für die Lamellengraphit-Auswertung}
\label{subsec: Erstellung AnordnungsklassifAMGuss}

\subsection{Methoden zur Bestimmung der Anordnungstypen A-E von Lamellengraphit}
\label{subsec:AnordnungstypenLamellengraphit}

\subsection{Bewertungsergebnisse einer Lamellengraphit-Auswertung}

\label{subsec:ErgebnisseAMGuss}




\chapter{Konzept}
\label{ch:Konzept}

\section{Definition der Anforderungen an einen allgemeingültigen\\ Anordnungsklassifikator}

\section{Erzeugung von Bildern mit unterschiedlichen Ausgangskalibrierungen}

\section{Untersuchung der Auswirkungen von Bildskalierungen auf die Reproduzierbarkeit von Messergebnissen unter Anwendung verschiedener Interpolationsverfahren}

\subsection{Messung der Bilder mit AMGuss vor und nach der Skalierung}

\subsection{Anwendung \textit{einstufiger} Skalierungen auf die erzeugten Bilder}

\subsection{Anwendung \textit{mehrstufiger} Skalierungen auf die erzeugten Bilder}

\section{Fehleranalyse}

\subsection{Entwicklung eines statischen Kennzahlensystems}

\subsection{Ermittlung der Laufzeitkomplexitäten und Performancemessung}



\chapter{Umsetzung}

\section{Rahmenbedingungen}

\subsection{Technologie-Stack}

\section{Implementierung}

\subsection{Modellierung und algorithmische Beschreibung der Implementierung}

\section{Erzeugung von Bildern mit verschiedenen Ausgangskalibrierungen}

\section{Skalierung der erzeugten Bilder (einstufig/mehrstufig)}

\subsection{Bilineare Interpolation}

\subsection{Bikubische Interpolation}

\subsection{Flächenbasierte Interpolation}

\subsection{Nearest-Neighbor-Interpolation}

\subsection{LANCZOS-Interpolation}



\chapter{Evaluierung}

\section{Vergleich angewendeten Interpolationsverfahren}

\subsection{Laufzeitkomplexität und Performance}

\subsection{Kennzahlenvergleich und Interpretation}




\chapter{Zusammenfassung und Auswertung}
\section{Zusammenfassung}
\section{Auswertung}




\end{document}